% % % % % % % % % % % % % % % % % % % % % % % % % % % % %
% Vorlage zur Erstellung einer Monographie/Dissertation %
% Bastian Wimmer 2013  Compiler: PdfLaTeX & BibTeX      %
% % % % % % % % % % % % % % % % % % % % % % % % % % % % %

%Setzen des Dokumententyps
\documentclass[12pt,a4paper]{article}

%Einbinden der nötigen Pakete, hier gilt eigentlich FINGER WEG! Sprache wird allerdings in dieser Datei umgestellt bei Babel
\usepackage[a4paper,left=35mm,width=145mm,top=20mm]{geometry}
%Automatische Verzeichnisse ins Inhaltsverzeichnis
\usepackage{tocbibind}

\usepackage[utf8]{inputenc} % Kodierung
\usepackage[ngerman]{babel} % deutsche Sprache
\usepackage{acronym}
%Verwendete Schrift z.B. lmodern
%\usepackage{lmodern}
%\usepackage{libertine}
\usepackage{mathptmx} % Für Times New Roman Font
%\usepackage{mathpazo,eulervm}
%\usepackage{tgpagella,eulervm}
%\usepackage[charter]{mathdesign}
%\usepackage[osf,sc]{mathpazo}
%\usepackage[sc]{mathpazo}
%\usepackage{
%lmodern, 
%hfoldsty 
% charter
%}

%Sprachenbezogene Pakete
%\usepackage[english]{babel} % englische Sprache
%\usepackage[babel, german=guillemets]{csquotes}
\usepackage[babel, german=quotes]{csquotes} %Deutsche Anführungszeichen

\usepackage[T1]{fontenc}
\usepackage{eurosym}
% Das € Zeichen kann nun wirklich mit € angezeigt werden
\DeclareUnicodeCharacter{20AC}{\euro}
%Die Pakete für den Index
\usepackage{makeidx}
\makeindex
%Diverse Standardpakete
\usepackage[
		pdftex,
		pdfpagelabels=true,
	 	citecolor=black,
       	filecolor=blue,
      	urlcolor=black,
       	bookmarks=true,
       	bookmarksopen=true,
       	bookmarksopenlevel=3,
       	plainpages=false,
       	pdfpagelabels=true,
       	pdfborder={0 0 0},
       	breaklinks=true,]{hyperref}
\hypersetup{colorlinks=true,
			breaklinks=true,
			urlcolor=black,
			linkcolor=black,
			menucolor=black,
			% Diese Angaben kommen in die PDF Eigenschaften
			pdftitle    = {Bachelorthesis - Leon Morten Richter},
			pdfsubject  = {Bachelorthesis},
			pdfauthor   = {Leon Morten Richter},
			pdfkeywords = {Gamification, Serious Games, Commandline, bash},
			pdfcreator  = {TeXStudio, MacTex \& Me},
			pdfproducer = {LaTeX with Hyperref}}
\usepackage[hypcap]{caption}

%Zitierweise im APA-Stil
\usepackage{cite} % Zitate - MUSS VOR apacite eingebunden werden
\usepackage{apacite} % Apacite Style einbinden

%%Zeilenabstand
\usepackage{setspace}
%\setstretch{1,2381}
\onehalfspacing

% Das braucht TODO, damit es keine Warnung gibt 
\setlength {\marginparwidth }{2cm} 

% Pakete für Grafik, Tabellen u.a.
\usepackage{verbatim} %für multiline comments
%\usepackage[nonumberlist, acronym, toc, section]{glossaries}
\usepackage{textcomp}
\usepackage{graphicx} % support the \includegraphics command and options
\usepackage{booktabs}
\usepackage{array} % for better arrays (eg matrices) in maths
\usepackage{subfig} % make it possible to include more than one captioned figure/table in a single float
\usepackage{multicol}
\usepackage{multirow}
\usepackage{ifthen}
\usepackage{float}
\usepackage{csquotes}
\usepackage{tabularx}
\newcolumntype{C}[1]{>{\centering\arraybackslash}m{#1}} 
\newcolumntype{L}[1]{>{\raggedright\arraybackslash}m{#1}}
\newcolumntype{R}[1]{>{\raggedleft\arraybackslash}m{#1}}
\usepackage{paralist}
\usepackage{siunitx}
\sisetup{
add-decimal-zero = false,
add-integer-zero = false,
}

%Befehl für die Linien auf der Titelseite
\newcommand{\HRule}{\rule{\linewidth}{0.2mm}}
%Absatz einrücken
%Legt die Einrücktiefe der ersten Zeile nach Überschrift fest
\usepackage{indentfirst}

% Horiz. Linien in Tabellen verbessern
\newcommand{\forloop}[5][1]{%
\setcounter{#2}{#3}%
\ifthenelse{#4}{#5\addtocounter{#2}{#1}%
\forloop[#1]{#2}{\value{#2}}{#4}{#5}}%
{}}
\newcounter{crcounter}
\newcommand{\compensaterule}[1]{%
\forloop{crcounter}{1}{\value{crcounter} < #1}%
{\vspace*{-\aboverulesep}\vspace*{-\belowrulesep}}}
\newcommand{\multirowbt}[3]{\multirow{#1}{#2}%
{\compensaterule{#1}#3}}

%Captions left
\usepackage[font=footnotesize,labelfont=bf,singlelinecheck=false,format=plain,,justification=justified,indention=0cm]{caption}

% Misc
\usepackage[colorinlistoftodos]{todonotes} % für die guten TODOS





%Beginn des Hauptdokuments
\begin{document}
%Titelseite, im entsprechenden Dokument anpassen
\begin{titlepage}
\begin{center}


% Title
{ \Large \bfseries
Analyse der Wirksamkeit der Spielelemente Abzeichen und Fortschrittsanzeige hinsichtlich Motivation und Leistung im Kontext der Kommandozeile
}\\[2.5cm]

% Type of thesis
\textsc{ Bachelorarbeit}\\[2.0cm]

% Meta information 
\textsc{ 
im Studiengang Wirtschaftsinformatik \\
der technischen Fakultät \\
der Christian-Albrechts-Universität zu Kiel \\
im Sommersemester 2020 
}\\[2.5cm]

% Author
\textsc{ 
vorgelegt von \\
Leon Morten Richter (1105170)
}\\[2.5cm]

% Move that shit to the bottom
\vfill

% Supervisors
\begin{flushleft}
    \begin{tabbing}
        Erstgutachterin: \=  TODO \\
        Zweitgutachter: \> TODO \\[1.25cm]
    \end{tabbing}
\textsc{Kiel, den \today}
\end{flushleft}

% Bottom of the page
\end{center}
\end{titlepage}

% Inhaltsverzeichnis mit kleinen römschen Seitenzahlen, ``Roman'' für echte römische Zahlen
\pagenumbering{roman}
%Inhalt
\tableofcontents
\newpage
%Tabellenverzeichnis
\listoftables
\newpage
%Abbildungsverzeichnis
\listoffigures
\newpage

%Abkürzungen
\section*{Abkürzungsverzeichnis}
\begin{acronym}[Bash]
 \acro{KDE}{K Desktop Environment}
 \acro{SQL}{Structured Query Language}
 \acro{Bash}{Bourne-again shell}
 \acro{JDK}{Java Development Kit}
 \acro{VM}{Virtuelle Maschine}
 \acro{I2C}[I²C]{Inter-Integrated Circuit}
\end{acronym}
\addcontentsline{toc}{section}{Abkürzungsverzeichnis}
\clearpage

%Zurück zur arabischen Seitennummerierung
\pagenumbering{arabic}

%Den Seitenzähler auf 1 zurückstellen für den Hauptteil
\setcounter{page}{1}

%Die Kapitel der Arbeit mit Include einbinden, Verzeichnisse richtig anpassen bitte
\section{Einführung}
Hello Writer \cite[S. 320]{finsterwald_fostering_2013}. Hello \LaTeX{} is great! \cite{finsterwald_fostering_2013} \cite{ziegler_talent_2013}.
Lorem ipsum dolor sit amet, consectetur adipisici elit, sed eiusmod tempor incidunt ut labore et dolore magna aliqua. Ut enim ad minim veniam, quis nostrud exercitation ullamco laboris nisi ut aliquid ex ea commodi consequat. Quis aute iure reprehenderit in voluptate velit esse cillum dolore eu fugiat nulla pariatur. Excepteur sint obcaecat cupiditat non proident, sunt in culpa qui officia deserunt mollit anim id est laborum. \cite[S.34]{finsterwald_fostering_2013}. Laut \index{Soziallehre ! allgemein} Lorem ipsum dolor sit amet, consectetur adipisici elit, sed eiusmod tempor incidunt ut labore et dolore magna aliqua. Ut enim ad minim veniam, quis nostrud exercitation ullamco laboris nisi ut aliquid ex ea commodi consequat. Quis aute iure reprehenderit in voluptate velit esse cillum dolore eu fugiat nulla pariatur. Excepteur sint obcaecat cupiditat non proident, sunt in culpa qui officia deserunt mollit anim id est laborum. \cite[S.34]{finsterwald_fostering_2013} öäüß 10 €
\subsection{Erster Punkt}
Lorem ipsum dolor sit amet, consectetur adipisici elit, sed eiusmod tempor incidunt ut labore et dolore magna aliqua. Ut enim ad minim veniam, quis nostrud exercitation ullamco laboris nisi ut aliquid ex ea commodi consequat. Quis aute iure reprehenderit in voluptate velit esse cillum dolore eu fugiat nulla pariatur. Excepteur sint obcaecat cupiditat non proident, sunt in culpa qui officia deserunt mollit anim id est laborum. \cite{zeichner_teaching_2000}. Lorem ipsum dolor sit amet, consectetur adipisici elit, sed eiusmod tempor incidunt ut labore et dolore magna aliqua. Ut enim ad minim veniam, quis nostrud exercitation ullamco laboris nisi ut aliquid ex ea commodi consequat. Quis aute iure reprehenderit in voluptate \cite{kitsantas_college_2007}.

\subsection{Verschiedene Zitationen}
% (Pinker, 2008)
\cite{abrami_using_2013}

% (Author, p. 2)
\cite[p. 2]{heinrich_preparation_2007}

% in Text citation
\citeauthor{pintrich_motivational_1990}

% 2007, p. 2 , evtl in Klammern im Text
\citeyear[p. 2]{alshammari_meta-analysis_2013}

% im Text: Author (year)
\citeauthor{ziegler_hochbegabung_2008} \citeyear{ziegler_hochbegabung_2008}

\subsection{Tabelle}
%Example Table, wenn keine Fussnoten benötigt werden footnote entfernen und die beiden Zeilen mit Minipage auskommentieren
\begin{table}[htb]
\begin{minipage}{\linewidth}
\renewcommand{\footnoterule}{}
\renewcommand{\thefootnote}{\alph{footnote}}
\caption{Einige experimentelle Zahlen.}
\label{tab:tab1}
\centering
\begin{tabular}{lcc}
\toprule
          & \multicolumn{2}{c}{Factor 2} \\ 
          	\cmidrule{2-3}
Factor 1  & Condition A  & Condition B \footnote{Eine weitere Fussnote}  \\ 
\midrule
First     & 586 (231)    & 649 (255)     \\
          &    2.2       &    7.5        \\
Second    & 590 (195) \footnote{Dies ist eine Fussnote innerhalb der Tabelle}   & 623 (231)     \\
          &    2.8       &    2.5        \\ 
\bottomrule
\end{tabular}
\end{minipage}
\end{table}
%enquote setzt deutsche Anführungszeichen.
Damit wäre das \enquote{Monster} Tabelle entzaubert.
Nun eine Formel.
%Dafür eigenet sich der Daum Equation Editor als Google Chrome Erweiterung hervorragend.Achtung: In einer Formelumgebung geht kein''caption!''
\begin{equation}
\left( \frac { \sum _{ f }^{ 1 }{ 5i }  }{ { \infty  }^{ 2 } }  \right) \log { f^{ x }\left( x \right)  } 
\label{eq:Equ01}
\end{equation}

%Example Enumeration
\begin{enumerate}
	 \item first item ... ... ...
     \item second item ... ... ...
\end{enumerate}

Lorem ipsum dolor sit amet, consectetur adipisici elit, sed eiusmod tempor incidunt ut labore et dolore magna aliqua. Ut enim ad minim veniam, quis nostrud exercitation ullamco laboris nisi ut aliquid ex ea commodi consequat. Quis aute iure reprehenderit in voluptate velit esse cillum dolore eu fugiat nulla pariatur. Excepteur sint obcaecat cupiditat non proident, sunt in culpa qui officia deserunt mollit anim id est laborum.
Hier sieht man eine Tabelle, die Fußnoten beinhaltet:
Eine weitere Tabelle sieht man auf Seite \pageref{tab:tab2}, diesmal mit Fussnoten in der Tabelle, die mit Minipage ermöglicht werden.
\begin{table}[htb]
\begin{minipage}{\linewidth}
\renewcommand{\footnoterule}{}
\renewcommand{\thefootnote}{\alph{footnote}}
\caption{Einige experimentelle Zahlen.}
\label{tab:tab2}
\centering
\begin{tabular}{lcc}
\toprule
          & \multicolumn{2}{c}{Factor 2} \\ 
          	\cmidrule{2-3}
Factor 1  & Condition A  & Condition B \footnote{Eine weitere Fussnote}  \\ 
\midrule
First     & 586 (231)    & 649 (255)     \\
          &    2.2       &    7.5        \\
Second    & 590 (195) \footnote{Dies ist eine Fussnote innerhalb der Tabelle}   & 623 (231)     \\
          &    2.8       &    2.5        \\ 
\bottomrule
\end{tabular}
\end{minipage}
\end{table}

% Leerzeile bedeutet neuer Absatz
\index{Stress} Stress Lorem ipsum dolor sit amet, consectetur adipisici elit, sed eiusmod tempor incidunt ut labore et dolore magna aliqua. Ut enim ad minim veniam, quis nostrud exercitation ullamco laboris nisi ut aliquid ex ea commodi consequat. Quis aute iure reprehenderit in voluptate velit esse cillum dolore eu fugiat nulla pariatur. Excepteur sint obcaecat cupiditat non proident, sunt in culpa qui officia deserunt mollit anim id est laborum. Glaubenslehre \index{Glaubenslehre} Lorem ipsum dolor sit amet, consectetur adipisici elit, sed eiusmod tempor incidunt ut labore et dolore magna aliqua. Ut enim ad minim veniam, quis nostrud exercitation ullamco laboris nisi ut aliquid ex ea commodi consequat. Quis aute iure reprehenderit in voluptate velit esse cillum dolore eu fugiat nulla pariatur. Excepteur sint obcaecat cupiditat non proident, sunt in culpa qui officia deserunt mollit anim id est laborum.Soziallehre \index{Soziallehre ! liturgisch} Lorem ipsum dolor sit amet, consectetur adipisici elit, sed eiusmod tempor incidunt ut labore et dolore magna aliqua. Ut enim ad minim veniam, quis nostrud exercitation ullamco laboris nisi ut aliquid ex ea commodi consequat. Quis aute iure reprehenderit in voluptate velit esse cillum dolore eu fugiat nulla pariatur. Excepteur sint obcaecat cupiditat non proident, sunt in culpa qui officia deserunt mollit anim id est laborum..

\section{Zweiter Punkt}
Lorem ipsum dolor sit amet, consectetur adipisici elit, sed eiusmod tempor incidunt ut labore et dolore magna aliqua. Ut enim ad minim veniam, quis nostrud exercitation ullamco laboris nisi ut aliquid ex ea commodi consequat. Quis aute iure reprehenderit in voluptate

%\include{./subdocs/kapitel02}
%\include{./subdocs/kapitel03}
%\include{./subdocs/kapitel04}
%\include{./subdocs/kapitel05}
%\include{./subdocs/kapitel06}
%\include{./subdocs/kapitel07}
%\include{./subdocs/kapitel08}
%\include{./subdocs/kapitel09}
%\include{./subdocs/kapitel10}
%Seitenumbruch für die folgende Bibliografie
\clearpage
%%Bibliografie, bitte richtigen Bib-Dateinamen eintragen
\bibliography{./subdocs/biblio}
\bibliographystyle{apacite}

%Appendix falls benötigt, hier ebenfalls die Kapitel einfügen
\appendix
\section*{Appendix}
\begin{table}[htb]
\begin{minipage}{\linewidth}
\renewcommand{\footnoterule}{}
\renewcommand{\thefootnote}{\alph{footnote}}
\caption{Einige experimentelle Zahlen.}
\label{tab:tab1}
\centering
\begin{tabular}{lcc}
\toprule
          & \multicolumn{2}{c}{Factor 2} \\ 
          	\cmidrule{2-3}
Factor 1  & Condition A  & Condition B \footnote{Eine weitere Fussnote}  \\ 
\midrule
First     & 586 (231)    & 649 (255)     \\
          &    2.2       &    7.5        \\
Second    & 590 (195) \footnote{Dies ist eine Fussnote innerhalb der Tabelle}   & 623 (231)     \\
          &    2.8       &    2.5        \\ 
\bottomrule
\end{tabular}
\end{minipage}
\end{table}
%\include{./subdocs/appendixB}
\addcontentsline{toc}{section}{Appendix}
\newpage

%Index falls gewünscht. Nicht vergessen den <Index Befehl> laufen zu lassen, ansonsten die folgenden Zeilen mit % auskommentieren bis end
% Index soll Stichwortverzeichnis heissen
\renewcommand{\indexname}{Stichwortverzeichnis} 
%Achtung bei englischer Dokumentsprache

% Stichwortverzeichnis soll im Inhaltsverzeichnis auftauchen
%\addcontentsline{toc}{section}{Stichwortverzeichnis}
\newpage
% Ausgabe des Index
\printindex
\end{document}
