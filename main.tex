% % % % % % % % % % % % % % % % % % % % % % % % % % % % %
% Bachelorthesis                                        %
% Leon Morten Richter  Compiler: PdfLaTeX & BibTeX      %
% % % % % % % % % % % % % % % % % % % % % % % % % % % % %

%Setzen des Dokumententyps
\documentclass[
    12pt,
    a4paper,
]{article}

%Einbinden der nötigen Pakete, hier gilt eigentlich FINGER WEG! Sprache wird allerdings in dieser Datei umgestellt bei Babel
\usepackage[a4paper,left=35mm,width=145mm,top=20mm]{geometry}
%Automatische Verzeichnisse ins Inhaltsverzeichnis
\usepackage{tocbibind}
%	\renewcommand{\tableofcontents}{\begingroup
%		\tocsection
%		\tocchapter
%		\tocfile{\contentsname}{toc}
%	\endgroup}
%	\renewcommand{\listoffigures}{\begingroup
%    	\tocsection
%        \tocchapter
%    	\tocfile{\listfigurename}{lof}
% 	\endgroup}
%	\renewcommand{\listoftables}{\begingroup
%    	\tocsection
%    	\tocfile{\listtablename}{lot}
% 	\endgroup}

\usepackage[utf8]{inputenc} % Kodierung
\usepackage[ngerman]{babel} % deutsche Sprache
\usepackage{acronym}
%Verwendete Schrift z.B. lmodern
%\usepackage{lmodern}
%\usepackage{libertine}
%\usepackage{mathpazo,eulervm}
%\usepackage{tgpagella,eulervm}
%\usepackage[charter]{mathdesign}
\usepackage[osf,sc]{mathpazo}
%\usepackage[sc]{mathpazo}
%\usepackage{
%lmodern, 
%hfoldsty 
% charter
%}


%Sprachenbezogene Pakete
%\usepackage[english]{babel} % englische Sprache
%\usepackage[babel, german=guillemets]{csquotes}
\usepackage[babel, german=quotes]{csquotes} %Deutsche Anführungszeichen

\usepackage[T1]{fontenc}
\usepackage{eurosym}
% Das € Zeichen kann nun wirklich mit € angezeigt werden
\DeclareUnicodeCharacter{20AC}{\euro}
%Die Pakete für den Index
\usepackage{makeidx}
\makeindex
%Diverse Standardpakete
\usepackage[
		pdftex,
		pdfpagelabels=true,
	 	citecolor=black,
       	filecolor=blue,
      	urlcolor=black,
       	bookmarks=true,
       	bookmarksopen=true,
       	bookmarksopenlevel=3,
       	plainpages=false,
       	pdfpagelabels=true,
       	pdfborder={0 0 0},
       	breaklinks=true,]{hyperref}
\hypersetup{colorlinks=true,
			breaklinks=true,
			urlcolor=black,
			linkcolor=black,
			menucolor=black,
			% Diese Angaben kommen in die PDF Eigenschaften
			pdftitle    = {Help-Seeking},
			pdfsubject  = {Dissertation},
			pdfauthor   = {Bastian Wimmer},
			pdfkeywords = {Keyword 1,Keyword 2, Keyword 3},
			pdfcreator  = {TeXStudio, MacTex \& Me},
			pdfproducer = {LaTeX with Hyperref}}
\usepackage[hypcap]{caption}

%Zitierweise im APA-Stil
\usepackage{cite} % Zitate - MUSS VOR apacite eingebunden werden
\usepackage{apacite} % Apacite Style einbinden

%%Zeilenabstand
\usepackage{setspace}
%\setstretch{1,2381}
\onehalfspacing

% Pakete für Grafik, Tabellen u.a.
\usepackage{verbatim} %für multiline comments
%\usepackage[nonumberlist, acronym, toc, section]{glossaries}
\usepackage{textcomp}
\usepackage{graphicx} % support the \includegraphics command and options
\usepackage{booktabs}
\usepackage{array} % for better arrays (eg matrices) in maths
\usepackage{subfig} % make it possible to include more than one captioned figure/table in a single float
\usepackage{multicol}
\usepackage{multirow}
\usepackage{ifthen}
\usepackage{float}
\usepackage{csquotes}
\usepackage{tabularx}
\newcolumntype{C}[1]{>{\centering\arraybackslash}m{#1}} 
\newcolumntype{L}[1]{>{\raggedright\arraybackslash}m{#1}}
\newcolumntype{R}[1]{>{\raggedleft\arraybackslash}m{#1}}
\usepackage{paralist}
\usepackage{siunitx}
\sisetup{
add-decimal-zero = false,
add-integer-zero = false,
}

%Befehl für die Linien auf der Titelseite
\newcommand{\HRule}{\rule{\linewidth}{0.2mm}}
%Absatz einrücken
%Legt die Einrücktiefe der ersten Zeile nach Überschrift fest
\usepackage{indentfirst}

% Horiz. Linien in Tabellen verbessern
\newcommand{\forloop}[5][1]{%
\setcounter{#2}{#3}%
\ifthenelse{#4}{#5\addtocounter{#2}{#1}%
\forloop[#1]{#2}{\value{#2}}{#4}{#5}}%
{}}
\newcounter{crcounter}
\newcommand{\compensaterule}[1]{%
\forloop{crcounter}{1}{\value{crcounter} < #1}%
{\vspace*{-\aboverulesep}\vspace*{-\belowrulesep}}}
\newcommand{\multirowbt}[3]{\multirow{#1}{#2}%
{\compensaterule{#1}#3}}

%Captions left
\usepackage[font=footnotesize,labelfont=bf,singlelinecheck=false,format=plain,,justification=justified,indention=0cm]{caption}






%Beginn des Hauptdokuments
\begin{document}
%Titelseite, im entsprechenden Dokument anpassen
\begin{titlepage}
\begin{center}


% Title
{ \Large \bfseries
Analyse der Wirksamkeit der Spielelemente Abzeichen und Fortschrittsanzeige hinsichtlich Motivation und Leistung im Kontext der Kommandozeile
}\\[2.5cm]

% Type of thesis
\textsc{ Bachelorarbeit}\\[2.0cm]

% Meta information 
\textsc{ 
im Studiengang Wirtschaftsinformatik \\
der technischen Fakultät \\
der Christian-Albrechts-Universität zu Kiel \\
im Sommersemester 2020 
}\\[2.5cm]

% Author
\textsc{ 
vorgelegt von \\
Leon Morten Richter (1105170)
}\\[2.5cm]

% Move that shit to the bottom
\vfill

\todo[inline]{Korrekte Namen}
% Supervisors
\begin{flushleft}
    \begin{tabbing}
        Erstgutachterin: \=  TODO\\
        Zweitgutachter: \> TODO \\[1.25cm]
    \end{tabbing}
\textsc{Kiel, den \today}
\end{flushleft}

% Bottom of the page
\end{center}
\end{titlepage}

% Inhaltsverzeichnis mit kleinen römschen Seitenzahlen, ``Roman'' für echte römische Zahlen
\pagenumbering{roman}
%Inhalt
\tableofcontents
\newpage
%Tabellenverzeichnis
\listoftables
\newpage
%Abbildungsverzeichnis
\listoffigures
\newpage

%Abkürzungen
\input{./subdocs/shortcuts}

%Zurück zur arabischen Seitennummerierung
\pagenumbering{arabic}

%Den Seitenzähler auf 1 zurückstellen für den Hauptteil
\setcounter{page}{1}

%Die Kapitel der Arbeit mit Include einbinden, Verzeichnisse richtig anpassen bitte
%\section{Abstract}
Gamification bezeichnet die Verwendung von Spieldesignelementen in einem spielfremden Kontext. Das Ziel dieser Arbeit war es, zu prüfen, ob die Spielelemente Abzeichen und Fortschrittsbalken motivierend auf die Nutzer einer Kommandozeile wirken.
Dazu wurde eine virtuelle Kommandozeile um die genannten Designelemente erweitert. Die so entstandene Webanwendung bildete die Basis für ein Quiz, dessen Fragen durch typische Konsolenbefehle zu beantworten waren.
Auf Grundlage dieser Anwendung wurde eine quantitative Onlinestudie mit 278 Teilnehmern durchgeführt. Eine Betrachtung der Stichprobe als Ganzes zeigt zunächst keinen signifikanten Unterschied zwischen den Versuchsbedingungen und einer Kontrollgruppe hinsichtlich Gesamtspielzeit und Anzahl gelöster Aufgaben. Jedoch zeigt eine weiterführende Analyse der Daten, dass erfahrene Nutzer mittleren Alters, die einen Fortschrittsbalken sahen, im Mittel statistisch signifikant mehr Aufgaben lösten als eine Kontrollgruppe.
Die Ergebnisse beleuchten das schwierige Zusammenspiel verschiedenster Faktoren, die für Motivation verantwortlich sind und deuten auf einen motivierenden Effekt der Fortschrittsanzeige bei erfahrenen Konsolenanwendern hin. 
%\section{Einleitung}
Digitale Fähigkeiten spielen auf dem Arbeitsmarkt eine zunehmend wichtigere Rolle. Ein in Großbritannien durchgeführter Report kommt etwa zu dem Ergebnis, dass 82\% sämtlicher Stellenausschreibungen im Netz digitale Kenntnisse erfordern \cite{no_longer_optional}. Gleichzeitig erhalten Stellen, die digitale Fähigkeiten erfordern, im Mittel ein um 29\% höheres Gehalt als Stellen, die keinerlei entsprechende Anforderungen listen. Zusätzlich senken digitale Fertigkeiten das Risiko für Jobverlust durch Automatisierung um 59\% und wirken positiv auf die individuelle Karriereentwicklung. Unter dem Sammelbegriff digitale Fähigkeiten fallen dabei unterschiedlichste Fertigkeiten, die sich je nach Anforderungsprofil unterscheiden. Diese reichen von simplen EDV-Kenntnissen, etwa der Bedienung von Microsoft-Word, bis hin zu Kenntnissen in der Systemadministration oder Softwareentwicklung. Für die Letztgenannten ist das Beherrschen der Kommandozeile unerlässlich. Diese ist Teil eines jeden Betriebssystems und stellt, als rein textbasierte Schnittstelle, die einfachste Form der Mensch-Computer-Interaktion dar \cite{Kumar2005}. Damit unterscheidet sich die Kommandozeile hinsichtlich Bedienung und Arbeitsweise grundlegend von einer graphischen Oberfläche. Das Erlernen ebendieser Schnittstelle kann anfänglich mit einem hohen Maß an Frustration verbunden sein und stellt damit eine Hürde dar, die es für Anfänger zu überwinden gilt.

Das Überwinden entsprechender Hürden fällt leichter, wenn die Eigenmotivation möglichst hoch ist. Im Kontext der Kommandozeile ist dies insbesondere dann der Fall, wenn sich die Lernenden durch die erworbenen Fähigkeiten verbesserte Einkommensmöglichkeiten und gesteigerte Karrieremöglichkeiten versprechen. Neben solchen monetären Anreizen gibt es auch nicht-monetäre Anreize wie Anerkennung, Selbstverwirklichung oder Beschäftigungssicherheit \cite{shujaat}. 

Eine weitere Möglichkeit zur Erhöhung der Motivation ist Gamification. Mit dem Begriff Gamification wir der Einsatz von Spieldesignelementen in einem spielfremden Kontext bezeichnet \cite{deterding_game_2011}. Häufig verwendete Spieldesignelemente sind Abzeichen, Rankings, Levels und Fortschrittsanzeigen \cite{koch2013gamification}. Die Wirksamkeit dieser Spielelemente wurde bereits mehrfach empirisch nachgewiesen \cite{koivisto_rise_2019}. Jedoch gibt es zum aktuellen Zeitpunkt keinerlei wissenschaftliche Daten für die Wirkungsweise im Kontext einer Kommandozeile. Aus diesem Grund werde ich im Rahmen dieser Arbeit den Einfluss der Spielelemente Abzeichen und Fortschrittsanzeige auf Motivation und Leistung im Kontext der Kommandozeile analysieren. Dazu entwerfe ich eine interaktive Webanwendung, die einer klassischen Kommandozeile gleicht und integriere die genannten Spielelemente. 
%-------------------------------------------------------------------------------
% Theorie
%-------------------------------------------------------------------------------
\section{Theoretischer und empirischer Hintergrund}

Unter dem Begriff Gamification wird der Einsatz von Spielmechaniken und Spieldesignelementen in einem spielfremden Kontext verstanden \cite{deterding_game_2011}. Typische Beispiele für solche Spielelemente sind Abzeichen, Forschrittsanzeigen, Levels oder Ranglisten \cite{koch2013gamification}. Praktische Beispiele sind die Lernplattform Duolingo, die dem Erlernen von Sprachen dient, oder der Social-News-Aggregator Reddit. Im Fall von Duolingo schaltet der Nutzer neue Lektionen als Belohnung für den Abschluss bisheriger Lektionen frei. Reddit integriert ein Punktesystem. Nutzer können etwa durch das regelmäßige Posten qualitativer Beiträge \qq{Karmapunkte} sammeln. Zudem ist es möglich Beiträge mit einem \qq{Down-Vote} oder einem \qq{Up-Vote} zu bewerten.
Ziel ist es, dass sich Nutzer freiwillig mehr agieren, um ihr Ansehen in der Community zu steigern.

Die Wirksamkeit von Gamification konnte bereits für unterschiedlichste Einsatzgebiete empirisch nachgewiesen werden \cite{koivisto_rise_2019} und hat sich in der Forschung zur Mensch-Computer-Interaktion etabliert \cite{huotari_defining_2012}. 
Bereiche, in denen die Wirksamkeit von Gamification gezeigt werden konnte, sind Forschung/ Open Science \cite{brauer_erhohung_2019,kidwell_badges_2016}, Gesundheit und Fitness \cite{johnson_gamification_2016}, Marketing \cite{huotari_defining_2012} oder Produktion und Logistik \cite{warmelink_gamification_2018}.

%-------------------------------------------------------------------------------
% Abzeichen
%-------------------------------------------------------------------------------
\subsection{Spielelement Abzeichen}
Im Kontext der Gamification bezeichnen Abzeichen digitale, visuelle Artefakte, die dem Nutzer für die Erfüllung definierter Aufgaben verliehen werden \cite{antin_badges_2011}. \citeauthor{antin_badges_2011} unterteilen Abzeichen anhand ihrer sozialpsychologische Funktion in fünf Kategorien: Zielsetzung (Goal setting), Anweisung (Instruction), Reputation (Reputation),
Status/Bestätigung (Status / Affirmation) und Gruppenidentifizierung (Group Identification). 

\paragraph{Zielsetzung}
Abzeichen fordern den Nutzer heraus ein bestimmtes Ziel zu erreichen. Individuen streben selbst dann nach dem Erreichen bestimmter Ziele oder Meilensteine wenn sie selber mit keinerlei physische Gegenleistung erfahren. Die in einem Abzeichen dargestellten Ziele sind nicht immer explizit. Dies passiert wenn nur angegeben ist, wie man ein Abzeichen verdient, oder weil die notwendigen Aktivitäten subjektiv oder unpräzise definiert sind.

\paragraph{Anweisung}
Abzeichen können Anweisungen darüber geben, welche Arten von Aktivität innerhalb eines gegebenen Systems möglich sind. Diese
Funktion ist nützlich, um neue Benutzer in eine bestimmte Richtung zu lenken, aber auch um bestehenden Nutzer neue Wege zu offenbaren. Dabei ist es nicht notwendig, dass Nutzer die Abzeichen tatsächlich erreichen. Alleine das Betrachten der verfügbaren Abzeichen gibt dem Nutzer Aufschluss über geschätzte Aktivitäten.

\paragraph{Reputation}
Abzeichen liefern eine Grundlage für die Reputationsbewertungen einzelner Nutzer. Durch die Kapselung von Interessen, Fachwissen und vergangener Interaktionen helfen Abzeichen bei der Beurteilung der Reputation auf verschiedenen Ebenen. So geben einsehbare Abzeichen einen Einblick in nutzerspezifische Interessen, bieten einen Überblick über Fähigkeiten und Fachwissen und dienen der Einschätzung der Vertrauenswürdigkeit und der Zuverlässigkeit eines Nutzers. 

\paragraph{Status/Bestätigung}
Abzeichen wirken als Statussymbol. Sie verdeutlichen die eigene Leistung und bisherige Errungenschaften ohne explizite Prahlerei.
Die Macht der Statusbelohnungen ergibt sich insbesondere aus der Erwartung, dass andere positiv auf jemanden reagieren, der durch außerordentliche Aktivitäten entsprechende Abzeichen errungen hat. Abzeichen sind auch eine persönliche Bestätigung, da sie an vergangene Errungenschaften erinnern. Sie markieren wichtige Meilensteine und sind ein Beweis für vergangene Erfolge. Sie wirken damit auf eine ähnliche Art und Weise wie Trophäen.

\paragraph{Gruppenidentifizierung}
Abzeichen kommunizieren eine Reihe von gemeinsamen Aktivitäten, die eine Gruppe von Benutzern anhand gemeinsamer Erfahrungen zusammenschweißen. Das Erlangen von Abzeichen kann ein Gefühl der Solidarität vermitteln und die positive Gruppenidentifikation durch die Wahrnehmung der Ähnlichkeit zwischen einem Individuum und der Gruppe erhöhen.\\

Die Wirksamkeit von Abzeichen ist dabei mehrfach empirisch nachgewiesen. Wie \citeauthor{hamari_badges_2017} zeigen konnte, führt die Einführung von Abzeichen zu einer signifikanten Steigerung der Nutzerinteraktion. Einen ähnlich positiver Effekt auf das Lernverhalten von Studenten konnte durch \citeauthor{hamzah_influence_2015} nachgewiesen werden.


%-------------------------------------------------------------------------------
% Fortschrittsbalken
%-------------------------------------------------------------------------------
\subsection{Spielelement Fortschrittsbalken}
Das Spielelement Fortschrittsbalken ist ein etabliertes visuelles Spielelement.
Dabei handelt es sich um ein visuelles Anzeigeelement, das den aktuellen Fortschritt eines Auftrags in Form eines Balkens repräsentiert. Ein klassisches Beispiel ist etwa die Darstellung des Fortschritts einer Installation oder eines Ladevorgangs. Im wesentlichen kann man Fortschrittsanzeigen in zwei Kategorien einteilen. Zum einen gibt es bestimmte Anzeigen, die den absoluten Fortschritt  eines Vorgangs optisch wiedergibt. Die Anzeige gibt somit Aufschluss auf die Restdauer des Vorgangs. Häufig findet sich dabei eine zusätzliche Prozentangabe. Auf der anderen Seite existieren unbestimmte Fortschrittsanzeigen, die keinerlei Rückschluss auf den aktuellen Fortschritt erlauben. Dies kann beispielsweise durch einen Teilbalken, der sich fortwährend von links nach rechts bewegt, realisiert werden.

Trotz der Popularität dieses Anzeigeelements, gibt weniger Forschungsergebnisse bezüglich der Wirksamkeit als bei anderen Spielelementen \cite{koivisto_rise_2019}.
\citeauthor{olsson_visualisation_2016} kommenin ihrer Arbeit zu dem Ergebnis, dass ein Fortschrittsbalken eine geeignete Maßnahme ist,um den Überblick der Kursteilnehmer in Online-Umgebungen zu verbessern. Gleichzeitig weisen die Autoren auf einen möglichen positiven Einfluss von Abzeichen auf die Motivation der Teilnehmer hin. Fortschrittsbalken lösen ein Gefühl der Vollendung aus und vermitteln daher ein Gefühl der Zufriedenheit \cite{ryan_deci_2000}.


%-------------------------------------------------------------------------------
% Kritik und Einschränkungen
%-------------------------------------------------------------------------------
\subsection{Grenzen}
Trotz der empirisch belegten Wirksamkeit der genannten Spielmechaniken, sind die Ergebnisse nicht immer eindeutig und positiv.
So konnten \cite{ortiz_gamification_2017} durch den Einsatz von Abzeichen zwar eine statistisch signifikante Verbesserung des Engagements der Lernenden feststellen.
Allerdings war keine Verbesserung der Leistung und intrinsischen Motivation erkennbar.
\cite{toda_dark_2018} weißt sogar auf die Gefahr einer Reduzierung der Leistung, unerwünschter Seiteneffekte und potentiellem Motivationsverlust hin.
Zu ähnlichen Ergebnissen kommen \cite{liu_examining_2017}.
So verzeichneten Umfragen mit Fortschrittsbalken geringere Abschlussquoten als Umfragen ohne Fortschrittsindikatoren. \cite{dominguez_gamifying_2013} stellten in einem pädagogischen Kontext fest, dass Abzeichen zwar einen positiven Einfluss auf praktische Aufgaben haben, aber potentiell negativ auf schriftliche Abgaben wirken. Der Einsatz von Spielmechaniken ist damit möglicherweise mit unerwünschten Effekten auf Leistung, Motivation und Engagement verbunden.

Bei der Beurteilung der Wirkung und Effektivität von Gamification werden sehr häufig verschiedene Designelemente kombiniert. \citeauthor{mazarakis2018gamification} weist darauf hin, dass es nicht möglich ist, die tatsächliche Wirksamkeit individueller Spielelemente zu beurteilen. Es sei nahezu unmöglich den tatsächlichen Beitrag einzelner Elemente an einer Motivationssteigerung zu messen.

%-------------------------------------------------------------------------------
% Kritik und Einschränkungen
%-------------------------------------------------------------------------------
\subsection{Serious Games}
Serious Games sind Spiele, die nicht ausschließlich der reinen Unterhaltung dienen, sondern primär auf die Vermittlung von Wissen abzielen \cite[S.17]{michael_serious_2005}.
Dabei kommt es darauf an, dass die Spiele mit der Absicht kreiert wurden, dem Spieler einen Lerninhalt zu vermitteln.
Dagegen ist nicht entscheidend, ob der Spieler das Spiel als Lerninhalt versteht oder als reine Unterhaltung \cite[S.3]{bopp_serious_2009}.
Serious Games lassen sich grob in die Kategorien Educational  Games, 
Corporate  Games,  Health  Games,  Persuasive  Games,  Music  Games  sowie  Virtual  Worlds  und 
Mobile Learning Games unterteilen \cite[S.4]{bopp_serious_2009}.
Für diese Arbeit ist lediglich die Kategorie der Educational  Games relevant.
Bei dieser Art der Serious Games geht geht es um den pädagogischen Einsatz von Videospielen.
Charakteristisch für Educational  Games ist, dass die Lernerfahrung ein spezifisches Ziel verfolgt \cite{nielsen_overview_2006, bopp_serious_2009}.
Typischerweise besteht das Ziel in der Vermittlung bestimmter Fähigkeiten, wie Algebra, Rechtschreibung und weiteren Grundfertigkeiten.
Derartige Spiele fallen unter den Oberbegriff Edutainment \cite{nielsen_overview_2006}.
Diese Art der Spiele bietet den Spielern befriedigende Aufgaben, die zu der Entwicklung von neuen Fähigkeiten und Strategien führt \cite{stapleton_serious_2004}.
\citeauthor{vlachopoulos_effect_2017} bestätigen die umfassende empirische Evidenz in Bezug auf kognitive Lernergebnisse einschließlich Wissenserwerb, konzeptuelle Anwendung und Inhaltsverständnis.

%-------------------------------------------------------------------------------
% Gamification und Bildung
%-------------------------------------------------------------------------------

\subsection{Gamification und Bildung}
Ein aktives Forschungsfeld ist der Einsatz von Spielmechaniken in der Bildung \cite{ibanez_gamification_2014,landers_enhancing_2017}. Da der Lernerfolg wesentlich durch Motivation, Interesse und Engagement der Lernenden bedingt ist \cite{astin_student_1984}, sind Spielmechaniken ein vielversprechendes Mittel, um die Leistung der Lernenden zu erhöhen.
Dieser Bereich zählt neben Gesundheit und Fitness zu den empirisch am besten untersuchten Anwendungsgebieten in der Gamificationforschung \cite{koivisto_rise_2019}.
So wurden positive Effekte auf Motivation und Leistung mehrfach der Praxis nachgewiesen \cite{ibanez_gamification_2014,hamzah_influence_2015,strmecki_gamification_2015}. \cite{layth_khaleel_empirical_2019} konnten eine Verbesserung der Lernleistung im Kontext von Programmieraufgaben feststellen. Zu ähnlichen Ergebnissen kommen \cite{ortiz_gamification_2017}. Anzumerken ist jedoch, dass Letztere zwar eine Verbesserung des Engagements feststellen konnten, allerdings keine Auswirkungen auf die Motivation erkennbar war.
%%-------------------------------------------------------------------------------
% Methoden
%-------------------------------------------------------------------------------
\section{Methoden}

Um herauszufinden, welchen Einfluss die Spielmechaniken Abzeichen und Fortschrittsbalken auf die Motivation und die Leistung haben, wird eine quantitative Studie durchgeführt. Diese besteht aus einer interaktiven, browser-basierten Konsolenanwendung, die das Verhalten einer nativen Kommandozeile imitiert. Im Rahmen des Experiments wird den Teilnehmenden eine definierte Menge an Fragen gestellt, die durch Konsolenbefehle zu lösen sind. Die Teilnehmenden werden randomisiert einer von drei Versuchsbedingungen zugeordnet. Um externe Motivationseffekte zu vermeiden, ist die Anwendung in ihrer Grundform absichtlich demotivierend gestaltet und stellt auf diese Weise einen spielfremden Kontext dar.

\subsection{Zielgruppe}
Die Zielgruppe besteht aus Personen, die Berührungspunkte mit der Kommandozeile haben oder Interesse an der Kommandozeile zeigen. Die Fragen sind daher auch für Anfänger lösbar. Zusätzlich wird eine optionale Hilfsfunktion integriert, die hilfreiche Tipps bereitstellt. Damit richtet sich die Anwendung an Informatik-affine Personen, Studenten eines MINT-Faches, Entwickler, Systemadministratoren und Informatiker und setzt grundsätzliche Englischkenntnisse voraus. Durch die englische Lokalisierung der Anwendung wird ein breites Publikum angesprochen, wodurch auch Personen auf das Experiment aufmerksam werden sollten, die keinerlei deutsche Sprachkenntnisse aufweisen. Eine Lokalisation in unterschiedliche Sprachen ist aufgrund des begrenzten Zeitfensters dieser Arbeit nicht erfolgt.

\subsection{Versuchsablauf}
Das Experiment findet ausschließlich online statt. Um die Anzahl möglicher Teilnehmer zu maximieren, ist die Teilnahme von einem beliebigen Endgerät möglich. Dies schließt die Nutzung der Anwendung auf Smartphones und Tablets mit ein. Das Experiment ist außerdem beliebig unterbrechbar und kann jederzeit zu einem späteren Zeitpunkt fortgesetzt werden. In diesem Fall wird der Fortschritt im Browser gespeichert.

Zu Beginn des Experiments wird einmalig ein Willkommensbildschirm angezeigt. Dieser präsentiert die Rahmenbedingungen des Experiments. Zusätzlich wird auf den Quellcode der Anwendung verwiesen. Die Teilnehmenden werden außerdem gebeten, vier Multiple-Choice-Fragen zu beantworten, die einer grundlegenden demographischen Einordnung dienen:

\begin{enumerate}\label{demography}
	 \item \textbf{How old are you (in years)?} (<18, 18-29, 29-39, 39-59, 59+)
     \item \textbf{What is your gender?} (other, female, male)
     \item \textbf{How well would you describe your English skills?} (very good, good, not so good, nonexistent)
     \item \textbf{How often do you use the commandline?} (daily, occasionally (1x per week), rarely (1x per month), very rarely (1x per year or less), I have never used it)
\end{enumerate}

Eine Teilnahme an dem Experiment ist nur möglich, wenn sämtliche Fragen vollständig beantwortet wurden. Anschließend wird jede Person zufällig und dauerhaft einer von drei Versuchsbedingungen zugewiesen:


\begin{itemize}
	\item \textbf{Kontrollgruppe:} Probanden der Kontrollgruppe müssen die Fragen ohne motivierende Spielmechaniken beantworten.
	 
    \item \textbf{Experimentalbedingung Abzeichen (AB):} Probanden der Experimentalgruppe Abzeichen erhalten in regelmäßigen Abständen Abzeichen.

    \item \textbf{Experimentalbedingung Fortschrittsanzeige (FA):} Probanden der Experimentalgruppe sehen einen klassischen, horizontalen Fortschrittsbalken am oberen Bildschirmrand.
\end{itemize}

Zu Beginn des Experiments erhalten die Teilnehmenden eine kurze Einweisung: In dieser werden die grundlegende Bedienung, die Funktionalität des Terminals und die Kommandozeilenbefehle \textbf{clear}, \textbf{about}, \textbf{bug} und \textbf{help} erklärt. Diese Einweisung wird erneut angezeigt, wenn ein Teilnehmender das Experiment unterbricht und an einem späteren Zeitpunkt fortsetzt.

\paragraph{clear}
Dieser Befehl löscht die Historie der angezeigten Befehle und hinterlässt somit ein sauberes Terminal.

\paragraph{about}
Dieser Befehl leitet den Teilnehmenden auf eine Infoseite weiter, die Informationen über das Experiment zusammenfasst. Dazu zählen Informationen über den Autor der Studie, ein Verweis auf den Quellcode und die Datenschutzerklärung.

\paragraph{bug}
Dieser Befehl erlaubt es, einen Fehlerbericht zu erstellen und abzuschicken.

\paragraph{help}
Dieser Befehl präsentiert für jede Frage einen Lösungshinweis. \\

Auf den Instruktionstext folgend wird die aktuelle Aufgabe präsentiert. Ansonsten besteht die gesamte Webseite aus einer interaktiven Kommandozeile, die einem klassischen Terminal in der Bedienung und der Optik gleicht. Nach der vollständigen Bearbeitung sämtlicher Fragen, wird ein finaler Multiple-Choice-Test angezeigt, der die persönliche Einstellung des Teilnehmenden hinsichtlich der empfundenen Motivation, entsprechend der 5-Stufigen-Lickert-Skala, abfragt:

\begin{description}
\item[Frage]\hfill \\ How much do you agree with the following statement: \textit{I felt motivated in using the application.} ?
\item[Antwortmöglichkeiten]\hfill 


\begin{itemize}
	 \item Strongly disagree (1),
	 \item Disagree (2)
	 \item Neither agree nor disagre (3)
	 \item Agree (4)
	 \item Strongly agree (5)
\end{itemize}
\end{description}


\subsection{Beschreibung der Versuchsbedingungen}
Die Experimentalbedingung mit Abzeichen erhält eine Leiste am oberen Bildschirmrand, die fünf Abzeichen beinhaltet. Diese sind zunächst grau hinterlegt, um die Teilnehmenden indirekt auf weitere Abzeichen hinzuweisen. Erst wenn die Bedingungen für das Erreichen eines Abzeichens erfüllt sind, wird dieses farblich hervorgehoben. Das Erreichen eines Meilensteins wird zudem durch eine Animation begleitet. Dies geschieht unter der Annahme, dass eine optisch ansprechende und auffällige Animation die Motivation der Teilnehmer zusätzlich steigert. Die erforderlichen Bedingungen für das Erreichen eines Abzeichens lassen sich durch einen Klick auf das jeweilige Abzeichen einsehen.

Die Abzeichen wurden auf der Basis der Arbeit von \citeauthorwithyear{antin_badges_2011} gestaltet. Jedes Abzeichen erfüllt mindestens eine der fünf, von \citeauthorwithyear{antin_badges_2011} eingeführten, Hauptfunktionen von Abzeichen:

\paragraph{Abzeichen 1:}
Dieses Abzeichen wird für das Lösen der ersten Aufgabe verliehen. So soll der Nutzer auf weitere Abzeichen aufmerksam gemacht werden. Das Abzeichen hat entsprechend eine instruktive Funktion.

\paragraph{Abzeichen 2:}
Dieses Abzeichen wird für das Lösen aller Probleme verliehen und dient primär als Statussymbol.

\paragraph{Abzeichen 3:}
Dieses Abzeichen wird für die Kombination von mindestens drei Kommandozeilenbefehlen verliehen und weist damit explizit auf die Möglichkeit der Befehlsverkettung hin. Somit erfüllt dieses Abzeichen eine instruktive Funktion.

\paragraph{Abzeichen 4:}
Dieses Abzeichen wird für die Lösung der letzten Aufgabe verliehen und dient der Bestätigung der erbrachten Leistung. Da diese Aufgabe einen erhöhten Schwierigkeitsgrad aufweist, wirkt das Abzeichen gleichzeitig als Statussymbol.

\paragraph{Abzeichen 5:}
Dieses Abzeichen wird für den zehnten Fehlversuch vergeben. So wird ein exploratives Vorgehen belohnt und das Scheitern nicht bestraft. Dadurch ist der Nutzer motiviert, kreative Lösungswege zu finden. Das Abzeichen fungiert damit indirekt als Zielvorgabe.


\begin{figure}[htbp]
    \centering
    \includesvg[width = 50pt, height = 50pt]{img/goal.svg}
    \includesvg[width = 50pt, height = 50pt]{img/goal_reached.svg}
    \includesvg[width = 50pt, height = 50pt]{img/solutions.svg}
    \includesvg[width = 50pt, height = 50pt]{img/target.svg}
    \includesvg[width = 50pt, height = 50pt]{img/win.svg}
    \caption[Auflistung der unterschiedlichen Abzeichen]{Auflistung der unterschiedlichen Abzeichen - Von links nach rechts lesend: Abzeichen 1 bis 5.}
\end{figure}


Die Experimentalbedingung Fortschrittsbalken erhält einen horizontalen, animierten Fortschrittsbalken, der auffällig am oberen Bildschirmrand platziert ist. Um den Nutzer auf die Forschrittsanzeige aufmerksam zu machen, ist diese bereits zu Beginn des Experiments minimal gefüllt und animiert. Dabei handelt es sich um eine bestimmte Anzeige, die linear gefüllt und nach jeder beantworteten Frage aktualisiert wird. Die Anzeige gibt somit Aufschluss über auf die verbleibende Restdauer des Experiments.

Beide Experimentalbedingungen unterscheiden sich ausschließlich hinsichtlich der geschilderten visuellen Elemente. Um die Vergleichbarkeit sicherzustellen, werden identische Fragen in gleicher Reihenfolge abgefragt. Durch die Trennung der einzelnen Spielelemente in unterschiedliche Versuchsbedingungen kann jedes Spielelement außerdem individuell beurteilt werden. Auf diese Weise können die Spielelemente Abzeichen und Fortschrittsanzeige jeweils auf ihren Einfluss auf die Motivation und die Leistung untersucht werden.

\begin{figure}[htbp]
    \centering
    \includegraphics[width=\textwidth]{img/full_web.png}
    \caption{Beispielhafte Darstellung der Anwendung in der Fortschrittsanzeige-Bedingung (FA)}
\end{figure}

\subsubsection{Technische Implementation}
Bei der entwickelten Anwendung handelt es sich um eine interaktive Webanwendung, die sowohl auf Mobilgeräten als auch an einem klassischen Computer nutzbar ist. Die Anwendung besteht aus drei Komponenten:

\paragraph{Statische Webseite:}
Die Webseite basiert auf HTML, CSS und Javascript. Die graphische Darstellung einer Kommandozeile basiert auf dem jQuery Plugin 'jQuery Terminal Emulator'\footnote{https://terminal.jcubic.pl/}. Die Webseite kommuniziert Nutzerdaten und abgesetzte Befehle über eine REST-API mit dem Server. Die Webseite verfügt über keinerlei Anwendungslogik und stellt lediglich die empfangenen Daten des Servers dar. Für die gesamte Anwendung wurde ein dunkles Farbschema gewählt, das im Kern aus den Farben Grün und Schwarz besteht und einem Terminal ähnelt.

\paragraph{Server:}
Der Server ist in Python geschrieben und basiert auf dem Framework Flask\footnote{https://github.com/pallets/flask}. Der Server ist verantwortlich für die Nutzerverwaltung, die Nutzererkennung, das Ausführen und die anschließende Validierung abgesetzter Befehle sowie die Datenerfassung. Sämtliche Daten werden in einer MySQL-Datenbank gespeichert. Empfangene Befehle reicht der Server an den Executor weiter und nutzt dafür eine JSON-basierte REST-API. Der Server fungiert zusätzlich als Cache für bereits ausgeführte Befehle, damit jeder Befehl nur ein einziges Mal real ausgeführt werden muss.

\paragraph{Executor:}

  Da Konsolenbefehle kombinierbar sind, gibt es stets mehrere Antwortmöglichkeiten: Selbst simple Fragen wie \textit{Geben Sie \say{Hello World} auf der Kommandozeile aus} haben unterschiedliche Lösungsmöglichkeiten, etwa:
  \begin{center}
      \verb|echo Hello World|
  \end{center}
  oder 
   \begin{center}
      \verb|echo hello world > t && cat t|.
  \end{center}
  Daher ist es nötig, die abgesendeten Befehle real auszuführen. Diese Aufgabe übernimmt der Executor, indem er Befehle in separaten Docker-Containern ausführt und die Ausgabe sowie mögliche Seiteneffekte mit der erwarteten Ausgabe abgleicht. Dabei handelt es sich um eine simple Python API, die für das Starten, das Verwalten und das Löschen von Docker-Containern verantwortlich ist: Für jeden Befehl wird ein neuer Container gestartet, der auf einem minimalen Python-Image basiert und anschließend die Ausgabe des Befehls auf Korrektheit geprüft.


\subsubsection{Datenerfassung}
Während des Experiments werden unterschiedliche Daten erfasst, die sowohl der Auswertung des Experiments als auch dem Ausschluss von Mehrfachteilnahmen dienen:

\paragraph{Demographische Daten:}
Die in \ref{demography} dargestellten Fragen werden für jeden Nutzer in der Nutzertabelle gespeichert. Zusätzlich wird die jeweilige Experimentalbedingung in derselben Tabelle gespeichert.

\paragraph{Gerätespezifische Daten:}
Für jeden Nutzer wird die IP-Adresse sowie der User-Agent gespeichert, um Mehrfachteilnahmen zu identifizieren.

\paragraph{Nutzungsdaten:}
Während des Experiments werden unterschiedliche Daten erfasst, die der späteren Bewertung der Effektivität der einzelnen Maßnahmen dienen:

\begin{itemize}
	 \item abgesendete Befehle mit Zeitstempel
	 \item gelöste Aufgaben 
	 \item erreichte Abzeichen (nur für die Experimentalbedingung Abzeichen)
	 \item Feedback hinsichtlich der empfundenen Motivation (nur nach vollständiger Bearbeitung des Experiments)
\end{itemize}


\subsection{Pretest}\label{verlauf}
Vor der Veröffentlichung des Experiments wurde ein Pretest durchgeführt, der potentielle Fehler und Unklarheiten im Studiendesign aufzeigen sollte. Zusätzlich kann überprüft werden, ob die Studie grundlegend funktioniert. Im Rahmen des Tests wurden neben kleineren Fehlern wesentliche Probleme deutlich: Zum einen waren die Abzeichen deutlich zu unauffällig und wirkten laut Aussage der Teilnehmer wenig bis gar nicht motivierend. Aus diesem Grund wurden die Abzeichen größer und farbenfroher gestaltet. Zusätzlich wurde eine Animation abgespielt, sobald ein Abzeichen erreicht wurde. Außerdem gaben mehrere Teilnehmer an, dass die Fragen unverständlich oder zu komplex seien. Dies hat dazu geführt, dass die Komplexität der Fragen reduziert wurde. Die so entstandenen Fragen lassen sich durch die Kombination von maximal zwei Befehlen lösen. Zusätzlich wurde eine Hilfsfunktion eingebaut, die durch den Befehl \textbf{help} aufrufbar ist. Außerdem wurde die Reaktionszeit der Anwendung, definiert als Zeitspanne zwischen dem Absetzen eines Befehls und der Anzeige des Ergebnisses, als sehr langsam empfunden. Daher wurde das Image des Containers verkleinert und auf die nötigsten Programme reduziert. Abschließend wurde ein Cache für bereits ausgeführte Befehle eingeführt. Die geschilderten Maßnahmen führen dazu, dass die finale Anwendung eine mittlere Reaktionszeit von 0.5 bis 2 Sekunden pro abgesetztem Befehl aufweist.


\subsection{Statistische Auswertung}
Die Gesamtspielzeit ergibt sich aus der Differenz des ersten und des letzten abgeschickten Befehls. Zeitspannen von mehr als 30 Minuten ohne Nutzeraktivität - es wurde kein Befehl abgeschickt - werden als Inaktivität gewertet und von der Gesamtzeit abgezogen. Die aufgestellten Hypothesen werden mithilfe von t-Tests überprüft und die dafür erforderlichen Voraussetzungen Normalverteilung und Harianzhomogenität separat geprüft. Die t-Tests werden durch eine einfaktorielle ANOVA Varianzanalyse erweitert, durch die sich die drei Versuchsgruppen miteinander vergleichen lassen.
%Seitenumbruch für die folgende Bibliografie
\clearpage

%%Bibliografie, bitte richtigen Bib-Dateinamen eintragen
\bibliography{./subdocs/biblio}
\bibliographystyle{apacite}

%Appendix falls benötigt, hier ebenfalls die Kapitel einfügen
%\appendix
%\section*{Appendix}
\begin{table}[htb]
\begin{minipage}{\linewidth}
\renewcommand{\footnoterule}{}
\renewcommand{\thefootnote}{\alph{footnote}}
\caption{Einige experimentelle Zahlen.}
\label{tab:tab1}
\centering
\begin{tabular}{lcc}
\toprule
          & \multicolumn{2}{c}{Factor 2} \\ 
          	\cmidrule{2-3}
Factor 1  & Condition A  & Condition B \footnote{Eine weitere Fussnote}  \\ 
\midrule
First     & 586 (231)    & 649 (255)     \\
          &    2.2       &    7.5        \\
Second    & 590 (195) \footnote{Dies ist eine Fussnote innerhalb der Tabelle}   & 623 (231)     \\
          &    2.8       &    2.5        \\ 
\bottomrule
\end{tabular}
\end{minipage}
\end{table}
%\include{./subdocs/appendixB}
\addcontentsline{toc}{section}{Appendix}
%\newpage

%Index falls gewünscht. Nicht vergessen den <Index Befehl> laufen zu lassen, ansonsten die folgenden Zeilen mit % auskommentieren bis end
% Index soll Stichwortverzeichnis heissen
%\renewcommand{\indexname}{Stichwortverzeichnis} 
%Achtung bei englischer Dokumentsprache

% Stichwortverzeichnis soll im Inhaltsverzeichnis auftauchen
%\addcontentsline{toc}{section}{Stichwortverzeichnis}
%\newpage
% Ausgabe des Index
%\printindex
\end{document}
