%-------------------------------------------------------------------------------
% Methoden
%-------------------------------------------------------------------------------
\section{Methoden}

Um herauszufinden, welchen Einfluss die Spielmechaniken Abzeichen und Fortschrittsbalken auf Motivation und Leistung haben, wurde eine quantitative Studie durchgeführt. Dazu wurde ein Onlineexperiment durchgeführt. Das Experiment bestand aus einer interaktiven Konsolenanwendung, die im Browser läuft und das Verhalten einer nativen Kommandozeile möglichst exakt imitiert. Im Rahmen des Experiments wurde den Teilnehmern eine vordefinierte Menge an Fragen gestellt, die es durch die Eingabe typischer Konsolenbefehle zu lösen galt. Diese Anwendung ist in ihrer Grundform bewusst demotivierend gestaltet und stellte damit einen spielfremden Kontext dar. Dieser wurde anschließend um die Spiel-Design-Elemente Abzeichen und Fortschrittsbalken erweitert. Während des Experiments wurden relevante Daten erhoben. Diese dienen der Beurteilung der Wirksamkeit der genannten Spielmechaniken.

\subsection{Zielgruppe}
Die Zielgruppe bestand aus Personen, die Berührungspunkte mit der Kommandozeile haben oder Interesse an einem Erlernen grundlegender Fähigkeiten eben dieser haben. Die Fragen wurden daher so gewählt, dass diese auch durch Beginner lösbar sind.
Zusätzlich wurde eine optionale Hilfsfunktion integriert, die hilfreiche Tipps für das Lösen der Aufgabe bereitstellte.
Damit zielte die Anwendung u.a. auf informatikaffine Personen, Studenten eines MINT-Faches, Entwickler, Systemadministratoren und Informatiker ab.
Die Beschaffenheit der Zielgruppe ließ die Annahme zu, dass die Teilnehmer ausreichende Englischkenntnisse besaßen, um eine englische Anwendung zu bedienen. Daher ist die Anwendung ausschließlich auf Englisch lokalisiert.

\subsection{Versuchsablauf}
Das Experiment fand ausschließlich online statt. Die Teilnahme von einem beliebigen Endgerät möglich. Dabei wurde bewusst darauf geachtet, dass die Anwendung von mobilen Geräten erreichbar und auch nutzbar war. Dies geschah mit der Absicht die Anzahl potentieller Teilnehmer zu maximieren. Das Experiment war zu einem beliebigen Zeitpunkt durch den Probanden unterbrechbar. Der Fortschritt wurde gespeichert und eine Fortsetzung des Experiments war zu einem späteren Zeitpunkt möglich.

Zu Beginn des Experiments wurde jedem Teilnehmer einmalig ein Willkommensbildschirm angezeigt. Auf diesem wurden die Teilnehmer auf die Rahmenbedingungen des Experiments hingewiesen. Dazu zählte der Hinweis, dass es sich bei der Anwendung um ein Experiment handelt, ein Verweis auf den Quellcode der Anwendung\footnote{Link: https://github.com/M0r13n/terminal} und die Datenschutzerklärung. Die Teilnehmer wurden zusätzlich gebeten, vier Fragen zu beantworten. Dabei handelte es sich um einen Fragebogen, der durch Abhaken von Checkboxen zu beantworten war. Die Fragen dienten einer grundlegenden demographischen Einordnung der Teilnehmer. Es wurden folgende Fragen gestellt (die Antwortmöglichkeiten sind in Klammern“Gamification” between game and play, whole and parts dargestellt):

\begin{enumerate}\label{demography}
	 \item \textbf{How old are you (in years)?} (<18, 18-29, 29-39, 39-59, 59+)
     \item \textbf{What is your gender?} (other, female, male)
     \item \textbf{How well would you describe your English skills?} (very good, good, not so good, nonexistent)
     \item \textbf{How often do you use the commandline?} (daily, occasionally (1x per week), rarely (1x per month), very rarely (1x per year or less), I have never used it)
\end{enumerate}

Eine Teilnahme an dem Experiment war nur nach der erfolgreichen Beantwortung der Fragen möglich. Anschließend wurde jede Person zufällig und dauerhaft einer von drei Experimentalbedingungen  zugeteilt.


\begin{itemize}
	 \item \textbf{Kontrollgruppe:} Probanden der Kontrollgruppe mussten die Fragen ohne motivierende Spielmechaniken beantworten.
	 
    \item \textbf{Experimentalgruppe Abzeichen:} Probanden der Experimentalgruppe Abzeichen sollten durch das Spielelement Abzeichen motiviert werden. Unabhängige Variable: Abzeichen.

    \item \textbf{Experimentalgruppe Fortschrittsanzeige:} Probanden der Experimentalgruppe Fortschrittsanzeige sollten durch das Spielelement Fortschrittsanzeige motiviert werden. Unabhängige Variable: Fortschrittsanzeige.
\end{itemize}

Zu Beginn des Experiments erhielt jeder Teilnehmer eine Instruktion. In dieser wurde die Bedienung und Funktion des Terminals erklärt. Dabei wurde explizit auf die besonderen Kommandozeilenbefehle \textbf{clear}, \textbf{about}, \textbf{bug} und \textbf{help} hingewiesen.

\paragraph{clear}
Dieser Befehl löscht die Historie der angezeigten Befehle und hinterlässt somit ein sauberes Terminal.

\paragraph{about}
Dieser Befehl führt den Teilnehmer auf eine Infoseite. Auf dieser sind alle Informationen über das Experiment zusammengefasst.

\paragraph{bug}
Dieser Befehl erlaubt es dem Probanden einen Fehlerbericht zu erstellen.

\paragraph{help}
Dieser Befehl offenbart für jede Frage einen Lösungshinweis. \\

Unter dem Instruktionstext fand sich zudem die aktuelle Aufgabe, die es durch die Eingabe eines Konsolenbefehls zu lösen galt. Der Rest der Webseite bestand aus einer interaktiven Kommandozeile. Nach dem Beenden des Experiments sollten die Teilnehmer ihre persönliche Einstellung hinsichtlich empfundener Motivation, entsprechend der 5-Stufigen-Lickert-Skala, angeben. Konkret sollten die Teilnehmer folgende Frage beantworten:

\begin{description}
\item[Frage]\hfill \\ Wie sehr stimmst du folgender Aussage zu: \textit{Ich fühlte mich bei der Beantwortung der Fragen motiviert.} ?
\item[Antwortmöglichkeiten]\hfill 


\begin{itemize}
	 \item trifft zu (1),
	 \item trifft eher zu (2)
	 \item teils-teils (3)
	 \item trifft eher nicht zu (4)
	 \item trifft nicht zu (5)
\end{itemize}
\end{description}

Die Experimentalgruppe Abzeichen erhielt zusätzlich eine Leiste am oberen Bildschirmrand. Diese wurde visuell prominent gestaltet. In dieser Leiste befanden sich fünf Abzeichen. Diese wurden zunächst grau hinterlegt und erst durch das Erreichen eines bestimmten Meilensteins farblich hervorgehoben. Das Erreichen eines Meilensteins wurde zudem durch eine Animation begleitet. Dies geschah unter der Annahme, dass eine optisch ansprechende und auffällige Animation die Motivation der Teilnehmer weiter steigern würde. Die erforderlichen Bedingungen für das Erreichen eines Abzeichens ließen sich durch einen Klick auf das jeweilige Abzeichen erfahren. 

Die Abzeichen wurden auf der Basis der Arbeit von \citeauthor{antin_badges_2011} gestaltet. Jedes Abzeichen erfüllt eine der fünf, von \citeauthor{antin_badges_2011} eingeführten, Hauptfunktionen von Abzeichen.

\paragraph{Abzeichen 1:}
Dieses Abzeichen wird für das Lösen der ersten Aufgabe verliehen. So soll der Nutzer auf weitere Abzeichen aufmerksam gemacht werden. Das Abzeichen hat entsprechend eine instruktive Funktion.

\paragraph{Abzeichen 2:}
Dieses Abzeichen wird für das Lösen aller Probleme verliehen und dient primär als Statussymbol.

\paragraph{Abzeichen 3:}
Dieses Abzeichen wird für die Verkettung von mindestens drei Kommandozeilenbefehlen verliehen und weißt damit explizit auf die Möglichkeit der Befehlsverkettung hin. Entsprechend erfüllt das Abzeichen eine instruktive Funktion.

\paragraph{Abzeichen 4:}
Dieses Abzeichen wird für die Lösung der zwölften Aufgabe verliehen und dient der Bestätigung der erbrachten Leistung des Probanden. Da diese Aufgabe, im Vergleich zu anderen Aufgaben, vergleichsweise schwierig ist, wirkt das Abzeichen zusätzliche als Statussymbol.

\paragraph{Abzeichen 5:}
Dieses Abzeichen wird für den zehnten Fehlversuch vergeben. Der Nutzer soll nicht für sein Scheitern bestraft werden. Stattdessen wird der Versuch belohnt. Dadurch ist der Nutzer angehalten kreative Wege zu finden, um die Aufgaben zu lösen. Das Abzeichen fungiert damit indirekt als Zielvorgabe.


\begin{figure}[htbp]
    \centering
    \includesvg[width = 50pt, height = 50pt]{img/goal.svg}
    \includesvg[width = 50pt, height = 50pt]{img/goal_reached.svg}
    \includesvg[width = 50pt, height = 50pt]{img/solutions.svg}
    \includesvg[width = 50pt, height = 50pt]{img/target.svg}
    \includesvg[width = 50pt, height = 50pt]{img/win.svg}
    \caption{Von links nach rechts lesend: Abzeichen 1 bis 5.}
\end{figure}


Die Experimentalgruppe Fortschrittsbalken erhielt einen animierten Fortschrittsbalken. Dieser wurde auffällig am oberen Bildschirmrand platziert. Um den Nutzer auf die Forschrittsanzeige aufmerksam zu machen, war diese bereits zu Beginn des Experiments zu einem Prozent gefüllt. Dabei handelte es sich um eine bestimmte Anzeige, die sich entsprechend des aktuellen Fortschritts des Teilnehmers zunehmend füllte. Somit war es dem Probanden zusätzlich möglich auf die Restdauer des Experiments zu schließen.

Beide Versuchsbedingungen unterschieden sich ausschließlich hinsichtlich der geschilderten visuellen Elemente. Der Versuchsaufbau und das Verhalten der Anwendung war für alle Teilnehmer identisch. Selbiges galt für die Fragen. Jeder Teilnehmer erhielt identische Fragen in einer definierten Reihenfolge. Potentielle Unterschiede in der Nutzung der Anwendung sind demnach ausschließlich auf die Einführung der Spielelemente zurückzuführen. Gleichzeitig kann jedes Spielelement individuell beurteilt werden.  Auf diese Weise können die Spielelemente Abzeichen und Fortschrittsanzeige individuell auf ihren Einfluss auf Motivation und Leistung untersucht werden.

\begin{figure}[htbp]
    \centering
    \includegraphics[width=\textwidth]{img/progressbar.png}
    \caption{Ein bereits teilweise gefüllter Fortschrittsbalken.}
\end{figure}

\subsubsection{Technische Implementation}
Bei der entwickelten Anwendung handelt es sich um eine interaktive Webanwendung. Diese ist sowohl auf Mobilgeräten als auch an einem klassischen Computer nutzbar. Die Anwendung besteht aus drei Komponenten: Einer statischen Webseite, einem Server und einem Executor. 

\paragraph{Statische Webseite:}
Die Webseite basiert auf HTML, CSS und Javascript. Die graphische Darstellung einer Kommandozeile basiert auf dem jQuery Plugin \qq{jQuery Terminal Emulator}\footnote{https://terminal.jcubic.pl/}. Die Webseite kommuniziert über eine REST-API mit dem Server. Über diese Schnittstelle werden Nutzerdaten und abgesetzte Befehle kommuniziert. Die Webseite verfügt über keinerlei Anwendungslogik und stellt lediglich die empfangenen Daten des Servers graphisch dar. Die Webseite wurde absichtlich unmotivierend gestaltet. Es wurde ein dunkles Farbschema gewählt. Dieses besteht im Kern lediglich aus den Farben Grün und Schwarz. Das Design der gesamten Webseite ist stark an ältere Terminals angelehnt und auf das Nötigste reduziert. Es finden sich keinerlei zusätzliche graphische Elemente.

\paragraph{Server:}
Der Server ist in Python geschrieben und basiert auf dem verbreiteten Framework \qq{Flask}\footnote{https://github.com/pallets/flask}. Der Server ist verantwortlich für die Nutzerverwaltung, die Nutzererkennung, das Ausführen und die anschließende Validierung abgesetzter Befehle sowie die Datenerfassung. Sämtliche Daten werden in einer MySQL-Datenbank gespeichert. Empfangene Befehle reicht der Server an den Executor weiter. Die Kommunikation erfolgt auch hier über eine JSON-basierte REST-API. Der Server fungiert zusätzlich als Cache für bereits ausgeführte Befehle. Diese werden in der Datenbank zwischengespeichert. Somit muss jeder Befehl nur ein einziges Mal real ausgeführt werden.

\paragraph{Executor:}

  Da Konsolenbefehle kombinierbar sind, gibt es mehrere Antwortmöglichkeiten für ein und dieselbe Frage. Selbst einfache Fragen wie \textit{Geben Sie \say{Hello World} auf der Kommdozeile aus} haben nahezu unendlich viele Lösungsmöglichkeiten. Mögliche Antwortmöglichkeiten sind etwa 
  \begin{center}
      \verb|echo Hello World|
  \end{center}
  oder 
   \begin{center}
      \verb|echo hello world > t && cat t|.
  \end{center}
  Daher ist es nötig, die abgesendeten Befehle real auszuführen. Dazu wird jeder Befehl in einem eigenen Docker Container ausgeführt und die Ausgabe sowie mögliche Seiteneffekte mit der erwarteten Ausgabe abgeglichen. Diese Aufgabe übernimmt der Executor als separate Komponente. Dabei handelt es sich um eine simple Python API. Diese ist verantwortlich für das Starten, Verwalten und Löschen von Dockercontainern. Für jeden Befehl wird ein neuer Container gestartet und die Ausgabe des Befehls auf ihre Korrektheit überprüft. Die Container basieren auf einem minimalen Python Image. 


\subsubsection{Datenerfassung}
Während des Experiments wurden unterschiedliche Daten erfasst. Diese Daten dienen sowohl der Auswertung und Analyse des Experiments als auch dem Ausschluss von Mehrfachteilnahmen. Die folgenden Daten wurden für jeden Probanden gespeichert.

\paragraph{Demographische Daten:}
Die in \ref{demography} dargestellten Fragen wurden für jeden Nutzer in der Nutzertabelle gespeichert. Zusätzlich wurde die jeweilige Experimentalbedingung in derselben Tabelle gespeichert.

\paragraph{Gerätespezifische Daten:}
Für jeden Nutzer wurde die IP-Adresse sowie der User-Agent gespeichert. Auf diese Weise sollten Mehrfachteilnahmen des gleichen Nutzers ausgeschlossen werden.

\paragraph{Nutzungsdaten:}
Während des Experiments wurden unterschiedliche Daten erfasst. Diese Erfassung diente der späteren Bewertung der Effektivität der einzelnen Maßnahmen. Folgende Daten wurden erfasst:

\begin{itemize}
	 \item Abgesendete Befehle mit Zeitstempel
	 \item Gelöste Aufgaben 
	 \item Erreichte Abzeichen (nur für die Experimentalbedingung Abzeichen)
	 \item Feedback hinsichtlich der empfundenen Motivation (nur nach vollständiger Bearbeitung des Experiments)
\end{itemize}


\subsection{Forschungsverlauf}
Vor der Veröffentlichung des Experiments wurde ein Pretest durchgeführt. Durch diesen sollten potentielle Fehler und Unklarheiten im Studiendesign reduziert werden. Zusätzlich konnte überprüft werden, ob die Studie grundlegend funktioniert. Im Rahmen des Tests wurden neben kleineren Fehlern wesentliche Probleme deutlich. Zum einen waren die Abzeichen deutlich zu unauffällig und wirkten laut Aussage der Teilnehmer wenig bis gar nicht motivierend. Aus diesem Grund wurden die Abzeichen größer und farbenfroher gestaltet. Zusätzlich wurde eine Animation abgespielt, sobald ein Abzeichen erreicht wurde. Außerdem gaben mehrere Teilnehmer an, dass die Fragen unverständlich und zu komplex seien. Dies hat dazu geführt, dass die Komplexität drastisch reduziert wurde. Am Ende ließen sich alle Fragen durch maximal zwei Befehle lösen. Zusätzlich wurde eine Hilfsfunktion eingebaut. Durch den Befehl \textbf{help} erhielten Probanden weiterführende Informationen, die der Lösung der Aufgabe dienten. Die Reaktionszeit wurde als zu langsam empfunden. Daher wurde das Image des Containers drastisch reduziert und auf die nötigsten Programme reduziert. Zusätzlich wurde ein Cache für bereits ausgeführte Befehle eingeführt. Auf diese Weise konnte eine Reaktionszeit von 0.5 bis 2 Sekunden pro abgesetztem Befehl gewährleistet werden.

Das Experiment startete schließlich am 05.07.2020. Bis zum 05.08.2020 kamen jedoch nur 39 Teilnehmer zusammen. Aus diesem Grund wurde am 05.08.2020 entschlossen, das Experiment auf Facebook und in dem Forum Hardwareluxx\footnote{www.hardwareluxx.de} zu teilen. Zusätzlich wurde das Experiment auf Reddit in dem Subreddit \textit{/r/takemysurvey/} und auf HackerNews\footnote{news.ycombinator.com/} beworben. Leider wurde die Studie als Eigenwerbung bewertet und nach wenigen Stunden auf HackerNews gelöscht. 

\subsection{Statistische Auswertung}
Leistung messe ich durch die Anzahl beantworteter Fragen. Zusätzlich wird die Gesamtzeit als Vergleichsparameter herangezogen. Diese ergibt sich aus der Differenz des ersten und letzten abgeschickten Befehls. Zeitspannen von mehr als 30 Minuten ohne Nutzeraktivität (d.h. es wurde in einem Zeitraum von 30 Minuten kein Befehl abgesetzt) werden als Inaktivität gewertet und von der Gesamtzeit abgezogen. Dies geschieht unter der Annahme, dass motivierte Teilnehmer mehr Zeit investieren. Auf diese Weise wird der Effekt der unabhängigen Variablen Fortschrittsbalken und Abzeichen auf Motivation und Leistung quantifizierbar gemacht. Ausreißer werden von den Ergebnissen ausgeschlossen. Als Grenzbereich ist der Mittelwert plus zwei Standardabweichungen gewählt.

Die aufgestellten Hypothesen werden mithilfe von t-Tests überprüft. Die dafür erforderlichen Voraussetzungen sind erfüllt:

\paragraph{Unabhängigkeit der Messungen}
Da ein Proband immer einer festen Gruppen zugewiesen ist, ist es nicht möglich, dass Teilnehmer in mehreren Gruppen auftauchen.

\paragraph{Intervallskalierung}
Die abhängige Variable \textit{Anzahl beantworteter Fragen} lässt sich quantitativ mittels Zahlen darstellen und ist daher ein metrisches Messniveau.

\paragraph{Nominalskalierte, unabhängige Variable}
Es werden jeweils zwei unabhängige Gruppen miteinander verglichen.

\paragraph{Normalverteilung}
Das Vorliegen einer Normalverteilung wird durch Shapiro–Wilk Tests überprüft.

\paragraph{Varianzhomogenität}
Das Vorhandensein von Varianzhomogenität wird durch Levene-Tests überprüft.

Die t-Tests werden durch eine einfaktorielle ANOVA Varianzanalyse erweitert. Durch diese lassen sich die drei Versuchsgruppen miteinander vergleichen.

\todo[inline]{Sollte sich herausstellen, dass nicht-parametrische, nicht normalverteilte Daten vorliegen, führe ich statt dem t-test einen Mann-Whitney-U test und statt ANOVA einen 	Kruskal-Wallis-Test durch.}