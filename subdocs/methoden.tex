%-------------------------------------------------------------------------------
% Methoden
%-------------------------------------------------------------------------------
\section{Methoden}

Um herauszufinden, welchen Einfluss die Spielmechaniken Abzeichen und Fortschrittsbalken auf Motivation und Leistung haben, wurde eine quantitative Studie durchgeführt. Dazu wurde ein Onlineexperiment durchgeführt. Das Experiment bestand aus einer interaktiven Konsolenanwendung, die im Browser läuft und das Verhalten einer nativen Kommandozeile möglichst exakt imitiert. Im Rahmen des Experiments wurde den Teilnehmern eine vordefinierte Menge an Fragen gestellt, die es durch die Eingabe typischer Konsolenbefehle zu lösen galt. Diese Anwendung ist in ihrer Grundform bewusst demotivierend gestaltet und stellte damit einen spielfremden Kontext dar. Dieser wurde anschließend um die Spiel-Design-Elemente Abzeichen und Fortschrittsbalken erweitert. Während des Experiments wurden relevante Daten erhoben. Diese dienen der Beurteilung der Wirksamkeit der genannten Spielmechaniken.

\subsection{Zielgruppe}
Die Zielgruppe bestand aus Personen, die Berührungspunkte mit der Kommandozeile haben oder Interesse an einem Erlernen grundlegender Fähigkeiten eben dieser haben. Die Fragen wurden daher so gewählt, dass diese auch durch Beginner lösbar sind.
Zusätzlich wurde eine optionale Hilfsfunktion integriert, die hilfreiche Tipps für das Lösen der Aufgabe bereitstellte.
Damit zielte die Anwendung u.a. auf informatikaffine Personen, Studenten eines MINT-Faches, Entwickler, Systemadministratoren und Informatiker ab.
Die Beschaffenheit der Zielgruppe ließ die Annahme zu, dass die Teilnehmer ausreichende Englischkenntnisse besaßen, um eine englische Anwendung zu bedienen. Daher ist die Anwendung ausschließlich auf Englisch lokalisiert.

\subsection{Versuchsablauf}
Das Experiment fand ausschließlich online statt. Die Teilnahme von einem beliebigen Endgerät möglich. Dabei wurde bewusst darauf geachtet, dass die Anwendung von mobilen Geräten erreichbar und auch nutzbar war. Dies geschah mit der Absicht die Anzahl potentieller Teilnehmer zu maximieren. Das Experiment war zu einem beliebigen Zeitpunkt durch den Probanden unterbrechbar. Der Fortschritt wurde gespeichert und eine Fortsetzung des Experiments war zu einem späteren Zeitpunkt möglich.

Zu Beginn des Experiments wurde jedem Teilnehmer einmalig ein Willkommensbildschirm angezeigt. Auf diesem wurden die Teilnehmer auf die Rahmenbedingungen des Experiments hingewiesen. Dazu zählte der Hinweis, dass es sich bei der Anwendung um ein Experiment handelt, ein Verweis auf den Quellcode der Anwendung\footnote{Link: https://github.com/M0r13n/terminal} und die Datenschutzerklärung. Die Teilnehmer wurden zusätzlich gebeten, vier Fragen zu beantworten. Dabei handelte es sich um einen Fragebogen, der durch Abhaken von Checkboxen zu beantworten war. Die Fragen dienten einer grundlegenden demographischen Einordnung der Teilnehmer. Es wurden folgende Fragen gestellt (die Antwortmöglichkeiten sind in Klammern dargestellt):

\begin{enumerate}
	 \item \textbf{How old are you (in years)?} (<18, 18-29, 29-39, 39-59, 59+)
     \item \textbf{What is your gender?} (other, female, male)
     \item \textbf{How well would you describe your English skills?} (very good, good, not so good, nonexistent)
     \item \textbf{How often do you use the commandline?} (daily, occasionally (1x per week), rarely (1x per month), very rarely (1x per year or less), I have never used it)
\end{enumerate}

Eine Teilnahme an dem Experiment war nur nach der erfolgreichen Beantwortung der Fragen möglich. Anschließend wurde jede Person zufällig und dauerhaft einer von drei Experimentalbedingungen  zugeteilt. 

\paragraph{Kontrollgruppe}
Probanden der Kontrollgruppe mussten die Fragen ohne motivierende Spielmechaniken beantworten.

\paragraph{Experimentalgruppe Abzeichen}
Probanden der Experimentalgruppe Abzeichen sollten durch das Spielelement Abzeichen motiviert werden.

\paragraph{Experimentalgruppe Fortschrittsanzeige}
Probanden der Experimentalgruppe Fortschrittsanzeige sollten durch das Spielelement Fortschrittsanzeige motiviert werden.

Zu Beginn des Experiments erhielt jeder Teilnehmer eine Instruktion. In dieser wurde die Bedienung und Funktion des Terminals erklärt. Dabei wurde explizit auf die besonderen Kommandozeilenbefehle \textbf{clear}, \textbf{about}, \textbf{bug} und \textbf{help} hingewiesen.

\paragraph{clear}
Dieser Befehl löscht die Historie der angezeigten Befehle und hinterlässt somit ein sauberes Terminal.

\paragraph{about}
Dieser Befehl führt den Teilnehmer auf eine Infoseite. Auf dieser sind alle Informationen über das Experiment zusammengefasst.

\paragraph{bug}
Dieser Befehl erlaubt es dem Probanden einen Fehlerbericht zu erstellen.

\paragraph{help}
Dieser Befehl offenbart für jede Frage einen Lösungshinweis. \\



Unter dem Instruktionstext fand sich zudem die aktuelle Aufgabe, die es durch die Eingabe eines Konsolenbefehls zu lösen galt. Der Rest der Webseite bestand aus einer interaktiven Kommandozeile. Bis zu diesem Punkt war die Anwendung für alle Experimentalgruppen vollkommen identisch.

Die Experimentalgruppe Abzeichen erhielt zusätzlich eine Leiste am oberen Bildschirmrand. Diese wurde sehr prominent gestaltet, damit sie nicht zu übersehen ist. In dieser Leiste befanden sich fünf Abzeichen. Diese wurden zunächst grau hinterlegt und erst durch das Erreichen eines bestimmten Meilensteins farblich hervorgehoben. Das Erreichen eines Meilensteins wurde zudem durch eine Animation begleitet. Dies geschah unter der Annahme, dass eine optisch ansprechende und auffällige Animation die Motivation der Teilnehmer weiter steigern würde. Die erforderlichen Bedingungen für das Erreichen eines Abzeichens ließen sich durch einen Klick auf das jeweilige Abzeichen erfahren.

\subsubsection{Technische Implementation}

\subsection{Ein- und Ausschlusskriterien}

\subsection{Forschungsverlauf}

\subsection{Datenanalyse}

\subsection{Beschreibung der Validität und Reliabilität}