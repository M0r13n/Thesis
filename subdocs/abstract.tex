\section{Abstract}
Gamification bezeichnet die Verwendung von Spieldesignelementen in einem spielfremden Kontext. Das Ziel dieser Arbeit war es, zu prüfen, ob die Spielelemente Abzeichen und Fortschrittsbalken motivierend auf die Nutzer einer Kommandozeile wirken. Dazu wurde eine virtuelle Kommandozeile um die genannten Designelemente erweitert. Die so entstandene Anwendung stellt die Grundlage für eine quantitative Onlinestudie mit 279 Teilnehmern dar. Beide Spielelemente zeigten keinen statistisch signifikanten Effekt in Bezug auf die Anzahl gelöster Aufgaben, die Menge der abgesetzten Befehle oder die Gesamtspielzeit im Vergleich zu einer Kontrollgruppe. Jedoch zeigt eine weiterführende Analyse, dass erfahrene Nutzer mittleren Alters, die einen Fortschrittsbalken sahen, im Mittel statistisch signifikant mehr Aufgaben lösten als eine Kontrollgruppe. Die Ergebnisse beleuchten das schwierige Zusammenspiel verschiedenster Faktoren, die für Motivation verantwortlich sind und deuten auf einen motivierenden Effekt der Fortschrittsanzeige hin. 