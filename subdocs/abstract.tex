\section{Abstract}
Gamification bezeichnet die Verwendung von Spieldesignelementen in einem spielfremden Kontext. Das Ziel dieser Arbeit war es, zu prüfen, ob die Spielelemente Abzeichen und Fortschrittsbalken motivierend auf die Nutzer einer Kommandozeile wirken.
Dazu wurde eine virtuelle Kommandozeile um die genannten Designelemente erweitert. Die so entstandene Webanwendung bildete die Basis für ein Quiz, dessen Fragen durch typische Konsolenbefehle zu beantworten waren.
Auf Grundlage dieser Anwendung wurde eine quantitative Onlinestudie mit 278 Teilnehmern durchgeführt. Eine Betrachtung der Stichprobe als Ganzes zeigt zunächst keinen signifikanten Unterschied zwischen den Versuchsbedingungen und einer Kontrollgruppe hinsichtlich Gesamtspielzeit und Anzahl gelöster Aufgaben. Jedoch zeigt eine weiterführende Analyse der Daten, dass erfahrene Nutzer mittleren Alters, die einen Fortschrittsbalken sahen, im Mittel statistisch signifikant mehr Aufgaben lösten als eine Kontrollgruppe.
Die Ergebnisse beleuchten das schwierige Zusammenspiel verschiedenster Faktoren, die für Motivation verantwortlich sind und deuten auf einen motivierenden Effekt der Fortschrittsanzeige bei erfahrenen Konsolenanwendern hin. 