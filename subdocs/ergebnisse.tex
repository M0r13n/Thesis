%-------------------------------------------------------------------------------
% Methoden
%-------------------------------------------------------------------------------
\section{Ergebnisse}
Nachfolgend erfolgt die Untersuchung der Ergebnisse des Experiments.

\subsection{Beschreibung der Stichprobe}\label{beschreibung}
Die Studie startete am 05.07.2020 und endete am XX.09.2020. Das Experiment war für die Dauer der Studie durchgehend erreichbar. Insgesamt haben XXX Teilnehmer an dem Experiment teilgenommen. Davon haben XXX Personen das Experiment vollständig durchlaufen und sämtliche Fragen korrekt beantwortet. Aus dem Datensatz wurden insgesamt XXX Einträge entfernt. Diese werden nicht in der Auswertung berücksichtigt. Diese setzen sich wie folgt zusammen:

\begin{itemize}
  \item XXX Teilnehmer hatten identische IP-Adressen (es wurde nur der erste Versuch gewertet)
  \item XXX Teilnehmer haben keinen Befehl abgesendet
\end{itemize}

Damit hat der bereinigte Datensatz einen Umfang von XXX. 82\% der Teilnehmer sind männlich, 14,6\% weiblich und 3,4\% gaben ein anderes Geschlecht an. Kein Teilnehmer gab an, über \textit{non-existent} Englischkenntnisse zu verfügen. Dem gegenüber verfügten 34,8\% über \textit{very good} Englischkenntnisse, 47,2\% über \textit{good} Englischkenntnisse und 18,0\% über \textit{not so good} Englischkenntnisse. Die Altersverteilung der Teilnehmer ist wie folgt: 53,9\% gaben 18-29 Jahre an, 30,3\%  29-39 Jahre,  10,1\% 39-59 Jahre, 3,4\% weniger als 18 Jahre und lediglich 2,2\% gaben an, älter als 59 Jahre zu sein. Die meisten Teilnehmer nutzen die Kommandozeile mindestens einmal im Monat (28.1\% täglich, 40,4\% wöchentlich, 12,4\% monatlich). Nur 12,4\% gaben an, die Kommandozeile selten (1x im Jahr oder weniger) oder gar nicht zu nutzen (0\%). 

\subsection{Deskriptive Statistik}
Im Mittel verbrachten die Teilnehmer 6 Minuten und 14 Sekunden mit dem Experiment (SD = 8,57) und setzten dabei im Mittel 20 Befehle ab (SD = 21). Sowohl bei der Gesamtdauer als auch bei der Anzahl der abgesendeten Befehle gibt es extreme Ausreißer nach oben. So investierte ein Teilnehmer über 50 Minuten in das Experiment und verwendete dabei 1747 Befehle. Bei diesen Ausreißern gehe ich davon aus, dass eine, von den Spielelementen unabhängige, Intrinsische Motivation vorliegt. Aus diesem Grund habe ich beschlossen, entsprechende Datensätze, die stark vom Mittelwert abweichen, von den Ergebnissen auszuschließen. Als Grenzwert wählte ich den Mittelwert zuzüglich zwei Standardabweichungen. Dies führt dazu, dass sämtliche Teilnehmer mit mehr als 3015 Sekunden Spieldauer oder mehr als 62 Befehlen aus dem Datensatz eliminiert wurden. Damit umfasst der endgültige Datensatz XXX Teilnehmer.

% Anzahl Befehle
Die mittlere Anzahl abgesetzter Befehle ist der in der Versuchsbedingung 'Abzeichen' mit 24,7 (SD = 26.3) am höchsten und damit 25\% höher als in der Kontrollgruppe. Diese weist im Vergleich eine mittlere Anzahl abgesetzter Befehle von 20.1 (SD = 26.3) auf. Auffällig ist, dass Teilnehmer der Versuchsbedingung 'Fortschrittsanzeige' im Mittel lediglich 14.1 (SD = 26.3) Befehle absetzten (siehe Abbildung XX). Aufgrund der in Sektion \ref{beschreibung} geschilderten hohen Streuung der Ergebnisse, bietet es sich an, zusätzlich den Median zu betrachten. Auch bei einem Vergleich des Medians weist die Versuchsbedingung 'Abzeichen' mit 21.5 eine deutlich erhöhte Anzahl abgeschickter Befehle gegenüber der Kontrollgruppe auf. Mit 12.5 Befehlen bleibt die Kontrollgruppe ebenfalls oberhalb der Experimentalbedingung 'Fortschrittsanzeige' (9).

% Investierte Zeit
Betrachtet man die Zeit, die die Teilnehmer im Mittel in das Experiment investiert haben, fällt auf, dass die Kontrollgruppe mit XX Minuten den höchsten Mittelwert erreicht (vergleiche auch XX). Allerdings ist der Unterschied zu der Bedingung 'Abzeichen' mit XX Minuten marginal.  Auffällig ist auch hier die vergleichsweise sehr geringe Zeit, die Probanden der Versuchsbedingung 'Fortschrittsanzeige' im Mittel aufbrachten. Diese brachten mit XX Minuten lediglich XX\% der Zeit auf, die von Teilnehmern der Kontrollgruppe investiert wurden. Betrachtet man auch hier den Median, zeigt sich ein ähnliches Verhältnis wie bei der mittleren Anzahl abgesetzter Befehle. Die Bedingung 'Abzeichen' investierte mit XX Minuten mehr Zeit als Kontrollgruppe (XX Minuten) und Versuchsbedingung 'Fortschrittsanzeige' (XX Minuten). Zusammenfassend lässt sich sagen, dass Probanden der Experimentalbedingung 'Abzeichen' die meiste Zeit in das Experiment investierten und gleichzeitig die meisten Befehle ausprobierten. Dem gegenüber zeichnen sich die Probanden der Bedingung 'Fortschrittsanzeige' durch eine geringe Zeitinvestition und deutlich weniger Befehle aus.