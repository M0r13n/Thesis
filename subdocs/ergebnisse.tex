%-------------------------------------------------------------------------------
% Methoden
%-------------------------------------------------------------------------------
\section{Ergebnisse}
Nachfolgend erfolgt die Untersuchung der Ergebnisse des Experiments.

\subsection{Beschreibung der Stichprobe}\label{beschreibung}
Das Experiment startete am 05.07.2020 und endete am 11.09.2020. Bis zum 05.08.2020 kamen jedoch nur 39 Teilnehmer zusammen. Aus diesem Grund wurde am 05.08.2020 entschlossen, das Experiment auf Facebook und in dem Forum Hardwareluxx\footnote{www.hardwareluxx.de} zu teilen. Zusätzlich wurde das Experiment auf Reddit in dem Subreddit \textit{/r/takemysurvey/} und auf HackerNews\footnote{news.ycombinator.com/} beworben. Leider wurde die Studie als Eigenwerbung bewertet und nach wenigen Stunden auf HackerNews gelöscht. Da bis zum 05.09.2020 immer noch nicht genug Teilnehmer an dem Experiment teilnahmen, wurde die Studie erneut auf der Internetplattform HackerNews geteilt und zusätzlich durch Dr. rer. pol., Dipl.-Psych. Athanasios Mazarakis auf der Mensch und Computer 2020 (MuC) geteilt. Diese Kombination führte zu einem sprunghaften Anstieg der Teilnehmerzahl. So kamen innerhalb von vier Tagen 141 neue Teilnehmer hinzu.

Das Experiment war für die gesamte Dauer der Studie durchgehend erreichbar. In diesem Zeitraum nahmen insgesamt 279 Teilnehmer an dem Experiment teil. Zunächst wurden 13 Einträge aus dem Datensatz entfernt, die in der folgenden Auswertung nicht berücksichtigt werden. Bei diesen Einträgen handelt es sich um Mehrfachteilnahmen, die sich durch eine identische IP-Adresse sowie identische User-Agents auszeichneten. Es wurde in diesem Fall lediglich die erstmalige Teilnahme berücksichtigt. Sämtliche nachfolgenden Einträge wurde aus dem Datensatz eliminiert. Zusätzlich wurden Einträge gelöscht, bei denen der Proband keinen Befehl abgeschickt hat. Damit hat der bereinigte Datensatz einen Umfang von 266 Teilnehmern. Davon haben 18 Personen das Experiment vollständig durchlaufen und sämtliche Fragen korrekt beantwortet.

Insgesamt gaben 210 Teilnehmer an, männlich zu sein. Von den übrigen Probanden waren 28 weiblich und 27 Probanden gaben ein anderes Geschlecht an. Der Großteil der Teilnehmer gab an, über sehr gute (44 \%) oder gute (41 \%) Englischkenntnisse zu verfügen. 22 (8 \%) verfügten über schlechte Kenntnisse und 17 (6 \%) verfügten über keinerlei englische Sprachkenntnisse. Die Altersverteilung der Versuchspersonen ist wie folgt: 100 Probanden waren 18-29 Jahre alt, 92  29-39 Jahre,  42 39-59 Jahre, 14 weniger als 18 Jahre und 18 gaben an, älter als 59 Jahre zu sein. Damit waren 88 \% der Teilnehmer zwischen 18 und 59 Jahre alt. Davon nutzt ein Großteil die Kommandozeile mindestens einmal im Monat (46 \% täglich, 23 \% wöchentlich, 9 \% monatlich). Demgegenüber nutzen 11 \% die Kommandozeile selten (1x im Jahr oder weniger) oder gar nicht (10 \%). Im wesentlichen setzt sich die Stichprobe aus überwiegend jungen und hauptsächlich männlichen Versuchspersonen zusammen. Diese verfügen über gute Englischkenntnisse und haben bereits erste Erfahrungen mit der Kommandozeile haben.

\subsection{Deskriptive Statistik}
Im Mittel verbrachten die Teilnehmer 527 Sekunden (SD = 1588) mit dem Experiment und setzten dabei durchschnittlich 19.5 Befehle ab (SD = 23.0). Ein besonders motivierter Teilnehmer investierte fast 5 Stunden in das Experiment und verwendete dabei 84 unterschiedliche Befehle. Kein anderer Teilnehmer verbrachte mehr Zeit mit dem Experiment.

% Anzahl Befehle
\begin{figure}[htbp]
    \centering
    \includegraphics[width=0.5\textwidth]{img/auswertung/mean_subs.png}
    \caption{Mittelwerte der Gesamtzahl abgesetzter Befehle für die unterschiedlichen Versuchsbedingungen.}
    \label{mean_subs}
\end{figure}

Die Kontrollgruppe nutze mit durchschnittlich 19.8 Befehlen die meisten Kommandos (SD = 26.7). Die Versuchsbedingung 'Abzeichen' weist im Vergleich eine durchschnittliche Anzahl abgesetzter Befehle von 18.4 (SD = 23.8) auf, während es Probanden aus der Bedingung 'Fortschrittsanzeige' auf durchschnittlich 15.4 Befehle bringen (SD = 16.3). Die Mittelwerte und das zugehörige Konfidenzintervall sind in Abbildung \ref{mean_subs} dargestellt. Betrachtet man den Median der einzelnen Versuchsbedingungen, ist die Kontrollgruppe mit 11 Kommandos pro Teilnehmer weiterhin die aktivste Bedingung. Allerdings sind Probanden der Experimentalgruppe 'Fortschrittsanzeige' im Median (10 Befehle) aktiver als Teilnehmer, die ein Abzeichen erhielten (8.5 Befehle). Der Median der erfolgreich gelösten Aufgaben liegt bei den beiden Experimentalbedingungen bei 3 Aufgaben und bei der Kontrollgruppe bei 4. Somit unterscheiden sich die drei Versuchsbedingungen hinsichtlich der Anzahl beantworteter Fragen und dabei genutzter Konsolenbefehle nicht wesentlich voneinander. Auffällig ist die hohe Standardabweichung der genutzten Befehle in den einzelnen Bedingungen. Besonders die die Gruppe mit Abzeichen von die Kontrollgruppe streuen stark um den Mittelwert und haben damit häufig deutlich mehr oder deutlich weniger mit der Anwendung interagiert. Dies erklärt warum die Gruppe mit einer Fortschrittsanzeige im Mittel am wenigsten Befehle verwendete, aber einen im Median mehr Befehle nutzen als die Versuchsbedingung mit Abzeichen: Bei der Fortschrittsanzeige liegen schlicht weniger Ausreißer vor.  

% Investierte Zeit
\begin{figure}[htbp]
    \centering
    \includegraphics[width=0.5\textwidth]{img/auswertung/mean_time.png}
    \caption{Mittlere Gesamtspielzeit, die die unterschiedlichen Versuchsbedingungen in das Experiment investierten.}
    \label{mean_time}
\end{figure}

Betrachtet man die Zeit, die die Teilnehmer im Mittel in das Experiment investiert haben, fällt auf, dass die Teilnehmer der Versuchsbedingung 'Abzeichen' mit 654 Sekunden (SD = 2190) deutlich mehr Zeit als die Kontrollgruppe (M = 447 Sekunden; SD = 1161 Sekunden) investierten. Letztere hat wiederum mehr Zeit investiert, als die Probanden der Versuchsbedingung 'Fortschrittsanzeige'. Diese brachten 330 Sekunden (Standardabweichung 740) für das Experiment auf. Die entsprechenden Mittelwerte sind in Abbildung \ref{mean_time} dargestellt. Erneut zeigt sich eine, durch das große Streuungsintervall (siehe auch Abbildung \ref{mean_time}) induzierte, Verzerrung der Mittelwerte. Die jeweiligen Mediane der unterschiedlichen Bedingungen liegen deutlich näher beieinander, als es der Durchschnitt erwarten lässt. Der Median der Kontrollgruppe ist am höchsten (136). während Fortschrittsanzeige und Abzeichen im Median weniger Zeit investierten (AB = 109; FA = 116). Bemerkenswert ist die hohe Abbruchquote in den ersten 30 Sekunden: 83 Teilnehmer (31 \%) haben eine Gesamtspielzeit von unter einer halben Minute.

Aufgrund der hohen Diskrepanz zwischen Median und Mittelwert sowie der hohen Standardabweichung wurden extreme Ausreißer aus der weiteren Betrachtung ausgeschlossen. Bei diesen Ausreißern gehe ich davon aus, dass eine, von den Spielelementen unabhängige, Intrinsische Motivation vorliegt. Als Grenzwert wählte ich die zweifache Standardabweichung. Dies führt dazu, dass sämtliche Teilnehmer mit mehr als 3537 Sekunden Spieldauer aus dem Datensatz ausgeschlossen werden. Zusätzlich werden nur Teilnehmer berücksichtigt, die mindestens einen Befehl verwendet haben. Der so entstandene Datensatz hat einen Umfang von 240 (-26).

\subsection{Hypothesenüberprüfung}\label{hypo}
Die aufgestellten Hypothesen werden durch einen t-Test und eine einfaktorielle  ANOVA  Varianzanalyse überprüft. Zunächst sind die Voraussetzungen Normalverteilung und Varianzhomogenität zu prüfen. Da jede Versuchsbedingung einen Umfang von deutlich über 30 Probanden aufweist, kann von einer Normalverteilung ausgegangen werden. Um die Vorraussetzung der Varianzhomogenität zu prüfen, wurde ein Levene-Test durchgeführt. Bei diesem handelt es sich um einen Signifikanztest, welcher prüft, ob die Varianzen innerhalb von zwei oder mehr Grundgesamtheiten (Gruppen) gleich sind (H0). Daraus ergibt sich die Alternativhypothese (H1), dass mindestens ein Gruppenpaar ungleiche Varianzen besitzt. Befindet sich der p-Wert unterhalb  eines zuvor definierten Signifikanzniveaus sind die Unterschiede in den Varianzen der unterschiedlichen Stichproben signifikant. Folglich kann die Nullhypothese des Tests abgelehnt werden und es kann angenommen werden, dass die Varianzen ungleich sind. Für die Auswertung wurde ein übliches Signifikanzniveau von 5 \% gewählt.

\subsubsection{Überprüfung mittels t-Test}

\paragraph{Hypothese 1 }
\begin{center}
    \textit{Probanden, die Abzeichen erhalten (AB), beantworten im Mittel eine höhere Anzahl an Fragen als eine Kontrollgruppe ohne Abzeichen (KG).} 
\end{center}

Der Levene-Test für die Anzahl an beantworteten Fragen ergibt mit  p = .318 kein  statistisch  signifikantes  Ergebnis (Leven-Statistik = 1.005). Bei der folgenden Untersuchung kann daher von Varianzhomogenität ausgegangen werden. Es ist keine Korrektur der Ergebnisse erforderlich. Die Differenz der durchschnittlichen Anzahl beantworteter Fragen von Probanden der Experimentalbedingung 'Abzeichen' und Probanden der Kontrollgruppe ohne Spielelemente ist nicht signifikant (t(160) = -1.118; p = .867). Die Hypothese kann damit nicht bestätigt werden. Daher wurde die ursprüngliche Hypothese abgewandelt und es wurden die folgenden Hypothesen aufgestellt:

\begin{itemize}
    \item \textbf{1.A:} \textit{Probanden, die Abzeichen erhalten (AB), verbringen im Mittel mehr Zeit mit dem Experiment als eine Kontrollgruppe (KG) ohne Spielelemente.}
    \item \textbf{1.B:} \textit{Probanden, die Abzeichen erhalten (AB), benutzen durchschnittlich mehr Befehle als eine Kontrollgruppe (KG) ohne Spielemente.} 
\end{itemize}

Keine der zusätzlich aufgestellten Hypothesen hat einen auf dem 5\%-Niveau signifikanten p-Wert und es kann daher erneut angenommen werden, dass die Varianzen der einzelnen Gruppen gleich sind. Die Ergebnisse des entsprechenden Levene-Tests sind in Tabelle \ref{levene_hypo_1} dargestellt. Folglich kann der unkorrigierte t-Test (Student's t) angewendet werden. Ein Vergleich  der  Mittelwerte  der Gesamtspielzeit der Kontrollgruppe mit der Gruppe mit Abzeichen offenbart kein statistisch signifikantes Ergebnis (t(160) = 0.020; p = .492). Auch die durchschnittliche Befehlanzahl unterscheidet sich nicht signifikant zwischen den beiden Bedingungen, t(160) = -0.570 , p = .715. Somit kann weder Hypothese \textbf{H1} noch die abgewandelten Hypothesen \textbf{H1.A} und \textbf{H1.B} gestützt werden.


% ~*~*~*~*~*~*~*~*~*~*~*~*~*~*~*~ LEVENE - HYPO 1 ~*~*~*~*~*~*~*~*~*~*~*~*~*~*~*~
\begin{table}[htbp]
\centering
\caption{\textit{Levene-Test auf Varianzhomogenität für die Experimentalbedingung Abzeichen und die Kontrollgruppe (Hypothese 1).}}
\begin{tabular}{ p{4cm} p{2.0cm} p{2.0cm} p{2.0cm} p{2.0cm} }
 \hline
 & F & df1 &df2 &p \\
 \hline
  Befehlanzahl      & 0.044    & 1 &   160 & .834\\
  Gesamtspielzeit   & 0.110    & 1 &   160 & .741\\
  Gelöste Aufgaben  & 1.005    & 1 &   160 & .318\\
 \hline
 \multicolumn{5}{l}{%
 \small%
\textit{Anmerkungen}. * p < .05
}\\
\end{tabular}
\label{levene_hypo_1}
\end{table}
% ~*~*~*~*~*~*~*~*~*~*~*~*~*~*~*~ END ~*~*~*~*~*~*~*~*~*~*~*~*~*~*~*~
% ~*~*~*~*~*~*~*~*~*~*~*~*~*~*~*~ TEST - HYPO 1 ~*~*~*~*~*~*~*~*~*~*~*~*~*~*~*~
\begin{table}[htbp]
\centering
\caption{\textit{Zweistichproben-t-Test (Student's t) für die Experimentalbedingung Abzeichen und Kontrollgruppe (Hypothese 1).}}
\begin{tabular}{ p{4cm} p{2.0cm} p{2.0cm} p{2.0cm} }
 \hline
 & T &df & p \\
 \hline
  Befehlanzahl       & -0.570  &   160 & .715\\
  Gesamtspielzeit    &  0.020  &   160 & .492\\
  Gelöste Aufgaben   & -1.118  &   160 & .867\\
 \hline
 \multicolumn{4}{l}{%
 \small%
\textit{Anmerkungen}. $H_a:\: AB > KG$
}\\
\end{tabular}
\label{ttest_hypo_1}
\end{table}
% ~*~*~*~*~*~*~*~*~*~*~*~*~*~*~*~ END ~*~*~*~*~*~*~*~*~*~*~*~*~*~*~*~



\paragraph{Hypothese 2 }
\begin{center}
    \textit{Probanden, die eine Fortschrittsanzeige erhalten (FA), beantworten im Mittel eine höhere Anzahl an Fragen als eine Kontrollgruppe ohne Fortschrittsanzeige (KG).} 
\end{center}

Ein durchgeführter Levene-Test bestätigt mit p = .806 die Vermutung der Varianzhomogenität in den Gruppen. Mit t(150) = 0.373 und p = .355 ist die Differenz der durchschnittlichen Anzahl beantworteter Fragen von Probanden der Experimentalbedingung 'Fortschrittsanzeige' und Probanden der Kontrollgruppe nicht signifikant. Die für Hypothese 1 abgewandelten Hypothesen wurden auch für die Versuchsbedingung 'Fortschrittsanzeige' aufgestellt: 

\begin{itemize}
    \item \textbf{2.A:} \textit{Probanden, die eine Fortschrittsanzeige erhalten (FA), verbringen im Mittel mehr Zeit mit dem Experiment als eine Kontrollgruppe ohne Spielelemente (KG).}
    \item \textbf{2.B:} \textit{Probanden, die eine Fortschrittsanzeige erhalten (FA), benutzen durchschnittlich mehr Befehle als eine Kontrollgruppe ohne Spielelemente (KG).} 
\end{itemize}

 Wie zuvor sind die Varianzen in den Gruppen gleich, da der Levene-Test für keine der beiden Hypothesen signifikant ist (siehe Tabelle \ref{levene_hypo_2}). Ein durchgeführter t-Test ergibt bei einem Vergleich  der  Mittelwerte  der Gesamtspielzeit der Kontrollgruppe mit der Gruppe mit Fortschrittsanzeige kein statistisch signifikantes Ergebnis (t(150) = 0.822; p = .794). Selbiges gilt für die Differenz der Mittelwerte der beiden Gruppen im Bezug zu der Anzahl genutzter Befehle. Der entsprechende t-Test ist p = .137 nicht signifikant. Damit können Hypothese \textbf{H2} und die daraus abgeleiteten Hypothesen \textbf{H2.A} und \textbf{H2.B} nicht durch einen t-Test gestützt werden.


% ~*~*~*~*~*~*~*~*~*~*~*~*~*~*~*~ LEVENE - HYPO 2 ~*~*~*~*~*~*~*~*~*~*~*~*~*~*~*~
\begin{table}[htbp]
\centering
\caption{\textit{Levene-Test auf Varianzhomogenität für die Experimentalbedingung Fortschrittsanzeige und die Kontrollgruppe (Hypothese 2).}}
\begin{tabular}{ p{4cm} p{2.0cm} p{2.0cm} p{2.0cm} p{2.0cm} }
 \hline
 & F & df1 &df2 &p \\
 \hline
  Befehlanzahl      & 2.114     & 1 &   150 & .148\\
  Gesamtspielzeit   & 1.291     & 1 &   150 & .258\\
  Gelöste Aufgaben  & 0.061     & 1 &   150 & .806\\
 \hline
  \multicolumn{5}{l}{%
 \small%
\textit{Anmerkungen}. * p < .05
}\\
\end{tabular}
\label{levene_hypo_2}
\end{table}
% ~*~*~*~*~*~*~*~*~*~*~*~*~*~*~*~ END ~*~*~*~*~*~*~*~*~*~*~*~*~*~*~*~

% ~*~*~*~*~*~*~*~*~*~*~*~*~*~*~*~ TEST - HYPO 2 ~*~*~*~*~*~*~*~*~*~*~*~*~*~*~*~
\begin{table}[htbp]
\centering
\caption{\textit{Zweistichproben-t-Test (Student's t) für die Experimentalbedingung Abzeichen und Kontrollgruppe (Hypothese 1).}}
\begin{tabular}{  p{4cm} p{2.0cm} p{2.0cm} p{2.0cm}  }
 \hline
 & T &df & p \\
 \hline
  Befehlanzahl       & 1.099   &   150 & .863\\
  Gesamtspielzeit    & 0.822   &   150 & .794\\
  Gelöste Aufgaben   & 0.373   &   150 & .645\\
 \hline
 \multicolumn{4}{l}{%
 \small%
\textit{Anmerkungen}. $H_a:\: FA > KG$
}\\
\end{tabular}
\label{ttest_hypo_2}
\end{table}
% ~*~*~*~*~*~*~*~*~*~*~*~*~*~*~*~ END ~*~*~*~*~*~*~*~*~*~*~*~*~*~*~*~



% ~*~*~*~*~*~*~*~*~*~*~*~*~*~*~*~ ANOVA ~*~*~*~*~*~*~*~*~*~*~*~*~*~*~*~


\subsubsection{Überprüfung mittels ANOVA Varianzanalyse }
Bei der ANOVA (engl. \textit{Analysis of Variance}) handelt es sich um eine Varianzanalyse, die die Mittelwerte von mehr als 2 Gruppen vergleichbar macht. Dabei handelt es sich um eine Erweiterung des t-Tests, welcher auf den Vergleich von maximal 2 Stichproben beschenkt ist. Dies ermöglicht den Vergleich aller drei Versuchsbedingungen. Nachfolgend werden die zwei abhängigen Variablen \textbf{Gelöste Aufgaben} und \textbf{Gesamtspielzeit} jeweils durch eine einfaktorielle Varianzanalyse untersucht.

% Varianzhomogenität prüfen 
Da ein Levene-Test sowohl für die Anzahl der gelösten Aufgaben (p = .606) als auch für die Gesamtspielzeit (p = .288) kein signifikantes Ergebnis ergibt, kann von Varianzhomogenität in den Gruppen ausgegangen werden. Eine nachträgliche Korrektur der Ergebnisse ist entsprechend nicht erforderlich. Die Varianzanalyse zeigt keinen statistisch signifikanten Unterschied im Zusammenhang der gelösten Aufgaben zwischen den einzelnen Versuchsbedingungen, F (2.237) = 0.626; p = .536. Selbiges gilt für die Gesamtspielzeit: Die Varianzanalyse zeigt zudem keinen statistisch  signifikanten Unterschied zwischen den einzelnen Versuchsbedinungen hinsichtlich der Spielzeit,  F(2.237) = 0.453, p = .636.


% ~*~*~*~*~*~*~*~*~*~*~*~*~*~*~*~ END ~*~*~*~*~*~*~*~*~*~*~*~*~*~*~*~

\subsection{Zwischenfazit t-Test und ANOVA}
Die geschilderten Ergebnisse entsprechen nicht meinen ursprünglichen Erwartungen. Tatsächlich kann weder durch einen t-Test noch durch eine ANOVA ein signifikanter Effekt auf die Anzahl gelöster Aufgaben, die Menge der abgesetzten Befehle oder die Gesamtspielzeit nachgewiesen werden. Als Konsequenz kann keine meiner ursprünglich aufgestellten Hypothesen bestätigt werden. Die Spielelemente Abzeichen und Fortschrittsbalken zeigen damit in meinem Experiment keinerlei statisch signifikanten, positiven Effekt auf das Durchhaltevermögen (gemessen durch die Gesamtspielzeit) der Teilnehmer. Auch lässt sich keine erhöhte Aktivität der Probanden feststellen, da sich die Versuchsbedingungen hinsichtlich der gelösten Aufgaben und abgesetzten Befehle nicht signifikant unterscheiden. Stattdessen kann im Fall der Fortschrittsanzeige ein tendenziell negativer Einfluss beobachtet werden. Dieser zeigt sich insbesondere in der Gesamtspielzeit, welche im Mittel deutlich geringer ist als in den Alternativbedingungen. 

% ~*~*~*~*~*~*~*~*~*~*~*~*~*~*~*~ Weiterführende Analyse ~*~*~*~*~*~*~*~*~*~*~*~*~*~*~*~
\subsection{Weiterführende Analyse}
Um die Daten besser zu verstehen und um die Ergebnisse erklärbar zu machen, habe ich die Daten ausführlicher untersucht: Der Verlauf des Experiments kann zeitlich in zwei Abschnitte unterteilt werden (siehe Abschnitt \ref{beschreibung}). Da anzunehmen ist, dass ab dem 06.09.2020 sehr versierte Teilnehmer, die bereits Erfahrung mit der Kommandozeile haben, an dem Experiment teilnahmen, habe ich beschlossen, das Experiment zeitlich in zwei disjunkte Abschnitte zu unterteilen. Dazu teilte ich den Datensatz in einen Abschnitt vor dem 06.09.2020 und einen Abschnitt ab dem 06.09.2020. Der erste Abschnitt besteht damit aus Teilnehmern, die bis zum 05.09.2020 um 23:59:59 an dem Experiment teilgenommen haben, und wird nachfolgend als \textbf{Gruppe 1} bezeichnet. Entsprechend setzt sich die andere Hälfte aus Teilnehmern zusammen, die nach dem 06.09.2020 00:00 Uhr teilnahmen, und wird als \textbf{Gruppe 2} bezeichnet. Teilnehmer ohne Interaktion (keine abgeschickten Befehle) werden erneut nicht berücksichtigt.

\subsubsection{Weiterführende Analyse - Zusammensetzung der Gruppen}
Zunächst wurden die demographischen Daten der beiden Gruppen miteinander verglichen. Die vollständige Gegenüberstellung ist in Tabelle \ref{demo_g12} dargestellt. Beide Stichproben setzen sich zu einem Großteil aus männlichen Probanden zusammen (> 80 \%). Der Anteil weiblicher und diverser Teilnehmer ist zwischen den Gruppen nahezu invers. So sind in Gruppe 1 13 \% weiblich und 6\% divers, während in Gruppe 2 6 \% weiblich und 14\% divers sind. Gruppe 1 verfügt durchschnittlich über weniger gute Englisch Kenntnisse als Gruppe 2 (35 \% sehr gut vs 52 \% sehr gut). Gleichzeitig ist der Altersdurchschnitt in Gruppe 2 höher als in Gruppe 1. So ist der Großteil der Probanden aus Gruppe 2 zwischen 20 und 39 Jahre alt (39 \%). Demgegenüber gaben die Teilnehmer der ersten Gruppe überwiegend ein Alter zwischen 18 und 29 Jahren an (49\%). Die Probanden aus Gruppe 2 nutzen die Kommandozeile zudem deutlich häufiger. Beispielsweise nutzen 66 \% der Probanden aus Gruppe 2 das Terminal täglich, während dies in Gruppe 1 lediglich 2 \% sind. Damit ergeben sich die folgenden Profile für den jeweiligen Versuchsabschnitt:

Gruppe 1 setzt sich aus jungen, überwiegend männlichen Teilnehmern zusammen, die über gute Englischkenntnisse verfügen und bereits erste Erfahrungen mit der Kommandozeile haben. 

Gruppe 2 setzt sich aus überwiegend männlichen Teilnehmern mittleren Alters zusammen, die über gute bis sehr gute Englischkenntnisse verfügen und die die Kommandozeile häufig nutzen. 

% ~*~*~*~*~*~*~*~*~*~*~*~*~*~*~*~ Vergleich Demographie ~*~*~*~*~*~*~*~*~*~*~*~*~*~*~*~
\begin{table}[htbp]
\centering
\caption{\textit{Vergleich Demographie zwischen Gruppe 1 und Gruppe 2}}
\begin{tabular}{  p{6cm} p{3cm} p{3cm}  }
  \hline
  Merkmal & Gruppe 1 & Gruppe 2 \\
  \hline
  Umfang (absolut) & 121  & 123 \\
  \hline
  \multicolumn{3}{c}{Geschlecht} \\
  \hline
  Weiblich                      & 12.5 \%       & 5.7  \%       \\
  Männlich                      & 81.7 \%       & 80.5 \%       \\
  Divers                        & 5.8  \%       & 13.8 \%       \\
  \hline
  \multicolumn{3}{c}{Englischkenntnisse} \\
  \hline
  Sehr gut                      & 34.7 \%       & 52.0 \%      \\
  Gut                           & 48.8 \%       & 35.0 \%       \\
  Nicht so gut                  & 11.6 \%       & 6.5 \%        \\
  Nicht vorhanden               & 5.0  \%       & 6.5 \%        \\
  \hline
  \multicolumn{3}{c}{Alter} \\
  \hline
  <18                           & 6.6  \%       & 3.3  \%       \\
  18-29                         & 48.8 \%       & 27.6 \%       \\
  29-39                         & 32.2 \%       & 39.0 \%       \\
  39-59                         & 9.1  \%       & 22.0 \%       \\
  59+                           & 3.3  \%       & 8.1 \%        \\
  \hline
  \multicolumn{3}{c}{Häufigkeit der Nutzung der Kommandozeile} \\
  \hline
  Täglich (1x pro Tag)          & 28.9 \%       & 65.9  \%      \\
  Gelegentlich (1x pro Woche)   & 34.7 \%       & 13.8  \%      \\
  Selten (1x pro Monat)         & 11.6 \%       & 7.3  \%       \\
  Sehr selten (1x pro Jahr)     & 17.4 \%       & 4.9  \%       \\
  Nie                           & 7.4  \%       & 8.1  \%       \\
  \hline
\end{tabular}
\label{demo_g12}
\end{table}
% ~*~*~*~*~*~*~*~*~*~*~*~*~*~*~*~ END ~*~*~*~*~*~*~*~*~*~*~*~*~*~*~*~

\subsubsection{Weiterführende Analyse - Hypothesenprüfung}


Im folgenden Abschnitt wurden die beiden Gruppen gesondert untersucht und analysiert. Wie zuvor wurden extreme Ausreißer sowie Teilnehmer ohne Interaktion (keinen Befehl gesendet) aus den Datensätzen eliminiert. Der jeweilige Grenzwert ergibt sich erneut aus dem Mittelwert zuzüglich der 2-fachen Standardabweichung. In der Stichprobe, die bis zum 06.09.2020 zu Stande gekommen ist, liegt der Grenzwert der Spielzeit bei 5089 Sekunden. Teilnehmer, deren Gesamtspielzeit diesen Grenzwert übersteigt, wurden nachfolgend nicht berücksichtigt. Der so entstandene Datensatz hat einen Umfang von 117 Personen. Der entsprechende Schwellwert lag im Fall der zweiten Gruppe bei 1360 Sekunden. Dies führte zu einer Stichprobengröße von 116 Personen. Ziel war es, meine aufgestellten und in Abschnitt \ref{hypo} angepassten Hypothesen für die jeweilige Gruppe zu prüfen. Es wurden für jede Gruppe folgende Hypothesen geprüft:

\paragraph{Hypothese 1}
\begin{itemize}
    \item \textbf{1.A:} \textit{Probanden, die Abzeichen erhalten (AB), verbringen im Mittel mehr Zeit mit dem Experiment als eine Kontrollgruppe (KG) ohne Spielelemente.}
    \item \textbf{1.B:} \textit{Probanden, die Abzeichen erhalten (AB), senden im Mittel mehr Befehle ab, als eine Kontrollgruppe (KG).}
    \item \textbf{1.C:} \textit{Probanden der Experimentalbedingung 'Abzeichen' (AB) lösen mehr Aufgaben als Personen der Kontrollgruppe.} 
\end{itemize}

\paragraph{Gruppe 1}
Die Experimentalbedingung Abzeichen unterschied sich mit $ t(71)=-0.325 $ und $p=.746$ nicht signifikant von der Kontrollgruppe hinsichtlich der absoluten Anzahl genutzter Befehle. Auch die durchschnittliche Spielzeit zwischen Personen, die ein Abzeichen erhielten, und der Kontrollgruppe unterschied sich nicht signifikant ($t(71)=-0.396; p=.696$). Selbiges gilt für die Aufgabenzahl. Das Spielelement Abzeichen führte zu keinem signifikanten Unterschied der beantworteten Fragen ( $ t(71)=-1.510; p=.135 $ )p. Damit kann keine Hypothese bestätigt werden und es kann innerhalb der ersten Stichprobe nicht von einem motivierenden Effekt des Abzeichens ausgegangen werden.

\paragraph{Gruppe 2}
Es konnte bei der zweiten Gruppe kein motivierender Einfluss durch ein Abzeichen beobachtet werden. Weder die Anzahl der genutzten Befehle ( $ t(82)=-0.4926; p=0.624 $ ) noch die Gesamtspielzeit ($t(82)=-0.0767; p=0.939$) oder die gelösten Aufgaben ( $t(82)=-0.3709; p=0.712$ ) unterschied sich signifikant. Daher kann erneut keine Hypothese bestätigt werden und es kann nicht davon ausgegangen werden, dass Teilnehmer aus der zweiten Stichprobe durch Abzeichen motiviert wurden. 


\paragraph{Hypothese 2}
\begin{itemize}
    \item \textbf{2.A:} \textit{Probanden, die eine Fortschrittsanzeige erhalten (FA), verbringen im Mittel mehr Zeit mit dem Experiment als eine Kontrollgruppe (KG) ohne Spielelemente.}
    \item \textbf{2.B:} \textit{Probanden, die eine Fortschrittsanzeige erhalten (FA), senden im Mittel mehr Befehle ab, als eine Kontrollgruppe (KG).}
    \item \textbf{2.C:} \textit{Probanden der Experimentalbedingung 'Fortschrittsanzeige' (FA) lösen mehr Aufgaben als Personen der Kontrollgruppe.} 
\end{itemize}

\paragraph{Gruppe 1}
Bei einem Vergleich der Experimentalbedingung Fortschrittsanzeige und der Kontrollgruppe lag Varianzheterogenität vor (Levene-Tests: $p\in\{.004, .030, .003\}$). Daher wurde ein Welch-Test durchgeführt. Die durchschnittliche Spielzeit von der Experimentalbedingung (M = 248) unterschied sich messbar von der Kontrollgruppe (M = 481). Jedoch war der Unterschied statisch nicht signifikant, t(50) = 1.45; p = .15. Auch die Differenz genutzter Befehle von Experimentalbedingung und Kontrollgruppe war nicht signifikant, t(40.5)=1.9; p=.065. Die durchschnittliche Anzahl gelöster Aufgaben war in der Stichprobe mit Fortschrittsanzeige (M = 3) niedriger als in der Kontrollgruppe (M = 4.73). Der Differenz war signifikant: t(47.2) = 2, p = .05. Dies führt dazu, dass keine Hypothese bestätigt werden kann. Stattdessen kann gezeigt werden, dass Probanden der Experimentalbedingung 'Fortschrittsanzeige' (FA) weniger Aufgaben lösten als die Kontrollgruppe. 


\paragraph{Gruppe 2}
Teilnehmer, die eine animierte Fortschrittsanzeige angezeigt bekamen, investierten durchschnittlich 274 Sekunden in das Experiment und nutzen durchschnittlich 23.25 Befehle. Die Kontrollgruppe investierte vergleichsweise wenig Zeit (M = 202) in den Versuch und nutze ebenfalls weniger Befehle (M = 18.07 ). Ein t-Test ergab, dass die Differenz der gelösten Aufgaben statistisch signifikant ist, t(72) = -2.16, p=.034, und stützt Hypothese \textbf{2.C}. Demgegenüber sind mit p = .217 und p = .185 die Unterschiede der Spielzeit und Befehlsanzahl nicht signifikant.\\
 
Zusammengefasst lässt sich sagen, dass sich die Ergebnisse in der zweiten Stichprobe im wesentlichen mit einen Erwartungen decken. Die Ergebnisse der ersten Gruppe hingegen weichen deutlich von meinen ursprünglichen Hypothesen ab. Die Unterschiede zwischen den zwei Stichproben sind dabei stellenweise extrem. So zeigt eine Fortschrittsanzeige in beiden Stichproben einen deutlichen Effekt auf das Durchhaltevermögen der Teilnehmer: In Gruppe 1 muss von einem demotivierendem Effekt der Fortschrittsanzeige ausgegangen werden, während in Gruppe 2 ein motivierender Effekt zu beobachten war. In beiden Fällen handelt es sich dabei um eine statisch signifikante Differenz. Generell weichen die Daten der Gruppen deutlich voneinander ab. Eine entsprechende Gegenüberstellung findet sich in Tabelle \ref{final}. Besonders auffällig sind die deutlich höheren maximalen Spielzeiten in Gruppe 1 gegenüber Gruppe 2. Personen der ersten Stichprobe investierten in jeder Versuchsbedingung im Maximum $\sim3$ Mal mehr Zeit für das Experiment, obwohl extreme Ausreißer nicht berücksichtigt wurden. Als Konsequenz sind die entsprechenden Standardabweichungen deutlich größer als die der zweiten Gruppe.


\begin{table}[htbp]
\centering
\caption{\textit{Gegenüberstellung der Gesamtspielzeit, Anzahl genutzter Befehle und Aufgabenanzahl zwischen Gruppe 1 und Gruppe 2.}}
\begin{tabular}{ p{2.5cm}  p{3.5cm}  p{0.85cm} p{0.85cm}  p{0.85cm}  p{0.85cm} p{0.85cm} p{0.85cm}}
 \hline
 & & \multicolumn{3}{c}{Gruppe 1} & \multicolumn{3}{c}{Gruppe 2}\\
 & & KG & AB & FA & KG & AB & FA\\
 \hline
  Gesamtspielzeit   & Mittel                    & 482   & 417   & 248   & 202  & 199  & 275   \\
                    & Median                    & 181   & 157   & 55.5  & 140  & 105  & 171   \\
                    & Standardabweichung        & 753   & 629   & 557   & 252  & 252  & 260   \\
                    & Min                       & 0     & 0     & 0     & 0    & 0    & 0     \\
                    & Max                       & 3300  & 2847  & 3420  & 933  & 1108 & 1012  \\
 \hline
  Aufgaben          & Mittel                    & 4.73  & 3.49  & 3     & 4.74 & 4.48   & 6.5     \\
                    & Median                    & 3.5   & 3.0   & 2.5   & 4    & 4      & 5.5     \\
                    & Standardabweichung        & 4.14  & 2.01  & 2.83  & 3.32 & 3.15   & 3.65     \\
                    & Min                       & 0     & 0     & 0     & 0    & 0      & 1     \\
                    & Max                       & 13    & 13    & 13    & 13   & 13     & 13     \\
  \hline
  Befehlsanzahl     & Mittel                    & 18.8 & 17    & 10.7  & 18.1   & 16.3   & 23.3     \\
                    & Median                    & 10   & 8     & 5.5   & 13.5   & 9      & 16.5     \\
                    & Standardabweichung        & 21.4 & 24.5  & 11.5  & 17.6   & 15.6   & 17.9     \\
                    & Min                       & 1    & 1     & 1     & 1      & 2      & 1     \\
                    & Max                       & 78   & 137   & 48    & 71     & 73     & 62     \\

  
 \hline
 \multicolumn{8}{l}{%
 \small%
\textit{Anmerkungen}. Extreme Ausreißer werden nicht berücksichtigt.
}\\
\end{tabular}
\label{final}
\end{table}


\subsubsection{Feedback der Teilnehmer}
Wurden alle Fragen korrekt beantwortet, wurden die Teilnehmer gebeten, Feedback hinsichtlich der empfundenen Motivation zu geben. Von 18 Teilnehmern, die das Experiment beendeten, gaben 17 eine entsprechende Rückmeldung ab. Wie in Tabelle \ref{feedback} zu sehen, gab über die Hälfte der Teilnehmer an, tendenziell motiviert gewesen zu sein. Lediglich drei Teilnehmer gaben an, sich eher nicht motiviert gefühlt zu haben. Aufgrund des geringen Umfangs der Stichprobe kann keine statistisch belastbare Aussage auf Basis des Feedbacks getroffen werden. Jedoch kann angenommen werden, dass sich Teilnehmer, die das Experiment beendeten, tendenziell motiviert fühlten.

\begin{table}[htbp]
\centering
\caption{\textit{Tabellarische Darstellung des finalen Feedbacks der Probanden.}}
\begin{tabular}{ p{4cm}   p{2cm}}
 \hline
 Feedback & Anzahl \\
 \hline
 trifft zu & 6 \\
 trifft eher zu & 5 \\
 teils-teils & 3 \\
 trifft eher nicht zu & 3 \\
 trifft nicht zu & 0 \\
 \hline
\end{tabular}
\label{feedback}
\end{table}