%-------------------------------------------------------------------------------
% Methoden
%-------------------------------------------------------------------------------
\section{Ergebnisse}
Nachfolgend erfolgt die Untersuchung der Ergebnisse des Experiments.

\subsection{Beschreibung der Stichprobe}\label{beschreibung}
Die Studie startete am 05.07.2020 und endete am 14.09.2020. Das Experiment war für die gesamte Dauer der Studie durchgehend erreichbar. In diesem Zeitraum nahmen 279 Teilnehmer an dem Experiment teil. Zunächst wurden 13 Einträge aus dem Datensatz entfernt, die in der folgenden Auswertung nicht berücksichtigt werden. Bei diesen Einträgen handelt es sich um Mehrfachteilnahmen, die sich durch eine identische IP-Adresse sowie identische User-Agents auszeichneten. Es wurde in diesem Fall lediglich die erstmalige Teilnahme berücksichtigt. Sämtliche nachfolgenden Einträge wurde aus dem Datensatz eliminiert. Zusätzlich wurden Einträge gelöscht, bei denen der Proband keinen Befehl abgeschickt hat. Damit hat der bereinigte Datensatz einen Umfang von 244 Teilnehmern. Davon haben 18 Personen das Experiment vollständig durchlaufen und sämtliche Fragen korrekt beantwortet.

Von den verbleibenden Versuchspersonen gaben 81.1 \% der Teilnehmer an, männlich zu sein. 9.1 \% gaben ein weibliches Geschlecht und 9.9 \% ein anderes Geschlecht an. Der Großteil der Teilnehmer gab an, über sehr gute (43.4 \%) oder gute (41.8 \%) Englischkenntnisse zu verfügen. Lediglich 9.0 \% verfügten über schlechte Kenntnisse und 5.7 \% verfügten über keinerlei englische Sprachkenntnisse. Die Altersverteilung der Versuchspersonen ist wie folgt: 38.1 \% gaben 18-29 Jahre an, 35.7 \%  29-39 Jahre,  15.6 \% 39-59 Jahre, 4.9 \% weniger als 18 Jahre und 5.7 \% gaben an, älter als 59 Jahre zu sein. Damit waren zum Zeitpunkt des Experiments 89.3 \% der Teilnehmer zwischen 18 und 59 Jahre alt. Davon nutzt ein Großteil die Kommandozeile mindestens einmal im Monat (47.5 \% täglich, 24.2 \% wöchentlich, 9.4 \% monatlich). Demgegenüber nutzen 11.1 \% die Kommandozeile selten (1x im Jahr oder weniger) oder gar nicht (7.8 \%). Damit setzt sich die Stichprobe aus überwiegend jungen Versuchspersonen, die über gute Englischkenntnisse verfügen und bereits erste Erfahrungen mit der Kommandozeile haben, zusammen.

\subsection{Deskriptive Statistik}

\begin{figure}[htbp]
    \centering
    \includegraphics[width=0.5\textwidth]{img/auswertung/density.png}
    \caption{Die Verteilungskurve aller Teilnehmer hinsichtlich der Gesamtspielzeit.}
    \label{density}
\end{figure}

Im Mittel verbrachten die Teilnehmer 8 Minuten und 46 Sekunden (527 Sekunden, $SD = 1588$ Sekunden) mit dem Experiment und setzten dabei durchschnittlich 19.5 Befehle ab (SD= 23.0). Sowohl bei der Gesamtdauer als auch bei der Anzahl der abgesendeten Befehle gibt es extreme Ausreißer nach oben. So investierte ein Teilnehmer fast 5 Stunden in das Experiment und verwendete dabei 84 unterschiedliche Befehle. Diese enorme Streuung nach oben wird auch in Abbildung \ref{density} deutlich. Aus diesem Grund wird nachfolgend zusätzlich der Median betrachtet, da dieser robuster gegenüber Ausreißern ist.

% Anzahl Befehle
\begin{figure}[htbp]
    \centering
    \includegraphics[width=0.5\textwidth]{img/auswertung/mean_subs.png}
    \caption{Mittelwerte für die Gesamtzahl abgesetzter Befehle. EB= Experimentalbedingung Abzeichen, KG= Kontrollgruppe und EP = Experimentalbedingung Fortschrittsanzeige.}
    \label{mean_subs}
\end{figure}

Im Mittel nutzen Teilnehmer der Kontrollgruppe mit 21.9 Befehlen die meisten Kommandos ($SD = 27.3$). Die Versuchsbedingung 'Abzeichen' weist im Vergleich eine durchschnittliche Anzahl abgesetzter Befehle von 19.6 (SD = 24.1) auf, während es Probanden aus der Bedingung 'Fortschrittsanzeige' auf durchschnittlich 16.9 Befehle bringen ($SD=16.3$). Die Mittelwerte sind in Abbildung \ref{mean_subs} dargestellt. Hier fällt erneut das breite Streuungsintervall auf. Betrachtet man stattdessen den Median der einzelnen Versuchsbedingungen, bleibt die Kontrollgruppe mit 13 Kommandos pro Teilnehmer die aktivste Bedingung. Allerdings sind Probanden der Experimentalgruppe 'Fortschrittsanzeige' im Median ($\tilde x = 11$ Befehle) aktiver als Teilnehmer, die ein Abzeichen erhielten ($\tilde x = 9$ Befehle). Der Median der erfolgreich gelösten Aufgaben liegt im Fall der Experimentalbedingungen bei 3 Aufgaben und bei der Versuchsbedingung bei 4.

% Investierte Zeit
\begin{figure}[htbp]
    \centering
    \includegraphics[width=0.5\textwidth]{img/auswertung/mean_time.png}
    \caption{Mittelwerte für die Gesamtzeit, die die Teilnehmer in das Experiment investierten. EB= Experimentalbedingung Abzeichen, KG= Kontrollgruppe und EP = Experimentalbedingung Fortschrittsanzeige.}
    \label{mean_time}
\end{figure}

Betrachtet man die Zeit, die die Teilnehmer im Mittel in das Experiment investiert haben, fällt auf, dass die Teilnehmer der Versuchsbedingung 'Abzeichen' mit 797 Sekunden (SD= 2256 Sekunden) deutlich mehr Zeit als die Kontrollgruppe (	$\bar{x} = 495$ Sekunden; SD = 1212 Sekunden) investierten. Letztere hat wiederum mehr Zeit investiert, als die Probanden der Versuchsbedingung 'Fortschrittsanzeige'. Diese brachten es auf vergleichsweise kurze 364 Sekunden (SD= 769 Sekunden). Die entsprechenden Mittelwerte sind in Abbildung \ref{mean_time} dargestellt. Betrachtet man auch hier den Median und nicht die Mittelwerte, ändert sich das Ergebnis deutlich. Die jeweiligen Mediane der unterschiedlichen Bedingungen liegen deutlich näher beieinander. So bringen es Teilnehmer der Bedingung 'Abzeichen' auf 149 Sekunden (~2.5 Minuten) und bringen damit ähnlich viel Zeit für das Experiment wie die Probanden der Kontrollgruppe (154 Sekunden) auf. Etwas weniger Zeit wurde im Durchschnitt von Teilnehmern investiert, die eine Fortschrittsanzeige angezeigt bekamen ($\tilde x = 132$ Sekunden).

Aufgrund der hohen Diskrepanz zwischen Median und Mittelwert sowie der extrem hohen Standardabweichung habe ich beschlossen, extreme Ausreißer aus der weiteren Betrachtung auszuschließen. Bei diesen Ausreißern gehe ich davon aus, dass eine, von den Spielelementen unabhängige, Intrinsische Motivation vorliegt. Als Grenzwert wählte ich $\pm 2\sigma$. Dies führt dazu, dass sämtliche Teilnehmer mit mehr als 3703 Sekunden Spieldauer aus dem Datensatz ausgeschlossen werden. Damit reduziert sich der Datensatz im Folgenden auf einen Umfang von 240 Probanden.

\subsection{Hypothesenüberprüfung}\label{hypo}
Die aufgestellten Hypothesen werden durch einen t-Test und eine einfaktorielle  ANOVA  Varianzanalyse überprüft. Zunächst sind die Voraussetzungen Normalverteilung und Varianzhomogenität zu prüfen. Nur wenn diese erfüllt sind, liefern t-Test und ANOVA statistisch signifikante Ergebnisse. Da jede Versuchsbedingung einen Umfang von deutlich über 30 Probanden aufweist, kann von einer Normalverteilung ausgegangen werden. Um die Vorraussetzung der Varianzhomogenität zu prüfen, habe ich einen Levene-Test durchgeführt. Bei diesem handelt es sich um einen Signifikanztest, welcher prüft, ob die Varianzen innerhalb von zwei oder mehr Grundgesamtheiten (Gruppen) gleich sind (H0). Daraus ergibt sich die Alternativhypothese (H1), dass mindestens ein Gruppenpaar ungleiche Varianzen besitzt. Befindet sich der p-Wert unterhalb  eines zuvor definierten Signifikanzniveaus sind die Unterschiede in den Varianzen der unterschiedlichen Stichproben signifikant. Folglich kann die Nullhypothese des Tests abgelehnt werden und es kann angenommen werden, dass die Varianzen ungleich sind. Für den Test gehe ich von einem Signifikanzniveau von 5\% aus. 

\subsubsection{Überprüfung mittels t-Test}

\paragraph{Hypothese 1 }
\begin{center}
    \textit{Probanden, die Abzeichen erhalten (EB), beantworten im Mittel eine höhere Anzahl an Fragen als eine Kontrollgruppe ohne Abzeichen (KG).} 
\end{center}


Die Differenz der durchschnittlichen Anzahl beantworteter Fragen von Probanden der Experimentalbedingung 'Abzeichen' ($\bar{x} =4.19;\:SD=3.24$) und Probanden der Kontrollgruppe ohne Spielelemente ($\bar{x} =4.80;\:SD=3.64$) war nicht signifikant ($t (160) = -1.1181;\: p = 0.867$). Die Hypothese kann damit nicht bestätigt werden. Aus diesem Grund wurden zusätzlich zwei abgewandelte Hypothesen getestet, um ein tiefergehendes Verständnis der Daten zu erlangen:

\begin{itemize}
    \item \textbf{1.A:} \textit{Probanden, die Abzeichen erhalten (EB), verbringen im Mittel mehr Zeit mit dem Experiment als eine Kontrollgruppe (KG) ohne Spielelemente.}
    \item \textbf{1.B:} \textit{Probanden, die Abzeichen erhalten (EB), benutzen durchschnittlich mehr Befehle als eine Kontrollgruppe (KG) ohne Spielemente.} 
\end{itemize}

Die Ergebnisse des t-Tests für die drei Hypothesen sind in Tabelle \ref{ttest_hypo_1} dargestellt. Der p-Wert bleibt auch bei beiden Hypothesen deutlich oberhalb des Signifikanzniveaus. Folglich müssen beiden Hypothesen abgelehnt werden. Bei allen drei Hypothesen war die Signifikanz des Levene-Tests oberhalb des üblichen Grenzwertes von .05 (siehe Tabelle \ref{levene_hypo_1}). Damit kann angenommen werden, dass die Varianzen der beiden Stichproben gleich sind und Varianzhomogenität vorliegt. Entsprechend sind die Voraussetzungen für die Durchführung eines t-Tests erfüllt.

% ~*~*~*~*~*~*~*~*~*~*~*~*~*~*~*~ LEVENE - HYPO 1 ~*~*~*~*~*~*~*~*~*~*~*~*~*~*~*~
\begin{table}[htbp]
\centering
\begin{tabular}{ |p{4cm}||p{2.0cm}|p{2.0cm}|p{2.0cm}|p{2.0cm}| }
 \hline
 \multicolumn{5}{|c|}{Test auf Varianzhomogenität (Levene's)} \\
 \hline
 & F & df1 &df2 &p \\
 \hline
  Befehlanzahl      & 0.0439    & 1 &   160 & 0.834\\
  Gesamtspielzeit   & 0.1097    & 1 &   160 & 0.741\\
  Gelöste Aufgaben  & 1.0044    & 1 &   160 & 0.318\\
 \hline
\end{tabular}
\caption{Levene-Test auf Varianzhomogenität für die Experimentalbedingung Abzeichen und die Kontrollgruppe (Hypothese 1). Auf dem 5\%-Niveau signifikante p-Werte sind durch ein * gekennzeichnet}
\label{levene_hypo_1}
\end{table}
% ~*~*~*~*~*~*~*~*~*~*~*~*~*~*~*~ END ~*~*~*~*~*~*~*~*~*~*~*~*~*~*~*~
% ~*~*~*~*~*~*~*~*~*~*~*~*~*~*~*~ TEST - HYPO 1 ~*~*~*~*~*~*~*~*~*~*~*~*~*~*~*~
\begin{table}[htbp]
\centering
\begin{tabular}{ |p{4cm}||p{2.0cm}|p{2.0cm}|p{2.0cm}| }
 \hline
 \multicolumn{4}{|c|}{Zweistichproben-t-Test (Students's t)} \\
 \hline
 & T &df & p \\
 \hline
  Befehlanzahl       & -0.5704  &   160 & 0.715\\
  Gesamtspielzeit    &  0.0196  &   160 & 0.492\\
  Gelöste Aufgaben   & -1.1181  &   160 & 0.867\\
 \hline
\end{tabular}
\caption{Zweistichproben-t-Test für die Experimentalbedingung Abzeichen und Kontrollgruppe (Hypothese 1). Anmerkung: $H_a:\: EB > KG$}
\label{ttest_hypo_1}
\end{table}
% ~*~*~*~*~*~*~*~*~*~*~*~*~*~*~*~ END ~*~*~*~*~*~*~*~*~*~*~*~*~*~*~*~



\paragraph{Hypothese 2 }
\begin{center}
    \textit{Probanden, die eine Fortschrittsanzeige erhalten (EP), beantworten im Mittel eine höhere Anzahl an Fragen als eine Kontrollgruppe ohne Fortschrittsanzeige (KG).} 
\end{center}
Mit $t(150)=0.373$ und $p=0.355$ ist die Differenz der durchschnittlichen Anzahl beantworteter Fragen von Probanden der Experimentalbedingung 'Fortschrittsanzeige' ($\bar{x}  =4.58;\:SD=3.63$) und Probanden der Kontrollgruppe ($\bar{x} =4.80;\:SD=3.64$) nicht signifikant. Damit kann Hypothese 2 nicht bestätigt werden. Die für Hypothese 1 abgewandelten Hypothesen wurden auch für die Versuchsbedingung 'Fortschrittsanzeige' aufgestellt: 

\begin{itemize}
    \item \textbf{2.A:} \textit{Probanden, die eine Fortschrittsanzeige erhalten (EP), verbringen im Mittel mehr Zeit mit dem Experiment als eine Kontrollgruppe ohne Spielelemente (KG).}
    \item \textbf{2.B:} \textit{Probanden, die eine Fortschrittsanzeige erhalten (EP), benutzen durchschnittlich mehr Befehle als eine Kontrollgruppe ohne Spielelemente (KG).} 
\end{itemize}

Die Ergebnisse für die Hypothesen sind in Tabelle \ref{levene_hypo_2} dargestellt. Erneut ist kein Test signifikant und folglich kann keine Hypothese bestätigt werden. Mit $p\in\{.148, .258, .806\}$ ist die Varianzhomogenität für jede Hypothese geben und der t-Test darf verwendet werden.

% ~*~*~*~*~*~*~*~*~*~*~*~*~*~*~*~ LEVENE - HYPO 2 ~*~*~*~*~*~*~*~*~*~*~*~*~*~*~*~
\begin{table}[htbp]
\centering
\begin{tabular}{ |p{4cm}||p{2.0cm}|p{2.0cm}|p{2.0cm}|p{2.0cm}| }
 \hline
 \multicolumn{5}{|c|}{Test auf Varianzhomogenität (Levene's)} \\
 \hline
 & F & df1 &df2 &p \\
 \hline
  Befehlanzahl      & 2.1140     & 1 &   150 & 0.148\\
  Gesamtspielzeit   & 1.2906     & 1 &   150 & 0.258\\
  Gelöste Aufgaben  & 0.0607     & 1 &   150 & 0.806\\
 \hline
\end{tabular}
\caption{Levene-Test auf Varianzhomogenität für die Experimentalbedingung Fortschrittsanzeige (Hypothese 2). Auf dem 5\%-Niveau signifikante p-Werte sind durch ein * gekennzeichnet}
\label{levene_hypo_2}
\end{table}
% ~*~*~*~*~*~*~*~*~*~*~*~*~*~*~*~ END ~*~*~*~*~*~*~*~*~*~*~*~*~*~*~*~
% ~*~*~*~*~*~*~*~*~*~*~*~*~*~*~*~ TEST - HYPO 2 ~*~*~*~*~*~*~*~*~*~*~*~*~*~*~*~
\begin{table}[htbp]
\centering
\begin{tabular}{ |p{4cm}||p{2.0cm}|p{2.0cm}|p{2.0cm}| }
 \hline
 \multicolumn{4}{|c|}{Zweistichproben-t-Test (Students's t)} \\
 \hline
 & T &df & p \\
 \hline
  Befehlanzahl       & 1.099   &   150 & 0.137\\
  Gesamtspielzeit    & 0.822   &   150 & 0.206\\
  Gelöste Aufgaben   & 0.373   &   150 & 0.355\\
 \hline
\end{tabular}
\caption{Zweistichproben-t-Test für die Experimentalbedingung Abzeichen und Kontrollgruppe (Hypothese 1). Anmerkung: $H_a:\: EP > KG$}
\label{ttest_hypo_2}
\end{table}
% ~*~*~*~*~*~*~*~*~*~*~*~*~*~*~*~ END ~*~*~*~*~*~*~*~*~*~*~*~*~*~*~*~

% ~*~*~*~*~*~*~*~*~*~*~*~*~*~*~*~ ANOVA ~*~*~*~*~*~*~*~*~*~*~*~*~*~*~*~
\subsubsection{Überprüfung mittels ANOVA Varianzanalyse }
Bei der ANOVA (engl. \textit{Analysis of Variance}) handelt es sich um eine Varianzanalyse, die die Mittelwerte von mehr als 2 Gruppen vergleichbar macht. Dabei handelt es sich um eine Erweiterung des t-Tests, welcher auf den Vergleich von maximal 2 Stichproben beschenkt ist. Dies ermöglicht mir den Vergleich aller drei Versuchsbedingungen. Nachfolgend werden die zwei abhängigen Variablen \textbf{Gelöste Aufgaben} und \textbf{Gesamtspielzeit} jeweils durch eine einfaktorielle Varianzanalyse untersucht.

% Varianzhomogenität prüfen 
Zunächst gilt es erneut die Voraussetzung der Varianzhomogenität mittels Levene-Test zu prüfen. Wie in Tabelle \ref{levene_amova} zu sehen, ist die Signifikanz deutlich oberhalb der Signifianzniveaus ($p\in \{.606, .288\}$) und bestätigt die Varianzgleichheit.

% ~*~*~*~*~*~*~*~*~*~*~*~*~*~*~*~ LEVENE - ANOVA ~*~*~*~*~*~*~*~*~*~*~*~*~*~*~*~
\begin{table}[htbp]
\centering
\begin{tabular}{ |p{4cm}||p{2.0cm}|p{2.0cm}|p{2.0cm}|p{2.0cm}| }
 \hline
 \multicolumn{5}{|c|}{Test auf Varianzhomogenität (Levene's)} \\
 \hline
 & F & df1 &df2 &p \\
 \hline
  Gelöste Aufgaben      & 0.503     & 2 &   237 & 0.606\\
  Gesamtspielzeit       & 1.252     & 2 &   237 & 0.288\\
 \hline
\end{tabular}
\caption{Levene-Test auf Varianzhomogenität für EB, KG, EP. Auf dem 5\%-Niveau signifikante p-Werte sind durch ein * gekennzeichnet}
\label{levene_amova}
\end{table}
% ~*~*~*~*~*~*~*~*~*~*~*~*~*~*~*~ END ~*~*~*~*~*~*~*~*~*~*~*~*~*~*~*~

Mit $F(2,154) = 0.652$ und einer Signifikanz von $p=0.523$ gibt es keinen statisch signifikanten Unterschied der Gruppenmittelwerte hinsichtlich der Anzahl gelöster Aufgaben. Ein ähnliches Ergebnis liefert die ANOVA für die Gesamtspielzeit ($F(2, 156)=0.512$; $p=0.6$). Damit kann gezeigt werden, dass kein signifikanter Unterschied im Zusammenhang der Anzahl gelöster Aufgaben sowie der Gesamtspielzeit zwischen den einzelnen Versuchsbedingungen besteht.

Obwohl die Werte der einzelnen Gruppen sich nicht signifikant voneinander unterscheiden, habe ich einen Post-hoc Test durchgeführt, um tiefergehende Einblicke in die Daten zu erlangen. Ich verwendete dazu einen Tukey Test, da gleiche Varianzen vorliegen. Die Ergebnisse sind in Tabelle \ref{tukey_subs} (Befehle) und Tabelle \ref{tukey_time} (Gesamtspielzeit) dargestellt. Erwartungsgemäß findet sich kein signifikanter p-Wert. Dies ist in Anbetracht der nicht festgestellten Signifikanz in der ANOVA jedoch wenig verwunderlich. Bemerkenswert ist der maximale p-Wert von $p\approx1$ für den Vergleich der Experimentalbedingung 'Abzeichen' und der Kontrollgruppe hinsichtlich der Gesamtspielzeit. Damit verbringen die beiden Versuchsbedingungen nahezu identisch viel Zeit mit dem Experiment.

% ~*~*~*~*~*~*~*~*~*~*~*~*~*~*~*~ POSTHOC ~*~*~*~*~*~*~*~*~*~*~*~*~*~*~*~


\begin{table}[htbp]
\centering
\begin{tabular}{ p{2cm} p{4cm} p{2.0cm} p{2.0cm} p{2.0cm} }
 \hline
 \multicolumn{5}{c}{Tukey Post-Hoc Test (Gelöste Aufgaben)} \\
 \hline
 & & EB & KG & EP \\
 \hline
  EB    &   Mittlere Abweichung     & -     & -0.604        & -0.384          \\
        &   t                       & -     & -1.10         & -0.706          \\
        &   df                      & -     & 237           & 237              \\
        &   p                       & -     & 0.518         & 0.760     \\
  & & & &\\
  KG    &   Mittlere Abweichung     &       & -    & 0.220         \\
        &   t                       &       & -    & 0.389         \\
        &   df                      &       & -    & 237            \\
        &   p                       &       & -    & 0.920          \\
  & & & &\\
  EP    &   Mittlere Abweichung     &       &       & - \\
        &   t                       &       &       & - \\
        &   df                      &       &       & - \\
        &   p                       &       &       & - \\
  
 \hline
\end{tabular}
\caption{Tukey Post-Hoc Test für die Gruppen Experimentalbedingung Abzeichen (EB), Experimentalbedingung Fortschrittsbalken (EP) und Kontrollgruppe (KG). Abhängige Variable: \textbf{Anzahl gelöster Aufgaben}}
\label{tukey_subs}
\end{table}


\begin{table}[htbp]
\centering
\begin{tabular}{ p{2cm} p{4cm} p{2.0cm} p{2.0cm} p{2.0cm} }
 \hline
 \multicolumn{5}{c}{Tukey Post-Hoc Test (Gesamtspielzeit)} \\
 \hline
 & & EB & KG & EP \\
 \hline
  EB    &   Mittlere Abweichung     & -     & 1.89          & 76.9  \\
        &   t                       & -     & 0.0206        & 0.853         \\
        &   df                      & -     & 237           & 237          \\
        &   p                       & -     & \fbox{1.000}  & 0.670 \\
  & & & &\\
  KG    &   Mittlere Abweichung     &       & -         & 75.0   \\
        &   t                       &       & -         & 0.797  \\
        &   df                      &       & -         & 237   \\
        &   p                       &       & -         & 0.705 \\
  & & & &\\
  EP    &   Mittlere Abweichung     &       &           & -     \\
        &   t                       &       &           & -     \\
        &   df                      &       &           & -     \\
        &   p                       &       &           & -     \\
  
 \hline
\end{tabular}
\caption{Tukey Post-Hoc Test für die Gruppen Experimentalbedingung Abzeichen (EB), Experimentalbedingung Fortschrittsbalken (EP) und Kontrollgruppe (KG). Abhängige Variable: \textbf{Gesamtspielzeit}}
\label{tukey_time}
\end{table}

% ~*~*~*~*~*~*~*~*~*~*~*~*~*~*~*~ POSTHOC - END ~*~*~*~*~*~*~*~*~*~*~*~*~*~*~*~

\subsection{Zwischenfazit t-Test und ANOVA}
Die geschilderten Ergebnisse entsprechen nicht meinen ursprünglichen Erwartungen. Tatsächlich kann weder durch einen t-Test noch durch eine ANOVA ein signifikanter Effekt auf die Anzahl gelöster Aufgaben, die Menge der abgesetzten Befehle oder die Gesamtspielzeit nachgewiesen werden. Als Konsequenz kann keine meiner ursprünglich aufgestellten Hypothesen bestätigt werden. Die Spielelemente Abzeichen und Fortschrittsbalken zeigen damit in meinem Experiment keinerlei statisch signifikanten, positiven Effekt auf das Durchhaltevermögen (gemessen durch die Gesamtspielzeit) der Teilnehmer. Auch lässt sich keine erhöhte Aktivität der Probanden feststellen, da sich die Versuchsbedingungen hinsichtlich der gelösten Aufgaben und abgesetzten Befehle nicht signifikant unterscheiden. Stattdessen kann im Fall der Fortschrittsanzeige ein tendenziell negativer Einfluss beobachtet werden. Dieser zeigt sich insbesondere in der Gesamtspielzeit, welche im Mittel deutlich geringer ist als in den Alternativbedingungen. 

% ~*~*~*~*~*~*~*~*~*~*~*~*~*~*~*~ Weiterführende Analyse ~*~*~*~*~*~*~*~*~*~*~*~*~*~*~*~
\subsection{Weiterführende Analyse}
Um die Daten besser zu verstehen und um die Ergenisse erklärbar zu machen, habe ich die Daten ausführlicher untersucht.

% Alte Teilnehmer mit Teilnehmern von Hackernews und MUC vergleichen
Der Verlauf des Experiments kann zeitlich in zwei Abschnitte unterteilt werden. Bis zum 06.09.2020 kamen insgesamt 137 Teilnehmer zusammen. Da diese Zahl für eine statistisch haltbare Auswertung zu klein erschien, habe ich am selbigen Tag beschlossen, das Experiment auf der Internetplattform Hackernews\footnote{Link: https://news.ycombinator.com/} zu teilen.Bei dieser Seite handelt es sich um eine Social-News-Website, die vornehmlich Informatik-affine Personen anspricht. Zeitgleich wurde das Experiment von 	Dr. rer. pol., Dipl.-Psych. Athanasios Mazarakis auf der Mensch und Computer 2020 (MuC) geteilt. Diese Kombination führte zu einem sprunghaften Anstieg der Teilnehmerzahl. So kamen innerhalb von vier Tagen 141 neue Teilnehmer hinzu. Da anzunehmen ist, dass es sich dabei um sehr versierte Teilnehmer handelt, die bereits Erfahrung mit der Kommandozeile haben, habe ich beschlossen,
das Experiment zeitlich in zwei disjunkte Abschnitte zu teilen. Dazu teilte ich den Datensatz in einen Abschnitt vor dem 06.09.2020 und einen Abschnitt ab dem 06.09.2020. Der erste Abschnitt besteht damit aus Teilnehmern, die bis zum 05.09.2020 um 23:59:59 an dem Experiment teilgenommen haben, und wird nachfolgend als \textbf{Gruppe 1} bezeichnet. Entsprechend setzt sich die andere Hälfte aus Teilnehmern zusammen, die nach dem 06.09.2020 00:00 Uhr teilnahmen, und wird als \textbf{Gruppe 2} bezeichnet. Teilnehmer ohne Interaktion (keine abgeschickten Befehle) werden erneut nicht berücksichtigt.

\subsubsection{Weiterführende Analyse - Zusammensetzung der Gruppen}
Zunächst habe ich demographischen Daten der beiden Gruppen verglichen. Die vollständige Gegenüberstellung ist in Tabelle \ref{demo_g12} dargestellt. Beide Stichproben setzen sich zu einem Großteil aus männlichen Probanden zusammen (> 80 \%). Der Anteil von weiblicher und diverser Teilnehmer ist zwischen den Gruppen vertauscht. So sind in Gruppe 1 12.5 \% weiblich und 5.8\% divers, während in Gruppe 2 5.7 \% weiblich und 13.8\% divers sind. Gruppe 1 verfügt durchschnittlich über weniger gute Englisch Kenntnisse als Gruppe 2 (34.7 \% sehr gut vs 52.0 \% sehr gut). Gleichzeitig ist der Altersdurchschnitt in Gruppe 2 höher als in Gruppe 1. So ist der Großteil der Probanden aus Gruppe 2 zwischen 20 und 39 Jahre alt (39.0 \%). Demgegenüber gaben die Teilnehmer der ersten Gruppe überwiegend ein Alter zwischen 18 und 29 Jahren an (48.8\%). Die Probanden aus Gruppe 2 nutzen die Kommandozeile zudem deutlich häufiger. Beispielsweise nutzen 65.9 \% der Probanden aus Gruppe 2 das Terminal täglich, während dies in Gruppe 1 lediglich 28.9 \% sind. Damit ergeben sich die folgenden Profile für den jeweiligen Versuchsabschnitt:

Gruppe 1 setzt sich aus jungen, überwiegend männlichen Teilnehmern zusammen, die über gute Englischkenntnisse verfügen und bereits erste Erfahrungen mit der Kommandozeile haben. 

Gruppe 2 setzt sich aus überwiegend männlichen Teilnehmern mittleren Alters zusammen, die über gute bis sehr gute Englischkenntnisse verfügen und die die Kommandozeile häufig nutzen. 

% ~*~*~*~*~*~*~*~*~*~*~*~*~*~*~*~ Vergleich Demographie ~*~*~*~*~*~*~*~*~*~*~*~*~*~*~*~
\begin{table}[htbp]
\centering
\begin{tabular}{ |p{6cm}||p{3cm}|p{3cm}| }
 \hline
 \multicolumn{3}{|c|}{Vergleich Demographie zwischen Gruppe 1 und Gruppe 2} \\
 \hline
  Merkmal & Gruppe 1 & Gruppe 2 \\
  Umfang & 121  & 123 \\
  \hline
  \multicolumn{3}{|c|}{Geschlecht} \\
  \hline
  Weiblich                      & 12.5 \%       & 5.7  \%       \\
  Männlich                      & 81.7 \%       & 80.5 \%       \\
  Divers                        & 5.8  \%       & 13.8 \%       \\
  \hline
  \multicolumn{3}{|c|}{Englischkenntnisse} \\
  \hline
  Sehr gut                      & 34.7 \%       & 52.0 \%      \\
  Gut                           & 48.8 \%       & 35.0 \%       \\
  Nicht so gut                  & 11.6 \%       & 6.5 \%        \\
  Nicht vorhanden               & 5.0  \%       & 6.5 \%        \\
  \hline
  \multicolumn{3}{|c|}{Alter} \\
  \hline
  <18                           & 6.6  \%       & 3.3  \%       \\
  18-29                         & 48.8 \%       & 27.6 \%       \\
  29-39                         & 32.2 \%       & 39.0 \%       \\
  39-59                         & 9.1  \%       & 22.0 \%       \\
  59+                           & 3.3  \%       & 8.1 \%        \\
  \hline
  \multicolumn{3}{|c|}{Häufigkeit der Nutzung der Kommandozeile} \\
  \hline
  Täglich (1x pro Tag)          & 28.9 \%       & 65.9  \%      \\
  Gelegentlich (1x pro Woche)   & 34.7 \%       & 13.8  \%      \\
  Selten (1x pro Monat)         & 11.6 \%       & 7.3  \%       \\
  Sehr selten (1x pro Jahr)     & 17.4 \%       & 4.9  \%       \\
  Nie                           & 7.4  \%       & 8.1  \%       \\
  \hline
\end{tabular}
\caption{Vergleich Demographie zwischen Gruppe 1 und Gruppe 2. Die jeweiligen Merkmale sind als prozentualer Anteil des Gesamtumfangs der jeweiligen Gruppe angegeben. }
\label{demo_g12}
\end{table}
% ~*~*~*~*~*~*~*~*~*~*~*~*~*~*~*~ END ~*~*~*~*~*~*~*~*~*~*~*~*~*~*~*~


Im folgenden Abschnitt wurden die beiden Gruppen gesondert untersucht und analysiert. Wie zuvor wurden extreme Ausreißer sowie Teilnehmer ohne Interaktion (keinen Befehl gesendet) aus den Datensätzen eliminiert. In der Stichprobe, die bis zum 06.09.2020 zu Stande gekommen ist, liegt der Grenzwert der Spielzeit bei 5089 Sekunden ($\bar{x}\pm 2\sigma = 727 + 2*2181 = 5089 s$ ). Teilnehmer, deren Gesamtspielzeit diesen Grenzwert übersteigt, wurden nachfolgend nicht berücksichtigt. Der so entstandene Datensatz hat einen Umfang von 117 Personen. Der entsprechende Schwellwert lag im Fall der zweiten Gruppe bei 1360 Sekunden ($\bar{x}\pm 2\sigma = 330 + 2*515 = 1360 s$ ). Dies führte zu einer Stichprobengröße von 116 Personen. Ziel war es, meine aufgestellten und in Abschnitt \ref{hypo} angepassten Hypothesen für die jeweilige Gruppe zu prüfen. Es wurden für jede Gruppe folgende Hypothesen geprüft:

\paragraph{Hypothese 1}
\begin{itemize}
    \item \textbf{1.A:} \textit{Probanden, die Abzeichen erhalten (EB), verbringen im Mittel mehr Zeit mit dem Experiment als eine Kontrollgruppe (KG) ohne Spielelemente.}
    \item \textbf{1.B:} \textit{Probanden, die Abzeichen erhalten (EB), senden im Mittel mehr Befehle ab, als eine Kontrollgruppe (KG).}
    \item \textbf{1.C:} \textit{Probanden der Experimentalbedingung 'Abzeichen' (EB) lösen mehr Aufgaben als Personen der Kontrollgruppe.} 
\end{itemize}

\paragraph{Gruppe 1}
Die Experimentalbedingung Abzeichen unterschied sich mit $ t(71)=-0.325 $ und $p=0.746$ nicht signifikant von der Kontrollgruppe hinsichtlich der absoluten Anzahl genutzter Befehle. Auch die durchschnittliche Spielzeit zwischen Personen, die ein Abzeichen erhielten, und der Kontrollgruppe unterschied sich nicht signifikant ($t(71)=-0.396; p=0.696$). Selbiges gilt für die Aufgabenzahl. Das Spielelement Abzeichen führte zu keinem signifikanten Unterschied der beantworteten Fragen ( $ t(71)=-1.510; p=0.135 $ ). Damit kann keine Hypothese bestätigt werden und es kann innerhalb der ersten Stichprobe nicht von einem motivierenden Effekt des Abzeichens ausgegangen werden.

\paragraph{Gruppe 2}
Es konnte bei der zweiten Gruppe kein motivierender Einfluss durch ein Abzeichen beobachtet werden. Weder die Anzahl der genutzten Befehle ( $ t(82)=-0.4926; p=0.624 $ ) noch die Gesamtspielzeit ($t(82)=-0.0767; p=0.939$) oder die gelösten Aufgaben ( $t(82)=-0.3709; p=0.712$ ) unterschied sich signifikant. Daher kann erneut keine Hypothese bestätigt werden und es muss davon ausgegangen werden, dass Teilnehmer aus der zweiten Stichprobe nicht durch Abzeichen motiviert wurden. 


\paragraph{Hypothese 2}
\begin{itemize}
    \item \textbf{2.A:} \textit{Probanden, die eine Fortschrittsanzeige erhalten (EP), verbringen im Mittel mehr Zeit mit dem Experiment als eine Kontrollgruppe (KG) ohne Spielelemente.}
    \item \textbf{2.B:} \textit{Probanden, die eine Fortschrittsanzeige erhalten (EP), senden im Mittel mehr Befehle ab, als eine Kontrollgruppe (KG).}
    \item \textbf{2.C:} \textit{Probanden der Experimentalbedingung 'Fortschrittsanzeige' (EP) lösen mehr Aufgaben als Personen der Kontrollgruppe.} 
\end{itemize}

\paragraph{Gruppe 1}
Bei einem Vergleich der Experimentalbedingung Fortschrittsanzeige und der Kontrollgruppe lag Varianzheterogenität vor (Levene-Tests: $p\in\{.004, .030, .003\}$). Daher wurde ein Welch-Test durchgeführt. Die durchschnittliche Spielzeit von der Experimentalbedingung ($\bar{x}=248 s $) unterschied sich messbar von der Kontrollgruppe ($\bar{x}=481 s$). Jedoch war der Unterschied statisch nicht signifikant ($t(50)=1.45; p=.154$). Auch die Differenz genutzter Befehle von Experimentalbedingung und Kontrollgruppe war nicht signifikant($t(40.5)=1.9; p=.065$). Die durchschnittliche Anzahl gelöster Aufgaben war in der Stichprobe mit Fortschrittsanzeige ($\bar{x}=3$) niedriger als in der Kontrollgruppe ($\bar{x}= 4.73$). Der Differenz war signifikant: $t(47.2)=2, p=.05$. Dies führt dazu, dass keine Hypothese bestätigt werden kann. Stattdessen konnte gezeigt werden, dass Probanden der Experimentalbedingung 'Fortschrittsanzeige' (EP) weniger Aufgaben lösten als die Kontrollgruppe. 


\paragraph{Gruppe 2}
Teilnehmer, die eine animierte Fortschrittsanzeige angezeigt bekamen, investierten mit $\bar{x}_{EP}=274 s$ mehr Zeit in das Experiment
 und nutzen dabei mehr Befehle ($\bar{x}_{EP}=23.25$) als die Kontrollgruppe. Letztere investierte vergleichsweise wenig Zeit ($ \bar{x}_{KG} = 202s $ ) in den Versuch und nutze ebenfalls weniger Befehle ($ \bar{x}_{KG} = 18.07 $). Ein t-Test ergab, dass die Differenz der gelösten Aufgaben signifikant ist ($t(72)=-2.16, p=.034$) und bestätigt meine Hypothese \textbf{2.C}.  Probanden der Experimentalbedingung 'Fortschrittsanzeige' lösten durchschnittlich mehr Aufgaben als die Kontrollgruppe. Mit $ p=.217 $ und $p=.185$ sind die Unterschiede der Spielzeit und Befehlsanzahl messbar jedoch nicht signifikant.
 

Zusammengefasst lässt sich sagen, dass sich die Ergebnisse in der zweiten Stichprobe im wesentlichen mit einen Erwartungen decken. Die Ergebnisse der ersten Gruppe hingegen weichen deutlich von meinen ursprünglichen Hypothesen ab. Die Unterschiede zwischen den zwei Stichproben sind dabei stellenweise extrem. So zeigt eine Fortschrittsanzeige in beiden Stichproben einen deutlichen Effekt auf das Durchhaltevermögen der Teilnehmer: In Gruppe 1 muss von einem demotivierendem Effekt der Fortschrittsanzeige ausgegangen werden, während in Gruppe 2 ein motivierender Effekt zu beobachten war. In beiden Fällen handelt es sich dabei um eine statisch signifikante Differenz. Generell weichen die Daten der Gruppen deutlich voneinander ab. Eine entsprechende Gegenüberstellung findet sich in Tabelle \ref{final}. Besonders auffällig sind die deutlich höheren maximalen Spielzeiten in Gruppe 1 gegenüber Gruppe 2. Personen der ersten Stichprobe investierten in jeder Versuchsbedingung im Maximum $\sim3$ Mal mehr Zeit für das Experiment, obwohl extreme Ausreißer nicht berücksichtigt wurden. Als Konsequenz sind die entsprechenden Standardabweichungen deutlich größer als die der zweiten Gruppe.


\begin{table}[htbp]
\centering
\begin{tabular}{ p{3.5cm} | p{0.75cm} | p{1cm} p{1cm}  p{1cm} || p{1cm} p{1cm} p{1cm}}
 \hline
 \multicolumn{8}{c}{Vergleich Gruppe 1 und Gruppe 2} \\
 \hline
 & & \multicolumn{3}{c||}{Gruppe 1} & \multicolumn{3}{c}{Gruppe 2}\\
 & & EB & KG & EP & EB & KG & EP\\
 \hline
  Gesamtspielzeit   & $\bar{x}$     & 417   & 482   & 248   & 199  & 202  & 275   \\
                    & $\tilde{x}$   & 157   & 181   & 55.5  & 105  & 140  & 171   \\
                    & $\sigma$      & 629   & 753   & 557   & 252  & 204  & 260   \\
                    & min           & 0     & 0     & 0     & 0    & 0    & 0     \\
                    & max           & 2847  & 3300  & 3420  & 1108 & 933  & 1012  \\
 \hline
  Gelöste Aufgaben  & $\bar{x}$     & 3.49  & 4.73  & 3     & 4.48 & 4.74   & 6.5     \\
                    & $\tilde{x}$   & 3     & 3.5   & 2.5   & 4    & 4      & 5.5     \\
                    & $\sigma$      & 2.01  & 4.14  & 2.83  & 3.15 & 3.32   & 3.65     \\
                    & min           & 0     & 0     & 0     & 0    & 0      & 1     \\
                    & max           & 13    & 13    & 13    & 13   & 13     & 13     \\
  \hline
  Genutzte Befehle  & $\bar{x}$     & 17   & 18.8   & 10.7  & 16.3   & 18.1   & 23.3     \\
                    & $\tilde{x}$   & 8    & 10     & 5.5   & 9      & 13.5   & 16.5     \\
                    & $\sigma$      & 24.5 & 21.4   & 11.5  & 15.6   & 17.6   & 17.9     \\
                    & min           & 1    & 1      & 1     & 2      & 1      & 1     \\
                    & max           & 137  & 78     & 48    & 73     & 71     & 62     \\

  
 \hline
\end{tabular}
\caption{Gegenüberstellung der Gesamtspielzeit, Anzahl genutzter Befehle und Aufgabenanzahl zwischen Gruppe 1 und Gruppe 2. Extreme Ausreißer wurden nicht berücksichtigt.}
\label{final}
\end{table}



\subsubsection{Feedback der Teilnehmer}
Wurden alle Fragen korrekt beantwortet, wurden die Teilnehmer gebeten, Feedback hinsichtlich der empfundenen Motivation zu geben. Von 18 Teilnehmern, die das Experiment beendeten, gaben 17 eine entsprechende Rückmeldung ab. Wie in Tabelle \ref{feedback} zu sehen, gab über die Hälfte der Teilnehmer an, tendenziell motiviert gewesen zu sein. Lediglich drei Teilnehmer gaben an, sich eher nicht motiviert gefühlt zu haben. Aufgrund des geringen Umfangs der Stichprobe kann keine statistisch belastbare Aussage auf Basis des Feedbacks getroffen werden. Jedoch kann angenommen werden, dass sich Teilnehmer, die das Experiment beendeten, tendenziell motiviert fühlten.

\begin{table}[htbp]
\centering
\begin{tabular}{ p{4cm} |  p{2cm}}
 Feedback & Anzahl \\
 \hline
 trifft zu & 6 \\
 trifft eher zu & 5 \\
 teils-teils & 3 \\
 trifft eher nicht zu & 3 \\
 trifft nicht zu & 0 \\
 \hline
\end{tabular}
\caption{Tabellarische Darstellung des finalen Feedbacks der Probanden auf die Frage: "Ich fühlte mich während des Experiments motiviert."}
\label{feedback}
\end{table}