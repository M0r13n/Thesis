\usepackage[a4paper,left=35mm,width=145mm,top=20mm]{geometry}
%Automatische Verzeichnisse ins Inhaltsverzeichnis
\usepackage{tocbibind}

\usepackage[utf8]{inputenc} % Kodierung
\usepackage[ngerman]{babel} % deutsche Sprache
\usepackage{acronym}
%Verwendete Schrift z.B. lmodern
%\usepackage{lmodern}
%\usepackage{libertine}
\usepackage{mathptmx} % Für Times New Roman Font
%\usepackage{mathpazo,eulervm}
%\usepackage{tgpagella,eulervm}
%\usepackage[charter]{mathdesign}
%\usepackage[osf,sc]{mathpazo}
%\usepackage[sc]{mathpazo}
%\usepackage{
%lmodern, 
%hfoldsty 
% charter
%}

%Sprachenbezogene Pakete
%\usepackage[english]{babel} % englische Sprache
%\usepackage[babel, german=guillemets]{csquotes}
\usepackage[babel, german=quotes]{csquotes} %Deutsche Anführungszeichen

\usepackage[T1]{fontenc}
\usepackage{eurosym}
% Das € Zeichen kann nun wirklich mit € angezeigt werden
\DeclareUnicodeCharacter{20AC}{\euro}
%Die Pakete für den Index
\usepackage{makeidx}
\makeindex
%Diverse Standardpakete
\usepackage[
		pdftex,
		pdfpagelabels=true,
	 	citecolor=black,
       	filecolor=blue,
      	urlcolor=black,
       	bookmarks=true,
       	bookmarksopen=true,
       	bookmarksopenlevel=3,
       	plainpages=false,
       	pdfpagelabels=true,
       	pdfborder={0 0 0},
       	breaklinks=true,]{hyperref}
\hypersetup{colorlinks=true,
			breaklinks=true,
			urlcolor=black,
			linkcolor=black,
			menucolor=black,
			% Diese Angaben kommen in die PDF Eigenschaften
			pdftitle    = {Bachelorthesis - Leon Morten Richter},
			pdfsubject  = {Bachelorthesis},
			pdfauthor   = {Leon Morten Richter},
			pdfkeywords = {Gamification, Serious Games, Commandline, bash},
			pdfcreator  = {TeXStudio, MacTex \& Me},
			pdfproducer = {LaTeX with Hyperref}}
\usepackage[hypcap]{caption}

%Zitierweise im APA-Stil
\usepackage{cite} % Zitate - MUSS VOR apacite eingebunden werden
\usepackage{apacite} % Apacite Style einbinden

%%Zeilenabstand
\usepackage{setspace}
%\setstretch{1,2381}
\onehalfspacing

% Das braucht TODO, damit es keine Warnung gibt 
\setlength {\marginparwidth }{2cm} 

% Pakete für Grafik, Tabellen u.a.
\usepackage{verbatim} %für multiline comments
%\usepackage[nonumberlist, acronym, toc, section]{glossaries}
\usepackage{textcomp}
\usepackage{graphicx} % support the \includegraphics command and options
\usepackage{booktabs}
\usepackage{array} % for better arrays (eg matrices) in maths
\usepackage{subfig} % make it possible to include more than one captioned figure/table in a single float
\usepackage{multicol}
\usepackage{multirow}
\usepackage{ifthen}
\usepackage{float}
\usepackage{csquotes}
\usepackage{tabularx}
\newcolumntype{C}[1]{>{\centering\arraybackslash}m{#1}} 
\newcolumntype{L}[1]{>{\raggedright\arraybackslash}m{#1}}
\newcolumntype{R}[1]{>{\raggedleft\arraybackslash}m{#1}}
\usepackage{paralist}
\usepackage{siunitx}
\usepackage{svg}
\sisetup{
add-decimal-zero = false,
add-integer-zero = false,
}

%Befehl für die Linien auf der Titelseite
\newcommand{\HRule}{\rule{\linewidth}{0.2mm}}
%Absatz einrücken
%Legt die Einrücktiefe der ersten Zeile nach Überschrift fest
\usepackage{indentfirst}

% Horiz. Linien in Tabellen verbessern
\newcommand{\forloop}[5][1]{%
\setcounter{#2}{#3}%
\ifthenelse{#4}{#5\addtocounter{#2}{#1}%
\forloop[#1]{#2}{\value{#2}}{#4}{#5}}%
{}}
\newcounter{crcounter}
\newcommand{\compensaterule}[1]{%
\forloop{crcounter}{1}{\value{crcounter} < #1}%
{\vspace*{-\aboverulesep}\vspace*{-\belowrulesep}}}
\newcommand{\multirowbt}[3]{\multirow{#1}{#2}%
{\compensaterule{#1}#3}}


% custom commands
\newcommand{\qq}[1]{{ \glqq #1\grqq{}}}

%Captions left
\usepackage[font=footnotesize,labelfont=bf,singlelinecheck=false,format=plain,,justification=justified,indention=0cm]{caption}

% Misc
\usepackage[colorinlistoftodos]{todonotes} % für die guten TODOS



