%-------------------------------------------------------------------------------
% Theorie
%-------------------------------------------------------------------------------
\section{Theoretischer und empirischer Hintergrund}

Unter dem Begriff Gamification wird der Einsatz von Spielmechaniken und Spieldesignelementen in einem spielfremden Kontext verstanden \cite{deterding_game_2011}. Typische Beispiele für solche Spielelemente sind Abzeichen, Fortschrittsanzeigen, Levels oder Ranglisten \cite{koch2013gamification}. Praktische Beispiele sind die Lernplattform Duolingo, die dem Erlernen von Sprachen dient, oder der Social-News-Aggregator Reddit. Im Fall von Duolingo schaltet der Nutzer neue Lektionen als Belohnung für den Abschluss bisheriger Lektionen frei. Reddit hingegen integriert ein Punktesystem. Nutzer können etwa durch das regelmäßige Posten qualitativ hochwertiger Beiträge \qq{Karmapunkte} sammeln. Zudem ist es möglich, Beiträge mit einem \qq{Down-Vote} oder einem \qq{Up-Vote} zu bewerten.
Ziel ist es, dass sich Nutzer freiwillig mehr agieren, um ihr Ansehen in der Community zu steigern.

Die Wirksamkeit von Gamification konnte dabei bereits für unterschiedlichste Einsatzgebiete empirisch nachgewiesen werden \cite{koivisto_rise_2019} und hat sich in der Forschung zur Mensch-Computer-Interaktion etabliert \cite{huotari_defining_2012}. 
Bereiche, in denen die Wirksamkeit von Gamification gezeigt werden konnte, sind Forschung/ Open Science \cite{brauer_erhohung_2019,kidwell_badges_2016}, Gesundheit und Fitness \cite{johnson_gamification_2016}, Marketing \cite{huotari_defining_2012} oder Produktion und Logistik \cite{warmelink_gamification_2018}.

%-------------------------------------------------------------------------------
% Abzeichen
%-------------------------------------------------------------------------------
\subsection{Spielelement Abzeichen}
Im Kontext der Gamification bezeichnen Abzeichen digitale, visuelle Artefakte, die dem Nutzer für die Erfüllung definierter Aufgaben verliehen werden \cite{antin_badges_2011}. \citeauthor{antin_badges_2011} unterteilen Abzeichen anhand ihrer sozialpsychologische Funktion in fünf Kategorien: Zielsetzung (Goal setting), Anweisung (Instruction), Reputation (Reputation),
Status/Bestätigung (Status / Affirmation) und Gruppenidentifizierung (Group Identification). 

\paragraph{Zielsetzung}
Abzeichen fordern den Nutzer dazu heraus, ein definiertes Ziel zu erreichen. Individuen streben selbst dann nach dem Erreichen bestimmter Ziele oder Meilensteine, wenn sie selber keinerlei physische Gegenleistung erfahren. Die in einem Abzeichen dargestellten Ziele sind allerdings nicht immer explizit, etwa weil die notwendigen Aktivitäten subjektiv oder unpräzise definiert sind. Aus diesen Grund ist es wichtig, den Nutzer durch regelmäßiges Feedback auf seinen aktuellen Fortschritt hinzuweisen. 

\paragraph{Anweisung}
Abzeichen können richtungweisend wirken. So geben Abzeichen einen Hinweis darauf, welche Arten von Aktivität innerhalb eines Systems möglich sind. Diese Funktion ist nützlich, um neue Benutzer in eine bestimmte Richtung zu lenken, aber auch um bestehenden Nutzern neue Wege zu offenbaren. Dabei ist es nicht notwendig, dass Nutzer die Abzeichen tatsächlich erreichen. Alleine das bloße Betrachten der verfügbaren Abzeichen gibt dem Nutzer Aufschluss über wertgeschätzte Aktivitäten.

\paragraph{Reputation}
Abzeichen blinden die Grundlage für die Reputationsbewertungen einzelner Nutzer. Durch die Kapselung von Interessen, Fachwissen und vergangener Interaktionen helfen Abzeichen bei der Beurteilung der Reputation. So geben einsehbare Abzeichen einen Einblick in nutzerspezifische Interessen, bieten einen Überblick über Fähigkeiten und Fachwissen und dienen der Einschätzung der Vertrauenswürdigkeit und der Zuverlässigkeit eines Nutzers. Zugleich zeigen geben sie einen Hinweis auf bereits erbrachte Leistungen.

\paragraph{Status/Bestätigung}
Abzeichen wirken als Statussymbol. Sie verdeutlichen die eigene Leistung und bisherige Errungenschaften ohne explizite Prahlerei.
Die Macht der Statusbelohnungen ergibt sich insbesondere aus der Erwartung, dass andere positiv auf jemanden reagieren, der durch außerordentliche Aktivitäten entsprechende Abzeichen errungen hat. Abzeichen sind auch eine persönliche Bestätigung, da sie an vergangene Errungenschaften erinnern. Sie markieren wichtige Meilensteine und sind ein Beweis für vergangene Erfolge. Sie wirken damit auf eine ähnliche Art und Weise wie klassische Trophäen oder Medaillen.

\paragraph{Gruppenidentifizierung}
Abzeichen kommunizieren eine Reihe von gemeinsamen Aktivitäten, die eine Gruppe von Benutzern anhand gemeinsamer Erfahrungen zusammenschweißen. Das Erlangen von Abzeichen kann ein Gefühl der Solidarität vermitteln und die positive Gruppenidentifikation durch die Wahrnehmung der Ähnlichkeit zwischen einem Individuum und der Gruppe erhöhen.\\

Die Wirksamkeit von Abzeichen ist dabei mehrfach empirisch nachgewiesen. Wie \citeauthor{hamari_badges_2017} zeigen konnte, führt die Einführung von Abzeichen zu einer signifikanten Steigerung der Nutzerinteraktion. Einen ähnlich positiver Effekt auf das Lernverhalten von Studenten konnte durch \citeauthor{hamzah_influence_2015} nachgewiesen werden.


%-------------------------------------------------------------------------------
% Fortschrittsbalken
%-------------------------------------------------------------------------------
\subsection{Spielelement Fortschrittsbalken}
Das Spielelement Fortschrittsbalken ist ebenfalls ein etabliertes visuelles Spielelement.
Dabei handelt es sich um ein visuelles Anzeigeelement, das den aktuellen Fortschritt eines Auftrags in Form eines Balkens repräsentiert. Ein klassisches Beispiel ist etwa die Darstellung des Fortschritts einer Installation oder eines Ladevorgangs. Im wesentlichen kann man Fortschrittsanzeigen in zwei Kategorien einteilen. Zum einen gibt es die bestimmte Anzeige, die den absoluten Fortschritt eines Vorgangs anzeigt. Die Anzeige gibt somit Aufschluss auf die verbleibende Restdauer des Vorgangs. Häufig verfügen bestimmte Anzeigen zusätzlich über eine Prozentangabe. Auf der anderen Seite existieren unbestimmte Fortschrittsanzeigen. Diese erlauben keinerlei Rückschluss auf den aktuellen Fortschritt. Dies kann beispielsweise durch einen Teilbalken, der sich fortwährend von links nach rechts bewegt, realisiert werden.

Trotz der Popularität dieses Anzeigeelements, gibt insgesamt weniger Forschungsergebnisse, bezüglich der Wirksamkeit, als bei anderen Spielelementen \cite{koivisto_rise_2019}.
\citeauthor{olsson_visualisation_2016} kommen in ihrer Arbeit zu dem Ergebnis, dass ein Fortschrittsbalken eine geeignete Maßnahme ist, um den Überblick der Kursteilnehmer in Online-Umgebungen zu verbessern. Gleichzeitig weisen die Autoren auf einen möglichen positiven Einfluss von Abzeichen auf die Motivation der Teilnehmer hin. Fortschrittsbalken lösen ein Gefühl der Vollendung aus und vermitteln daher ein Gefühl der Zufriedenheit \cite{ryan_deci_2000}.


%-------------------------------------------------------------------------------
% Kritik und Einschränkungen
%-------------------------------------------------------------------------------
\subsection{Grenzen}
Trotz der empirisch belegten Wirksamkeit der genannten Spielmechaniken, sind die Ergebnisse nicht immer eindeutig und positiv.
So konnten \citeauthor{ortiz_gamification_2017} durch den Einsatz von Abzeichen zwar eine statistisch signifikante Verbesserung des Engagements der Lernenden feststellen.
Allerdings war keine Verbesserung der Leistung und intrinsischen Motivation erkennbar.
\citeauthor{toda_dark_2018} weißt sogar auf die Gefahr einer Reduzierung der Leistung, unerwünschter Seiteneffekte und potentiellem Motivationsverlust hin.
Zu ähnlichen Ergebnissen kommen \citeauthor{liu_examining_2017}.
So verzeichneten Umfragen mit Fortschrittsbalken geringere Abschlussquoten als Umfragen ohne Fortschrittsindikatoren. \citeauthor{dominguez_gamifying_2013} stellten in einem pädagogischen Kontext fest, dass Abzeichen zwar einen positiven Einfluss auf praktische Aufgaben haben, aber potentiell negativ auf schriftliche Abgaben wirken. Der Einsatz von Spielmechaniken ist damit möglicherweise mit unerwünschten Effekten auf Leistung, Motivation und Engagement verbunden.

Bei der Beurteilung der Wirkung und Effektivität von Gamification werden sehr häufig verschiedene Designelemente kombiniert. \citeauthor{mazarakis2018gamification} weisen darauf hin, dass es nicht möglich ist, die tatsächliche Wirksamkeit individueller Spielelemente zu beurteilen. Es sei nahezu unmöglich den tatsächlichen Beitrag einzelner Elemente an einer Motivationssteigerung zu messen. Aus diesem Grund ist es sinnvoll, einzelne Spielelemente individuell zu untersuchen.

%-------------------------------------------------------------------------------
% Kritik und Einschränkungen
%-------------------------------------------------------------------------------
\subsection{Serious Games}
Serious Games sind Spiele, die nicht ausschließlich der reinen Unterhaltung dienen, sondern primär auf die Vermittlung von Wissen abzielen \cite[S.17]{michael_serious_2005}.
Dabei kommt es darauf an, dass die Spiele mit der Absicht kreiert wurden, dem Spieler einen Lerninhalt zu vermitteln.
Dagegen ist nicht entscheidend, ob der Spieler das Spiel als Lerninhalt versteht oder als reine Unterhaltung \cite[S.3]{bopp_serious_2009}.
Serious Games lassen sich grob in die Kategorien Educational  Games, 
Corporate  Games,  Health  Games,  Persuasive  Games,  Music  Games  sowie  Virtual  Worlds  und 
Mobile Learning Games unterteilen \cite[S.4]{bopp_serious_2009}.
Für diese Arbeit ist lediglich die Kategorie der Educational  Games relevant.
Bei dieser Art der Serious Games geht es um den pädagogischen Einsatz von Videospielen.
Charakteristisch für Educational  Games ist, dass die Lernerfahrung ein spezifisches Ziel verfolgt \cite{nielsen_overview_2006, bopp_serious_2009}.
Typischerweise besteht ein solches Ziel in der Vermittlung bestimmter Fähigkeiten, wie Algebra, Rechtschreibung und weiteren Grundfertigkeiten.
Derartige Spiele fallen auch unter den Oberbegriff Edutainment \cite{nielsen_overview_2006}.
Diese Art der Spiele bietet den Spielern befriedigende Aufgaben, die zu der Entwicklung von neuen Fähigkeiten und Strategien führt \cite{stapleton_serious_2004}.
\citeauthor{vlachopoulos_effect_2017} bestätigen die umfassende empirische Evidenz in Bezug auf kognitive Lernergebnisse einschließlich Wissenserwerb, konzeptuelle Anwendung und Inhaltsverständnis.

%-------------------------------------------------------------------------------
% Gamification und Bildung
%-------------------------------------------------------------------------------

\subsection{Gamification und Bildung}
Ein aktives Forschungsfeld ist der Einsatz von Spielmechaniken in der Bildung \cite{ibanez_gamification_2014,landers_enhancing_2017}. Da der Lernerfolg wesentlich durch Motivation, Interesse und Engagement der Lernenden bedingt ist \cite{astin_student_1984}, sind Spielmechaniken ein vielversprechendes Mittel, um die Leistung der Lernenden zu erhöhen.
Dieser Bereich zählt neben Gesundheit und Fitness zu den empirisch am besten untersuchten Anwendungsgebieten in der Gamificationforschung \cite{koivisto_rise_2019}.
So wurden positive Effekte auf Motivation und Leistung mehrfach der Praxis nachgewiesen \cite{ibanez_gamification_2014,hamzah_influence_2015,strmecki_gamification_2015}. \citeauthor{layth_khaleel_empirical_2019} konnten eine Verbesserung der Lernleistung im Kontext von Programmieraufgaben feststellen. Zu ähnlichen Ergebnissen kommen \cite{ortiz_gamification_2017}. Anzumerken ist jedoch, dass Letztere zwar eine Verbesserung des Engagements feststellen konnten, allerdings keine Auswirkungen auf die Motivation ersichtlich war.


%-------------------------------------------------------------------------------
% Fragestellung
%-------------------------------------------------------------------------------

\subsection{Fragestellung und Zielsetzung}

Im Rahmen der Arbeit werden die Spielelemente Abzeichen und Fortschrittsbalken in eine Webanwendung integriert, die der Vermittlung grundlegender Kommandozeilenbefehlen dient.
Anschließend wird der Einfluss der Spielelemente hinsichtlich Motivation und Leistung der Benutzer untersucht.
Die zugrundeliegende Forschungsfrage lautet:

\begin{itemize}
    \item Welchen Einfluss haben die Spielmechaniken Abzeichen und Fortschrittsbalken auf Motivation und Leistung im Kontext der Kommandozeile?
\end{itemize}

Daraus leiten sich die folgenden Hypothesen ab:

\begin{itemize}
\item H1: Probanden, die Abzeichen erhalten, beantworten im Mittel eine höhere Anzahl an Fragen als eine Kontrollgruppe ohne Abzeichen.
\item H2: Probanden, die eine Fortschrittsanzeige sehen, beantworten im Mittel eine höhere Anzahl an Fragen als eine Kontrollgruppe ohne Fortschrittsanzeige.
\end{itemize}

\subsubsection{Zielgruppe}
Die Ergebnisse sind für alle Personen interessant, die Berührungspunkte mit der Kommandozeile haben.
Hervorzuheben sind insbesondere Schüler und Studenten, die keinerlei bestehende Erfahrung mit einer rein textbasierten Nutzerschnittstelle haben.
Die entstehende Anwendung könnte zukünftig dazu dienen, Einsteigern bei dem Erlernen der Kommandozeile zu unterstützen.
Gleichzeitig liefert die Arbeit weitere empirische Daten hinsichtlich der Wirkung von Abzeichen und Fortschrittsanzeige.

\subsubsection{Abgrenzung}
Bisherige Arbeiten haben die Wirksamkeit von Abzeichen und Fortschrittsbalken in den unterschiedlichsten Kontexten empirisch untersucht.
Die Ergebnisse deuten dabei auf ein positives Potential hinsichtlich der Wirkung der Spielelement auf Motivation, Leistung und Durchhaltevermögen hin. Allerdings sind die Ergebnisse nicht immer reproduzierbar und variieren je nach Kontext teilweise stark. Teilweise konnten sogar negative und unerwünschte Nebeneffekte festgestellt werden. Dies verdeutlicht, dass die Implementation von Gamification in jedem Einzelfall individuell zu prüfen und zu evaluieren ist.
Obwohl es eine Vielzahl existierender Serious Games, mit dem Ziel des Erlernens der Kommandozeile, gibt, die diese Spielmechaniken einsetzen, wurde die Wirksamkeit der Maßnahmen bisher noch nicht untersucht. Diese Lücke versucht diese Arbeit zu schließen.