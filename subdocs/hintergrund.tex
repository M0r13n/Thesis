\section{Theoretischer und empirischer Hintergrund}

Unter dem Begriff Gamification wird der Einsatz von Spielmechaniken und Spieldesignelementen in einem spielfremden Kontext verstanden \cite{deterding_game_2011}. Typische Beispiele für solche Spielelemente sind Abzeichen, Forschrittsanzeigen, Levels oder Ranglisten \cite{koch2013gamification}. Praktische Beispiele sind die Lernplattform Duolingo, die dem Erlernen von Sprachen dient, oder der Social-News-Aggregator Reddit. Im Fall von Duolingo schaltet der Nutzer neue Lektionen als Belohnung für den Abschluss bisheriger Lektionen frei. Reddit integriert ein Punktesystem. Nutzer können etwa durch das regelmäßige Posten qualitativer Beiträge \qq{Karmapunkte} sammeln. Zudem ist es möglich Beiträge mit einem \qq{Down-Vote} oder einem \qq{Up-Vote} zu bewerten.
Ziel ist es, dass sich Nutzer freiwillig mehr agieren, um ihr Ansehen in der Community zu steigern.

Die Wirksamkeit von Gamification konnte bereits für unterschiedlichste Einsatzgebiete empirisch nachgewiesen werden \cite{koivisto_rise_2019} und hat sich in der Forschung zur Mensch-Computer-Interaktion etabliert \cite{huotari_defining_2012}. 
Bereiche, in denen die Wirksamkeit von Gamification gezeigt werden konnte, sind Forschung/ Open Science \cite{brauer_erhohung_2019,kidwell_badges_2016}, Gesundheit und Fitness \cite{johnson_gamification_2016}, Marketing \cite{huotari_defining_2012} oder Produktion und Logistik \cite{warmelink_gamification_2018}.

\subsection{Spielelement Abzeichen}
Im Kontext der Gamification bezeichnen Abzeichen digitale, visuelle Artefakte, die dem Nutzer für die Erfüllung definierter Aufgaben verliehen werden \cite{antin_badges_2011}. \citeauthor{antin_badges_2011} unterteilen Abzeichen anhand ihrer sozialpsychologische Funktion in fünf Kategorien: Zielsetzung (Goal setting), Anweisung (Instruction), Reputation (Reputation),
Status/Bestätigung (Status / Affirmation) und Gruppenidentifizierung (Group Identification). 

\paragraph{Zielsetzung}
Abzeichen fordern den Nutzer heraus ein bestimmtes Ziel zu erreichen. Individuen streben selbst dann nach dem Erreichen bestimmter Ziele oder Meilensteine wenn sie selber mit keinerlei physische Gegenleistung erfahren. Die in einem Abzeichen dargestellten Ziele sind nicht immer explizit. Dies passiert wenn nur angegeben ist, wie man ein Abzeichen verdient, oder weil die notwendigen Aktivitäten subjektiv oder unpräzise definiert sind.

\paragraph{Anweisung}
Abzeichen können Anweisungen darüber geben, welche Arten von Aktivität innerhalb eines gegebenen Systems möglich sind. Diese
Funktion ist nützlich, um neue Benutzer in eine bestimmte Richtung zu lenken, aber auch um bestehenden Nutzer neue Wege zu offenbaren. Dabei ist es nicht notwendig, dass Nutzer die Abzeichen tatsächlich erreichen. Alleine das Betrachten der verfügbaren Abzeichen gibt dem Nutzer Aufschluss über geschätzte Aktivitäten.

\paragraph{Reputation}
Abzeichen liefern eine Grundlage für die Reputationsbewertungen einzelner Nutzer. Durch die Kapselung von Interessen, Fachwissen und vergangener Interaktionen helfen Abzeichen bei der Beurteilung der Reputation auf verschiedenen Ebenen. So geben einsehbare Abzeichen einen Einblick in nutzerspezifische Interessen, bieten einen Überblick über Fähigkeiten und Fachwissen und dienen der Einschätzung der Vertrauenswürdigkeit und der Zuverlässigkeit eines Nutzers. 

\paragraph{Status/Bestätigung}
Abzeichen wirken als Statussymbol. Sie verdeutlichen die eigene Leistung und bisherige Errungenschaften ohne explizite Prahlerei.
Die Macht der Statusbelohnungen ergibt sich insbesondere aus der Erwartung, dass andere positiv auf jemanden reagieren, der durch außerordentliche Aktivitäten entsprechende Abzeichen errungen hat. Abzeichen sind auch eine persönliche Bestätigung, da sie an vergangene Errungenschaften erinnern. Sie markieren wichtige Meilensteine und sind ein Beweis für vergangene Erfolge. Sie wirken damit auf eine ähnliche Art und Weise wie Trophäen.


\paragraph{Gruppenidentifizierung}
Abzeichen kommunizieren eine Reihe von gemeinsamen Aktivitäten, die eine Gruppe von Benutzern anhand gemeinsamer Erfahrungen zusammenschweißen. Das Erlangen von Abzeichen kann ein Gefühl der Solidarität vermitteln und die positive Gruppenidentifikation durch die Wahrnehmung der Ähnlichkeit zwischen einem Individuum und der Gruppe erhöhen.\\

Die Wirksamkeit von Abzeichen ist dabei mehrfach empirisch nachgewiesen. Wie \citeauthor{hamari_badges_2017} zeigen konnte, führt die Einführung von Abzeichen zu einer signifikanten Steigerung der Nutzerinteraktion. Einen ähnlich positiver Effekt auf das Lernverhalten von Studenten konnte durch \citeauthor{hamzah_influence_2015} nachgewiesen werden.


\subsection{Spielelement Fortschrittsbalken}
\subsection{Serious Games}

\subsection{Gamification und Bildung}
Ein weiteres aktives Forschungsfeld ist der Einsatz von Spielmechaniken in der Bildung \cite{ibanez_gamification_2014,landers_enhancing_2017}. Da der Lernerfolg wesentlich durch Motivation, Interesse und Engagement der Lernenden bedingt ist \cite{astin_student_1984}, sind Spielmechaniken ein vielversprechendes Mittel, um die Leistung der Lernenden zu erhöhen.
Dieser Bereich zählt neben Gesundheit und Fitness zu den empirisch am besten untersuchten Anwendungsgebieten in der Gamificationforschung \cite{koivisto_rise_2019}.
So wurden positive Effekte auf Motivation und Leistung mehrfach der Praxis nachgewiesen \cite{ibanez_gamification_2014,hamzah_influence_2015,strmecki_gamification_2015}. \cite{layth_khaleel_empirical_2019} konnten eine Verbesserung der Lernleistung im Kontext von Programmieraufgaben feststellen. Zu ähnlichen Ergebnissen kommen \cite{ortiz_gamification_2017}. Anzumerken ist jedoch, dass Letztere zwar eine Verbesserung des Engagements feststellen konnten, allerdings keine Auswirkungen auf die Motivation erkennbar war.

\subsection{Kommandozeile}