\section{Diskussion}
Ausgangspunkt dieser Arbeit war die Frage, ob der Einsatz von Gamification die Motivation zur Nutzung der Kommandozeile erhöht. Um diese Frage zu beantworten wurde ein virtuelles Terminal entwickelt, welches in der Bedienung und Gestaltung einem klassischen Terminal bis ins Detail genau entspricht. Auf Basis der so entstandenen Webanwendung wurde ein Quiz entwickelt, welches aus einer Reihe an Fragen bestand, die es durch die Eingabe gängiger Konsolenbefehle zu lösen galt. Das Terminal einzelner Nutzer wurde dabei zufällig durch die Spielelemente Abzeichen und Fortschrittsanzeige erweitert. Die so entstandene Anwendung wurde anschließend online verfügbar gemacht und auf unterschiedlichen Plattformen beworben. Die während des Experiments gesammelten Daten bilden die Grundlage für die Beantwortung der übergeordneten Fragestellung. 

Die Ergebnisse zeigen zunächst kein klares Bild und sind teilweise widersprüchlich. Bei einer Gesamtbetrachtung der Stichprobe ist kein motivierender Einfluss durch die Spielelemente Abzeichen und Fortschrittsanzeige erkennbar. Die verschiedenen gamifizierten Versuchsbedingungen unterscheiden sich dabei hinsichtlich Gesamtspielzeit, Anzahl genutzter Befehle oder Anzahl gelöster Aufgaben nicht signifikant von der Kontrollgruppe. Es konnten sogar tendenziell negative Effekte des Spielelements Fortschrittsanzeige beobachtet werden. Allerdings gibt es zwischen den einzelnen Teilnehmers stellenweise massive Unterschiede in der Nutzung der Anwendung. Die Gesamtstichprobe zeichnet sich entsprechend durch eine enorme Streuung aus. Insgesamt zeigte sich zudem ein hohes Maß intrinsischer Motivation der Teilnehmer. Diese bestand unabhängig von den Spielelementen und findet sich auch innerhalb der Kontrollgruppe. Dies ist problematisch, da so die Motivation bereits sehr hoch war und die einzelnen Versuchsbedingungen wenig Potential der Steigerung bieten konnten.

Eine detailliertere Betrachtung der Stichprobe zeigt, dass junge Teilnehmer mit mäßiger bestehender Erfahrung in der Nutzung der Kommandozeile weniger stark motiviert sind. Demgegenüber zeigten sich Teilnehmer mittleren Alters mit sehr guten Englischkenntnissen und deutlicher Vorerfahrung in der Nutzung des Terminals wesentlich motivierter. Bei entsprechenden Teilnehmern zeigt besonders die Fortschrittsanzeige einen statistisch signifikanten motivierenden Effekt. Dies deckt sich mit der Rückmeldung eines Teilnehmers, der keine Fortschrittsanzeige erhielt und bestehende Kenntnisse aufwies: Dieser äußerte, ohne Kenntnis über die Existenz einer Fortschrittsanzeige in anderen Versuchsbedingungen, den Wunsch, einen Fortschrittsbalken zu erhalten, um Rückschlüsse auf die Dauer des Experiments ziehen zu können. Dies hätte nach eigener Aussage seine Motivation das Experiment zu beenden, deutlich erhöht. Neben vereinzelten technischen Unzulänglichkeiten war dies die häufigste Rückmeldung durch Probanden.

Anzumerken ist, dass die Stichprobe, die bis zum 05.09.2020 zu Stande gekommen ist, deutlich von den ursprünglichen Erwartungen abweicht und sich zudem nicht mit den Ergebnissen der Stichprobe deckt, welche danach entstanden ist. Die Ergebnisse weichen dabei stellenweise drastisch von den aufgestellten Hypothesen ab und sind nicht ohne Weiteres erklärbar. Dies lässt die Vermutung zu, dass diese Versuchsgruppe möglicherweise manipuliert wurde und die Ergebnisse nicht repräsentativ sind. Auffällig ist dabei das massiv unterschiedliche Verhalten der Teilnehmer. So zeichnet sich diese Stichprobe durch einen deutlichen Dropout innerhalb der ersten Sekunden aus. Gleichzeitig gibt es einzelne Probanden, die extrem viel Zeit in das Experiment investierten und stellenweise mehrere Stunden für ein Experiment aufbrachten, für das Teilnehmer im Durchschnitt nur wenige Minuten aufbrachten. Zusätzlich fällt in dieser Stichprobe das signifikante schlechte Abschneiden der Fortschrittsanzeige auf.

Da Werbung und Verbreitung zunächst hauptsächlich im privaten Umfeld geschah, kann von einer persönlichen Nähe der Probanden und des Erstellers des Experiments ausgegangen werden. Diese kann potentiell zu dem Vorhandensein von Pflichtgefühlen oder persönlicher Zuneigung führen. Daraus könnte eine von den Spielelementen unabhängige intrinsische Motivation vorhanden gewesen sein. Der starke Abfall in den ersten Sekunden, kann damit zusammenhängen, dass auf diese Weise auch Teilnehmer aus dem privaten Umfeld das Experiment starteten, die eigentlich keine Berührungspunkte mit der Kommandozeile haben. Beispiele sind etwa Personen aus dem familiären Umfeld oder Freunde, die auf das Experiment aufmerksam geworden sind. Entsprechende Teilnehmer könnten aus Zuneigung das Quiz gestartet haben und aus Mangel an Erfahrung, Expertise oder Interesse an der Kommandozeile das Experiment vorzeitig beendet haben. Diese Vermutung wird durch die demographische Zusammensetzung der Stichprobe bestärkt: Probanden der ersten Stichprobe sind durchschnittlich deutlich jünger und haben weniger bestehende Erfahrung mit der Kommandozeile als Teilnehmer, die im späteren Verlauf des Experiments dazu kamen.

Das katastrophale Abschneiden des Spielelements Fortschrittsanzeige könnte dadurch erklärt werden, dass potentiell bereits unmotivierte Teilnehmer zusätzlich demotiviert werden, wenn diese auf die Gesamtdauer des Experiments schließen können. Personen, die sich nur eine kurze Interaktion mit dem Experiment als Zeichen der Zuneigung oder als Freundschaftsbeweis versprachen, wurde auf diese Weise möglicherweise abgeschreckt und verließen das Experiment entsprechend vorzeitig. Ein weiterer Faktor könnten technische Probleme gewesen sein. Da die Anwendung für die Browser entwickelt wurde und verschiedene Technologien wie Javascript und CSS nutze, kann die Darstellung der Anwendung abhängig von dem verwendeten Browser unterschiedlich sein. Insbesondere ältere oder weniger häufig verwendete Browser sind für entsprechende Probleme anfällig. Im Rahmen des Pretests konnte entsprechende Mängel beobachtet und behoben werden. Da dieser zwei technische Einschränkungen enthüllte, die die generelle Benutzbarkeit massiv einschränkten, kann nicht ausgeschlossen werden, dass in Einzelfällen die Fortschrittsanzeige unerwünschte Nebeneffekte aufwies. Diese könnten das Nutzererlebnis negativ beeinflusst haben. Tatsächlich zeigt ein Vergleich der User-Agent, dass anfänglich deutlich unterschiedliche Browser und Endgeräte an dem Experiment teilnahmen.  

\subsection{Ausblick}
Diese Arbeit konnte zeigen, dass es eine isolierte Erfassung motivierender Einflüsse auf das Nutzerverhalten nicht immer einfach ist. Motivation ergibt sich aus unterschiedlichen individuellen Faktoren, die nicht immer eindeutig sind. Dennoch konnte ein signifikanter, motivierender Effekt des Spielelements Fortschrittsanzeige bei Teilnehmern mittleren Alters beobachtet werden, die regelmäßig Kontakt mit dem Terminal haben. Dies verdeutlicht sowohl das Potential, dass in der Gamification steckt, als auch die enorme Wichtigkeit von guter, wissenschaftlicher Arbeit. Durch letztere ist es uns möglich, diejenigen Spielelemente zu identifizieren, die für den jeweiligen Kontext wirklich funktionieren und erlaubt es uns, motivierendere, nutzerzentrierte Anwendungen zu entwickeln.