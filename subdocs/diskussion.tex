\section{Diskussion}
Ausgangspunkt dieser Arbeit war die Frage, ob der Einsatz von Gamification die Motivation zur Nutzung der Kommandozeile erhöhen kann. Betrachtet man die Stichprobe als Ganzes ist kein motivierender Einfluss durch die Spielelemente Abzeichen und Fortschrittsanzeige feststellbar. Die verschiedenen gamifizierten Versuchsbedingungen unterscheiden sich hinsichtlich Gesamtspielzeit, Anzahl genutzter Befehle oder Anzahl gelöster Aufgaben nicht signifikant von der Kontrollgruppe. Es konnten sogar tendenziell negative - wenngleich nicht signifikante - Effekte des Spielelements Fortschrittsanzeige beobachtet werden. Außerdem kennzeichnet sich der vorliegende Datensatz durch eine markante Streuung und ein hohes Maß intrinsischer Motivation einzelner Teilnehmender. Letzteres bestand unabhängig von den Spielelementen und findet sich auch in der Kontrollgruppe. Diese Beobachtung ist problematisch, da so die Motivation einzelner Versuchspersonen bereits hoch war und die Spielelemente wenig Potential der Steigerung boten.

Auffällig ist, dass die Stichprobe, die bis zum 05.09.2020 entstanden ist, deutlich von den ursprünglichen Erwartungen abweicht und sich nicht mit den Ergebnissen der zweiten Stichprobe deckt. Die Ergebnisse in den beiden Stichproben weichen dabei stellenweise deutlich voneinander ab. Dies lässt die Vermutung zu, dass die erste Versuchsgruppe möglicherweise manipuliert wurde und die Ergebnisse nicht repräsentativ sind. Auffällig ist außerdem die hohe Anzahl von Versuchspersonen, die überdurchschnittlich viel Zeit in das Experiment investierten und stellenweise mehrere Stunden mit einem Experiment verbrachten, für das Teilnehmende durchschnittlich nur wenige Minuten aufbrachten. Zusätzlich fällt in der ersten Stichprobe das signifikant schlechte Abschneiden der Versuchsbedingung mit Fortschrittsanzeige auf.

Da Werbung und Verbreitung zunächst hauptsächlich im privaten Umfeld des Autors geschah, kann von einer persönlichen Nähe der Versuchspersonen und des Autors ausgegangen werden. Dies könnte zu dem Vorhandensein von Pflichtgefühlen oder persönlicher Zuneigung geführt haben, die eine von den Spielelementen unabhängige intrinsische Motivation bedingt haben. Das markante Ausscheiden von Versuchspersonen in den ersten Sekunden des Experiments, könnte damit zusammenhängen, dass Teilnehmende aus dem privaten Umfeld das Experiment starteten, die normalerweise keinerlei bestehende Berührungspunkte mit der Kommandozeile haben. Beispiel hierfür sind Personen aus dem familiären Umfeld oder Freunde, die das Experiment aus Zuneigung starteten und aufgrund von Mangel an Erfahrung mit der Kommandozeile das Experiment vorzeitig beendeten. Diese Vermutung wird durch die demographische Zusammensetzung der Stichprobe bestärkt: Versuchspersonen der ersten Stichprobe sind durchschnittlich deutlich jünger und haben weniger bestehende Erfahrung mit der Kommandozeile als Personen aus der zweiten Stichprobe.

Das negative Abschneiden des Spielelements Fortschrittsanzeige in der ersten Stichprobe könnte dadurch erklärt werden, dass potentiell bereits unmotivierte Teilnehmende zusätzlich demotiviert werden, wenn diese auf die Gesamtdauer des Experiments schließen können. Personen, die sich nur eine kurze Interaktion mit dem Experiment versprachen, wurde auf diese Weise möglicherweise abgeschreckt und verließen das Experiment vorzeitig. Ein weiterer Faktor könnten technische Probleme gewesen sein. Da die Anwendung für Browser entwickelt wurde und verschiedene Technologien wie Javascript und CSS nutze, kann die Darstellung der Anwendung abhängig von dem verwendeten Browser unterschiedlich sein. Insbesondere ältere oder selten verwendete Browser sind für Probleme dieser Art anfällig. Im Rahmen des Pretests konnten zwei technische Einschränkungen entdeckt werden, die die generelle Benutzbarkeit massiv einschränkten und es ist möglich, dass weitere technische Mängel bestehen, die die Benutzbarkeit der Anwendung in Einzelfällen beeinträchtigen. Probleme dieser Art könnten das Nutzererlebnis negativ beeinflusst haben und dazu geführt haben, dass Versuchspersonen demotiviert wurden. Diese Vermutung wird durch die Beobachtung gestützt, dass anfänglich häufig ältere oder wenig verbreitete Endgeräte an dem Experiment teilnahmen.

Eine weitere Limitation der Studie ist der überproportional hohe Anteil männlicher Versuchspersonen, der durch die Art und Weise der Verbreitung der Studie zu erklären ist. Das Experiment wurde überwiegend auf Plattformen beworben, die männlich dominiert sind, wodurch es zu einer solchen Verzerrung gekommen ist. Zukünftige Untersuchungen sollten aus diesem Grund gezielt weibliche Versuchspersonen ansprechen. Geeignete Kanäle sind weibliche Kollektive, die über einen technischen Hintergrund verfügen. Beispiele für entsprechende Organisationen sind 'Black Girls Code'\footnote{Link: blackgirlscode.com}, 'Django Girls'\footnote{Link: djangogirls.org} und 'Pink Programming'\footnote{Link: pinkprogramming.se}.

Ein möglicher Kritikpunkt an der Studie könnte sein, dass die Spielelemente zu unauffällig gestaltet wurden und daher die motivierende Wirkung gering ist. Trotz der bereits integrierten Animationen sind weitere Gestaltungsmöglichkeiten durchaus denkbar \textendash{} etwa eine auffällige Overlay-Animation, die den eigentlich Inhalt der Anwendung überdeckt. Hier kann das Live-Streaming-Videoportal Twitch als Vorbild dienen.

Trotzdem kann eine weiterführende Analyse einen statistisch signifikanten, motivierenden Effekt durch das Spielelement Fortschrittsanzeige zeigen. Dies deckt sich mit der Rückmeldung einer Versuchsperson, die keine Fortschrittsanzeige erhielt. Die geschilderte Person äußerte - ohne Kenntnis über die Existenz einer Fortschrittsanzeige in anderen Versuchsbedingungen - den Wunsch, einen Fortschrittsbalken zu erhalten, um Rückschlüsse auf die Dauer des Experiments ziehen zu können. Dies hätte nach eigener Aussage die Motivation das Experiment zu beenden, deutlich erhöht. Neben der Meldung technischer Probleme war dies die am häufigsten geäußerte Rückmeldung durch Versuchspersonen. 

Diese Arbeit kann zeigen, dass es eine isolierte Erfassung motivierender Einflüsse auf das Nutzerverhalten nicht trivial ist. Motivation ergibt sich aus unterschiedlichen individuellen Faktoren, die es zu isolieren gilt. Dennoch weist die Studie auf einen möglichen motivierenden Effekt des Spielelements Fortschrittsbalken hin. Dies verdeutlicht sowohl das Potential, das in Gamification steckt, als auch die enorme Bedeutung einer sauberen, wissenschaftlichen Arbeitsweise. Durch diese ist es möglich, diejenigen Spielelemente zu identifizieren, die für den jeweiligen Kontext funktionieren und könnte dazu beitragen, motivierende, nutzerzentrierte Anwendungen zu entwickeln.