\section{Einleitung}
Die Kommandozeile ist Teil jedes Betriebssystems und stellt, als rein textbasierte Schnittstelle, die einfachste Form der Mensch-Computer-Interaktion dar \cite{Kumar2005}. Obwohl Kommandozeilen die erste Form der Interaktion mit einem Computer waren und mittlerweile in vielen Bereichen durch grafische Oberflächen ersetzt wurden, sind sie weiterhin elementarer Bestandteil moderner Betriebssysteme. Zahlreiche Anwenderprogramme und Konfigurations/- oder Entwicklungswerkzeuge basieren vollständig oder zumindest teilweise auf einer Kommandozeile. Eine produktive Nutzung derartiger Werkzeuge erfordert Kenntnis über die verfügbaren Befehle und deren Parameter \cite[S.42]{Kumar2005}. Das Erlernen eben dieser kann anfänglich mit einem hohen Maß an Frustration verbunden sein. So ist es wenig verwunderlich, dass es eine Vielzahl unterschiedlicher Spiele gibt, deren Ziel darin besteht, die Grundlagen der Kommandozeile spielend zu vermitteln. Eine kurze Suche\footnote{Link: https://github.com/topics/terminal-game} nach dem Begriff \glqq terminal-game\grqq{} auf der Code-Hosting-Plattform GitHub bringt beispielsweise (Stand Mai  2020)  über  300  Ergebnisse  hervor. 

Diese Art der Spiele wird als Serious Game bezeichnet und gehört thematisch in den Bereich der Gamification. Gamification ist definiert als der Einsatz spieltypischer Elemente in einem spielfremden Kontext \cite{deterding_game_2011} und zielt in aller Regel auf eine Motivationssteigerung ab \cite{takahashi_gamification_2010}. Die  dabei  am  häufigsten  verwendeten  Spielelemente sind u. a. Abzeichen und Fortschrittsbalken. Trotz der offensichtlichen Popularität wurde die Effektivität dieser Spielelemente im Kontext der Kommandozeile bisher noch nicht empirisch untersucht. Das Ziel dieser Arbeit besteht daher in der Analyse der Wirkung der Spielelemente Abzeichen und Fortschrittsanzeige hinsichtlich Motivation und Leistung.

Um diese Frage zu beantworten wird eine interaktive Webanwendung entwickelt. Diese simuliert eine Kommandozeile. Im Rahmen des Experiments wird eine definierte Menge an Fragen gestellt, die die Probanden durch die Nutzung der Kommandozeile lösen sollen. Dabei  werden  in  verschiedenen  Versuchsbedingungen die Spielelemente Abzeichen und Fortschrittsbalken integriert. Die dabei erfassten Daten dienen schließlich der empirischen Beurteilung der Wirksamkeit der genannten Spielelemente.
