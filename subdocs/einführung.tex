\section{Einleitung}
Digitale Fähigkeiten spielen auf dem Arbeitsmarkt eine zunehmend wichtigere Rolle. Ein in Großbritannien durchgeführter Report kommt etwa zu dem Ergebnis, dass 82\% sämtlicher Stellenausschreibungen im Netz digitale Kenntnisse erfordern \cite{no_longer_optional}. Gleichzeitig erhalten Stellen, die digitale Fähigkeiten erfordern, im Mittel ein um 29\% höheres Gehalt als Stellen, die keinerlei entsprechende Anforderungen listen. Zusätzlich senken digitale Fertigkeiten das Risiko für Jobverlust durch Automatisierung um 59\% und wirken positiv auf die individuelle Karriereentwicklung. Unter dem Sammelbegriff digitale Fähigkeiten fallen dabei unterschiedlichste Fertigkeiten, die sich je nach Anforderungsprofil unterscheiden. Diese reichen von simplen EDV-Kenntnissen, etwa der Bedienung von Microsoft-Word, bis hin zu Kenntnissen in der Systemadministration oder Softwareentwicklung. Für die Letztgenannten ist das Beherrschen der Kommandozeile unerlässlich. Diese ist Teil eines jeden Betriebssystems und stellt, als rein textbasierte Schnittstelle, die einfachste Form der Mensch-Computer-Interaktion dar \cite{Kumar2005}. Damit unterscheidet sich die Kommandozeile hinsichtlich Bedienung und Arbeitsweise grundlegend von einer graphischen Oberfläche. Das Erlernen ebendieser Schnittstelle kann anfänglich mit einem hohen Maß an Frustration verbunden sein und stellt damit eine Hürde dar, die es für Anfänger zu überwinden gilt.

Das Überwinden entsprechender Hürden fällt leichter, wenn die Eigenmotivation möglichst hoch ist. Im Kontext der Kommandozeile ist dies insbesondere dann der Fall, wenn sich die Lernenden durch die erworbenen Fähigkeiten verbesserte Einkommensmöglichkeiten und gesteigerte Karrieremöglichkeiten versprechen. Neben solchen monetären Anreizen gibt es auch nicht-monetäre Anreize wie Anerkennung, Selbstverwirklichung oder Beschäftigungssicherheit \cite{shujaat}. 

Eine weitere Möglichkeit zur Erhöhung der Motivation ist Gamification. Mit dem Begriff Gamification wir der Einsatz von Spieldesignelementen in einem spielfremden Kontext bezeichnet \cite{deterding_game_2011}. Häufig verwendete Spieldesignelemente sind Abzeichen, Rankings, Levels und Fortschrittsanzeigen \cite{koch2013gamification}. Die Wirksamkeit dieser Spielelemente wurde bereits mehrfach empirisch nachgewiesen \cite{koivisto_rise_2019}. Jedoch gibt es zum aktuellen Zeitpunkt keinerlei wissenschaftliche Daten für die Wirkungsweise im Kontext einer Kommandozeile. Aus diesem Grund werde ich im Rahmen dieser Arbeit den Einfluss der Spielelemente Abzeichen und Fortschrittsanzeige auf Motivation und Leistung im Kontext der Kommandozeile analysieren. Dazu entwerfe ich eine interaktive Webanwendung, die einer klassischen Kommandozeile gleicht und integriere die genannten Spielelemente. 